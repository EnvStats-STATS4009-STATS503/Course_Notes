% Options for packages loaded elsewhere
\PassOptionsToPackage{unicode}{hyperref}
\PassOptionsToPackage{hyphens}{url}
\PassOptionsToPackage{dvipsnames,svgnames,x11names}{xcolor}
%
\documentclass[
  letterpaper,
  DIV=11,
  numbers=noendperiod]{scrartcl}

\usepackage{amsmath,amssymb}
\usepackage{iftex}
\ifPDFTeX
  \usepackage[T1]{fontenc}
  \usepackage[utf8]{inputenc}
  \usepackage{textcomp} % provide euro and other symbols
\else % if luatex or xetex
  \usepackage{unicode-math}
  \defaultfontfeatures{Scale=MatchLowercase}
  \defaultfontfeatures[\rmfamily]{Ligatures=TeX,Scale=1}
\fi
\usepackage{lmodern}
\ifPDFTeX\else  
    % xetex/luatex font selection
\fi
% Use upquote if available, for straight quotes in verbatim environments
\IfFileExists{upquote.sty}{\usepackage{upquote}}{}
\IfFileExists{microtype.sty}{% use microtype if available
  \usepackage[]{microtype}
  \UseMicrotypeSet[protrusion]{basicmath} % disable protrusion for tt fonts
}{}
\makeatletter
\@ifundefined{KOMAClassName}{% if non-KOMA class
  \IfFileExists{parskip.sty}{%
    \usepackage{parskip}
  }{% else
    \setlength{\parindent}{0pt}
    \setlength{\parskip}{6pt plus 2pt minus 1pt}}
}{% if KOMA class
  \KOMAoptions{parskip=half}}
\makeatother
\usepackage{xcolor}
\setlength{\emergencystretch}{3em} % prevent overfull lines
\setcounter{secnumdepth}{5}
% Make \paragraph and \subparagraph free-standing
\makeatletter
\ifx\paragraph\undefined\else
  \let\oldparagraph\paragraph
  \renewcommand{\paragraph}{
    \@ifstar
      \xxxParagraphStar
      \xxxParagraphNoStar
  }
  \newcommand{\xxxParagraphStar}[1]{\oldparagraph*{#1}\mbox{}}
  \newcommand{\xxxParagraphNoStar}[1]{\oldparagraph{#1}\mbox{}}
\fi
\ifx\subparagraph\undefined\else
  \let\oldsubparagraph\subparagraph
  \renewcommand{\subparagraph}{
    \@ifstar
      \xxxSubParagraphStar
      \xxxSubParagraphNoStar
  }
  \newcommand{\xxxSubParagraphStar}[1]{\oldsubparagraph*{#1}\mbox{}}
  \newcommand{\xxxSubParagraphNoStar}[1]{\oldsubparagraph{#1}\mbox{}}
\fi
\makeatother


\providecommand{\tightlist}{%
  \setlength{\itemsep}{0pt}\setlength{\parskip}{0pt}}\usepackage{longtable,booktabs,array}
\usepackage{calc} % for calculating minipage widths
% Correct order of tables after \paragraph or \subparagraph
\usepackage{etoolbox}
\makeatletter
\patchcmd\longtable{\par}{\if@noskipsec\mbox{}\fi\par}{}{}
\makeatother
% Allow footnotes in longtable head/foot
\IfFileExists{footnotehyper.sty}{\usepackage{footnotehyper}}{\usepackage{footnote}}
\makesavenoteenv{longtable}
\usepackage{graphicx}
\makeatletter
\newsavebox\pandoc@box
\newcommand*\pandocbounded[1]{% scales image to fit in text height/width
  \sbox\pandoc@box{#1}%
  \Gscale@div\@tempa{\textheight}{\dimexpr\ht\pandoc@box+\dp\pandoc@box\relax}%
  \Gscale@div\@tempb{\linewidth}{\wd\pandoc@box}%
  \ifdim\@tempb\p@<\@tempa\p@\let\@tempa\@tempb\fi% select the smaller of both
  \ifdim\@tempa\p@<\p@\scalebox{\@tempa}{\usebox\pandoc@box}%
  \else\usebox{\pandoc@box}%
  \fi%
}
% Set default figure placement to htbp
\def\fps@figure{htbp}
\makeatother

% load packages
\usepackage{geometry}
\usepackage{xcolor}
\usepackage{eso-pic}
\usepackage{fancyhdr}
\usepackage{sectsty}
\usepackage{fontspec}
\usepackage{titlesec}

%% Set page size with a wider right margin
\geometry{a4paper, total={170mm,257mm}, left=20mm, top=20mm, bottom=20mm, right=50mm}

%% Let's define some colours
\definecolor{uniblue}{HTML}{003865}
\definecolor{burgundy}{HTML}{7D2239}
\definecolor{cobalt}{HTML}{005C8A}
\definecolor{lavender}{HTML}{5B4D94}
\definecolor{leaf}{HTML}{006630}
\definecolor{moss}{HTML}{385A4F}
\definecolor{pillarbox}{HTML}{B30C00}
\definecolor{rust}{HTML}{9A3A06}
\definecolor{sandstone}{HTML}{52473B}
\definecolor{skyblue}{HTML}{005398}
\definecolor{slate}{HTML}{4F5961}
\definecolor{thistle}{HTML}{951272}

%\definecolor{light}{HTML}{E6E6FA} % original from template - redefined below as uni blue at 10 percent:
\colorlet{light}{uniblue!10}
%\definecolor{highlight}{HTML}{800080} % original from template - redefined below as uni's skyblue:
\colorlet{highlight}{skyblue}
%\definecolor{dark}{HTML}{330033} % original from template - redefined below as uni blue at 100 percent:
\colorlet{dark}{uniblue}

%% Let's add the border on the right hand side 
\AddToShipoutPicture{% 
    \AtPageLowerLeft{% 
        \put(\LenToUnit{\dimexpr\paperwidth-3cm},0){% 
            \color{light}\rule{3cm}{\LenToUnit\paperheight}%
          }%
     }%
     % logo
    \AtPageLowerLeft{% start the bar at the bottom right of the page
        \put(\LenToUnit{\dimexpr\paperwidth-2.25cm},27.2cm){% move it to the top right
            \color{light}\includegraphics[width=2.25cm]{_extensions/nrennie/PrettyPDF/uni_logo_boxed.jpg}
          }%
     }%
}

%% Style the page number
\fancypagestyle{mystyle}{
  \fancyhf{}
  \renewcommand\headrulewidth{0pt}
  \fancyfoot[R]{\thepage}
  \fancyfootoffset{3.5cm}
}
\setlength{\footskip}{20pt}

%% style the chapter/section fonts
\chapterfont{\color{uniblue}\fontsize{20}{16.8}\selectfont}
\sectionfont{\color{uniblue}\fontsize{20}{16.8}\selectfont}
\subsectionfont{\color{skyblue}\fontsize{14}{16.8}\selectfont}
\titleformat{\subsection}
  {\color{uniblue!90}\sffamily\Large\bfseries}{\thesubsection}{1em}{}[{\titlerule[0.8pt]}]
\subsubsectionfont{\color{cobalt}}

\renewcommand\thesection{\color{slate}\arabic{section}}
  
% left align title
\makeatletter
\renewcommand{\maketitle}{\bgroup\setlength{\parindent}{0pt}
\begin{flushleft}
  {\color{uniblue}\sffamily\huge\textbf{\@title}} \vspace{0.3cm} \newline
  {\Large {\@subtitle}} \newline
  \@author
\end{flushleft}\egroup
}
\makeatother

%%% Use some custom fonts
\setsansfont{Ubuntu}[
    Path=_extensions/nrennie/PrettyPDF/Ubuntu/,
    Scale=0.9,
    Extension = .ttf,
    UprightFont=*-Regular,
    BoldFont=*-Bold,
    ItalicFont=*-Italic,
    ]

\setmainfont{Ubuntu}[
    Path=_extensions/nrennie/PrettyPDF/Ubuntu/,
    Scale=0.9,
    Extension = .ttf,
    UprightFont=*-Regular,
    BoldFont=*-Bold,
    ItalicFont=*-Italic,
    ]
\KOMAoption{captions}{tableheading}
\makeatletter
\@ifpackageloaded{tcolorbox}{}{\usepackage[skins,breakable]{tcolorbox}}
\@ifpackageloaded{fontawesome5}{}{\usepackage{fontawesome5}}
\definecolor{quarto-callout-color}{HTML}{909090}
\definecolor{quarto-callout-note-color}{HTML}{0758E5}
\definecolor{quarto-callout-important-color}{HTML}{CC1914}
\definecolor{quarto-callout-warning-color}{HTML}{EB9113}
\definecolor{quarto-callout-tip-color}{HTML}{00A047}
\definecolor{quarto-callout-caution-color}{HTML}{FC5300}
\definecolor{quarto-callout-color-frame}{HTML}{acacac}
\definecolor{quarto-callout-note-color-frame}{HTML}{4582ec}
\definecolor{quarto-callout-important-color-frame}{HTML}{d9534f}
\definecolor{quarto-callout-warning-color-frame}{HTML}{f0ad4e}
\definecolor{quarto-callout-tip-color-frame}{HTML}{02b875}
\definecolor{quarto-callout-caution-color-frame}{HTML}{fd7e14}
\makeatother
\makeatletter
\@ifpackageloaded{caption}{}{\usepackage{caption}}
\AtBeginDocument{%
\ifdefined\contentsname
  \renewcommand*\contentsname{Table of contents}
\else
  \newcommand\contentsname{Table of contents}
\fi
\ifdefined\listfigurename
  \renewcommand*\listfigurename{List of Figures}
\else
  \newcommand\listfigurename{List of Figures}
\fi
\ifdefined\listtablename
  \renewcommand*\listtablename{List of Tables}
\else
  \newcommand\listtablename{List of Tables}
\fi
\ifdefined\figurename
  \renewcommand*\figurename{Figure}
\else
  \newcommand\figurename{Figure}
\fi
\ifdefined\tablename
  \renewcommand*\tablename{Table}
\else
  \newcommand\tablename{Table}
\fi
}
\@ifpackageloaded{float}{}{\usepackage{float}}
\floatstyle{ruled}
\@ifundefined{c@chapter}{\newfloat{codelisting}{h}{lop}}{\newfloat{codelisting}{h}{lop}[chapter]}
\floatname{codelisting}{Listing}
\newcommand*\listoflistings{\listof{codelisting}{List of Listings}}
\makeatother
\makeatletter
\makeatother
\makeatletter
\@ifpackageloaded{caption}{}{\usepackage{caption}}
\@ifpackageloaded{subcaption}{}{\usepackage{subcaption}}
\makeatother
\makeatletter
\@ifpackageloaded{tcolorbox}{}{\usepackage[skins,breakable]{tcolorbox}}
\makeatother
\makeatletter
\@ifundefined{shadecolor}{\definecolor{shadecolor}{rgb}{.97, .97, .97}}{}
\makeatother
\makeatletter
\@ifundefined{codebgcolor}{\definecolor{codebgcolor}{named}{light}}{}
\makeatother
\makeatletter
\ifdefined\Shaded\renewenvironment{Shaded}{\begin{tcolorbox}[sharp corners, breakable, boxrule=0pt, frame hidden, colback={codebgcolor}, enhanced]}{\end{tcolorbox}}\fi
\makeatother

\usepackage{bookmark}

\IfFileExists{xurl.sty}{\usepackage{xurl}}{} % add URL line breaks if available
\urlstyle{same} % disable monospaced font for URLs
\hypersetup{
  pdftitle={Tutorial Sheet 2},
  colorlinks=true,
  linkcolor={highlight},
  filecolor={Maroon},
  citecolor={Blue},
  urlcolor={highlight},
  pdfcreator={LaTeX via pandoc}}


\title{Tutorial Sheet 2}
\author{}
\date{}

\begin{document}
\maketitle

\pagestyle{mystyle}


\section{Journal club activity}\label{journal-club-activity}

This week we will focus on a reading group to discuss the following
paper:

\begin{itemize}
\tightlist
\item
  Linder, H. L., \& Horne, J. K. (2018). Evaluating statistical models
  to measure environmental change: A tidal turbine case study.
  \emph{Ecological Indicators}, \emph{84}, 765-792.

  \begin{itemize}
  \tightlist
  \item
    The paper can be accesses through the following DOI:
    \url{https://doi.org/10.1016/j.ecolind.2017.09.041}
  \item
    Or downloaded using the following link:
  \end{itemize}
\end{itemize}

This week, we will focus on how to read a methodological paper in
applied ecology. The \textbf{Goal of the Session} is to evaluate and
critique the \emph{statistical methods} presented in a scientific paper
related to an environmental problem. We will focus on
\emph{understanding} the author's framework, assessing the practical
\emph{implications} of their findings, and identifying the validity of
the conclusions drawn from the study.

\begin{tcolorbox}[enhanced jigsaw, rightrule=.15mm, breakable, leftrule=.75mm, bottomtitle=1mm, toprule=.15mm, opacityback=0, left=2mm, colback=white, opacitybacktitle=0.6, title=\textcolor{quarto-callout-note-color}{\faInfo}\hspace{0.5em}{Note}, colframe=quarto-callout-note-color-frame, coltitle=black, toptitle=1mm, titlerule=0mm, arc=.35mm, bottomrule=.15mm, colbacktitle=quarto-callout-note-color!10!white]

This tutorial is designed as direct practice for your final exam, which
will require you to write a critical essay on a given scientific topic.
The skills practiced here (e.g., deconstructing a paper's core argument,
evaluating its methodology, and synthesizing its contributions) are
precisely the skills that will help you with that task. Think of this as
a collaborative training session.

\end{tcolorbox}

\subsection{☑️ Pre-session work}\label{pre-session-work}

\textbf{Your Preparation (Please complete BEFORE the session):}

\begin{enumerate}
\def\labelenumi{\arabic{enumi}.}
\item
  \textbf{Read Strategically:} Don't get bogged down in every
  statistical detail on the first pass. Focus on understanding the
  \textbf{narrative}.

  \begin{itemize}
  \item
    \textbf{Abstract \& Introduction:} What is the context of the
    problem? What is the core problem they are solving, and why is it
    important?
  \item
    \textbf{Section 2 (Data):} Why have the authors selected this
    specific case study? How is data been collected and can you identify
    any potential sources of bias?
  \item
    \textbf{Section 2 (Methods):} What models have the authors used and
    why? how are these methods being compared? Understand the principles
    of the intervention analysis and how the model's ability to detect
    change has been measured? Can you think of any caveats on the usage
    of these metrics?
  \item
    \textbf{Section 3 (Results):} Focus on the \textbf{take-home
    messages} from tables and figures. How would you summaries the paper
    key findings?
  \item
    \textbf{Discussion \& Conclusion:} What do the authors claim is
    their key contribution? What are the broader implications? Have you
    identified any limitations with the study?
  \end{itemize}
\item
  \textbf{Take Notes on These Four Key Questions:}

  \begin{itemize}
  \item
    \textbf{The Core Problem:} In your own words, what is the ``gap'' in
    standard monitoring practices that this paper addresses?
  \item
    \textbf{The Evaluation Framework:} How did the authors test the
    models? What were the criteria for ``best''?
  \item
    \textbf{Main Recommendation:} What is the ``best practice'' they
    propose, and does it vary by objective (detect, quantify, forecast)?
  \item
    \textbf{Your Critical Assessment:} What is one major strength of
    this study's approach? What is one potential limitation or remaining
    question you have?
  \end{itemize}
\item
  \textbf{Bring:} Your annotated copy of the paper and your prepared
  notes.
\end{enumerate}

\subsection{\texorpdfstring{👥 \textbf{In-Person Group Activity: Paper
Discussion \& Critical
Review}}{👥 In-Person Group Activity: Paper Discussion \& Critical Review}}\label{in-person-group-activity-paper-discussion-critical-review}

\textbf{Activity Goal:} To collaboratively break down the paper's core
components, evaluate its methodological framework, and share your
insights with the whole class.

Work in small groups to tackle key questions about the paper's argument
and methods. Your tutor will lead the session by posing specific
questions, giving your group time to discuss, and then facilitating a
class-wide conversation to compare insights.

\subsection{📋 For tutors}\label{for-tutors}

\begin{enumerate}
\def\labelenumi{\arabic{enumi}.}
\tightlist
\item
  Ask the students to work on small groups (e.g, 4-5 students per group)
\item
  Begin the warmo-up session with a small group discussion and ask the
  students about what the found most interesting about the paper - give
  them around 5 minutes to discuss and then ask each group to present
  their answers.
\item
  For the following part each group will discuss different sections of
  the paper. This is a rough plan for you to guide this discussion but
  it could (and probably will) be adapted based on the discussion
  student have. Please spend some time discussing some of the ideas with
  each group
\end{enumerate}

\subsubsection{Tutorial structure}\label{tutorial-structure}

Total session: 60minutes

\begin{itemize}
\item
  Intro \& warm-up: 5-10 min
\item
  Group work: 25-30 min
\item
  Group presentations: 25 min per group (5 min per group approx)
\end{itemize}

Each group should prepare:

\begin{enumerate}
\def\labelenumi{\arabic{enumi}.}
\item
  2-3 key insights to share with the class.
\item
  1 remaining question or uncertainty
\item
  A visual aid (sketch/diagram on whiteboard or paper) explaining their
  section of the paper.
\end{enumerate}

\subsubsection{\texorpdfstring{\textbf{Output
Specifications}}{Output Specifications}}\label{output-specifications}

The number of groups depends on the number of students so there could me
more than one group working on a specific topic.

\subsubsection{\texorpdfstring{\textbf{Tutor
Guidance}}{Tutor Guidance}}\label{tutor-guidance}

\textbf{During group work:} - Rotate between groups, listen first before
intervening - Ask probing questions: ``\emph{Why do you think that?}''
``\emph{What evidence supports that?}'' - Try that all students
participate

\textbf{During presentations:} - Connect different groups' insights:
``\emph{Group 2's point about simulation relates to what Group 4
found\ldots{}}'' - Synthesize conflicting interpretations - Highlight
particularly insightful critiques

\begin{tcolorbox}[enhanced jigsaw, rightrule=.15mm, breakable, leftrule=.75mm, bottomtitle=1mm, toprule=.15mm, opacityback=0, left=2mm, colback=white, opacitybacktitle=0.6, title=\textcolor{quarto-callout-note-color}{\faInfo}\hspace{0.5em}{Note}, colframe=quarto-callout-note-color-frame, coltitle=black, toptitle=1mm, titlerule=0mm, arc=.35mm, bottomrule=.15mm, colbacktitle=quarto-callout-note-color!10!white]

\begin{itemize}
\item
  \textbf{For students who finish early:} - You can ask the to compare
  this paper's methods to another paper they've read or that is cited on
  the same paper. Ask them to sketch how they'd apply these methods if
  they were hired for example as statistics consultants - ask them to
  identify one ore more concept you'd like to learn more about.
\item
  \textbf{For advanced groups:} Challenge them to identify what's
  \emph{missing} from the paper
\item
  \textbf{For struggling groups:} Provide more directed questions or a
  summary template
\item
  \textbf{Mixed-skill groups:} Assign roles within groups (summarizer,
  questioner, connector, presenter)
\end{itemize}

\end{tcolorbox}

\subsubsection{Differentiation
Strategies}\label{differentiation-strategies}

\begin{itemize}
\tightlist
\item
  \textbf{For quiet groups:} Use think-pair-share or assign specific
  speaking roles
\item
  \textbf{For fast groups:} Add extension questions about broader
  implications
\item
  \textbf{For technical confusion:} Clarify one key concept, then let
  them apply it
\end{itemize}

\subsubsection{🔄 Alternative Structure}\label{alternative-structure}

You can also consider a ``\emph{jigsaw}'' approach if things are moving
fast where:

\begin{enumerate}
\def\labelenumi{\arabic{enumi}.}
\item
  Original groups become experts on their section
\item
  Regroup so new groups have one expert from each section
\item
  Experts teach their section to their new group
\item
  Then each group can take on a ``consultancy role'' scenario. E.g.,
  tell the students : ``\emph{You're a consulting firm hired to evaluate
  this research for a client}''
\end{enumerate}

Then each member of the team is assingned a role:

\begin{itemize}
\item
  Project Manager (oversees)
\item
  Statistician (methods expert)
\item
  Subject Expert (domain knowledge)
\item
  Communications Officer (presentation)
\end{itemize}

\begin{enumerate}
\def\labelenumi{\arabic{enumi}.}
\setcounter{enumi}{4}
\tightlist
\item
  Then, on the \textbf{Consultation Phase} each student analyze paper
  from their role's perspective.
\item
  Finally, students finish the session with a ``\emph{Client Meeting}''
  where they present their findings to the whole group.
\end{enumerate}

Time structure for jigsaw activity:

\begin{itemize}
\item
  Expert groups: 15 min approx
\item
  Teaching groups \& Consultancy role-play: 15 min approx
\item
  Client meeting/group presentation: 25 (5 min per group approx)
\end{itemize}

\subsubsection{Group Topics}\label{group-topics}

Here are some of examples of the focus- groups:

\begin{tcolorbox}[enhanced jigsaw, rightrule=.15mm, breakable, leftrule=.75mm, bottomtitle=1mm, toprule=.15mm, opacityback=0, left=2mm, colback=white, opacitybacktitle=0.6, title={Example: Group 1}, colframe=quarto-callout-tip-color-frame, coltitle=black, toptitle=1mm, titlerule=0mm, arc=.35mm, bottomrule=.15mm, colbacktitle=quarto-callout-tip-color!10!white]

\textbf{Topic}: \textbf{General summary of the paper, background and
data.}

Students should provide (i) an overall summary of the paper highlighting
the main findings and (ii) discuss the context of the problem in detail
(i.e.~sections 1 \& 2.1 , 2.2) . Questions to motivate the discussion
can be:

\begin{itemize}
\tightlist
\item
  What is the purpose of the study?
\item
  What is the motivation behind the case study the authors presented and
  is the explanation about the data collection process clear? Can you
  think of any potential sources of bias that the original paper has not
  considered?
\item
  Is the choice of statistical methods well-justified? Would you do
  things differently?
\end{itemize}

\end{tcolorbox}

\begin{tcolorbox}[enhanced jigsaw, rightrule=.15mm, breakable, leftrule=.75mm, bottomtitle=1mm, toprule=.15mm, opacityback=0, left=2mm, colback=white, opacitybacktitle=0.6, title={Example: Group 2}, colframe=quarto-callout-tip-color-frame, coltitle=black, toptitle=1mm, titlerule=0mm, arc=.35mm, bottomrule=.15mm, colbacktitle=quarto-callout-tip-color!10!white]

\textbf{Topic}: \textbf{Simulated scenarios.}

Students should discuss the baseline simulation models and change
scenarios (sections 2.3 thru 2.6) . Questions to motivate the discussion
can be:

\begin{itemize}
\tightlist
\item
  Is the explanation about the baseline simulation models clear? discuss
  how authors have chosen the amount of observation error .
\item
  Performance of candidate models has been tested using
  ``\emph{Before-After simulated datasets}'' - what problem you might
  face in a real-world data scenario? If you had unlimited resources
  would you propose a different design? (e.g., student can discuss BACI)
\item
  Explain and discuss Table 1 in detail.
\end{itemize}

\end{tcolorbox}

\begin{tcolorbox}[enhanced jigsaw, rightrule=.15mm, breakable, leftrule=.75mm, bottomtitle=1mm, toprule=.15mm, opacityback=0, left=2mm, colback=white, opacitybacktitle=0.6, title={Example: Group 3}, colframe=quarto-callout-tip-color-frame, coltitle=black, toptitle=1mm, titlerule=0mm, arc=.35mm, bottomrule=.15mm, colbacktitle=quarto-callout-tip-color!10!white]

\textbf{Topic}: \textbf{Modelling approaches}

Students should discuss the candidate models to detect change and how
are they being assessed (sections 2.7 thru 2.10) . Questions to motivate
the discussion can be:

\begin{itemize}
\tightlist
\item
  Are the choice of statistical model well-justified? Discuss how the
  authors have presented the different candidate models, do you think is
  a clear manner of explaining this?
\item
  How would the Intervention analysis presented in section 2.8. to a
  non-statistician (e.g., the general public)
\item
  Explain the power analysis that was conducted to evaluate model
  ability to detect change.
\item
  Summarise the different model fit and forecast accuracy metrics used
  by the authors.
\end{itemize}

\end{tcolorbox}

\begin{tcolorbox}[enhanced jigsaw, rightrule=.15mm, breakable, leftrule=.75mm, bottomtitle=1mm, toprule=.15mm, opacityback=0, left=2mm, colback=white, opacitybacktitle=0.6, title={Example: Group 4}, colframe=quarto-callout-tip-color-frame, coltitle=black, toptitle=1mm, titlerule=0mm, arc=.35mm, bottomrule=.15mm, colbacktitle=quarto-callout-tip-color!10!white]

\textbf{Topic}: \textbf{Results}

Students should discuss the results of the paper (section 3) . Questions
to motivate the discussion can be:

\begin{itemize}
\tightlist
\item
  What are the main findings reported on this section?
\item
  Students should explain to their classmates the figures presented in
  this section and what are the implications for the study.
\item
  Then, ask them how would they communicate these results to the general
  public? Imagine they are working for a consultancy company and they
  have to explain this to a group of environmental scientists and
  conservationists.
\end{itemize}

\end{tcolorbox}

\begin{tcolorbox}[enhanced jigsaw, rightrule=.15mm, breakable, leftrule=.75mm, bottomtitle=1mm, toprule=.15mm, opacityback=0, left=2mm, colback=white, opacitybacktitle=0.6, title={Example: Group 5}, colframe=quarto-callout-tip-color-frame, coltitle=black, toptitle=1mm, titlerule=0mm, arc=.35mm, bottomrule=.15mm, colbacktitle=quarto-callout-tip-color!10!white]

\textbf{Topic}: \textbf{Discussion and recommendations}

Students should discuss Section 4 \& 5 . Questions to motivate the
discussion can be:

\begin{itemize}
\tightlist
\item
  Summarise the main points the authors have discussed? What have the
  authors recommended on each scenario and why? Do you agree with the
  authors recommendation?
\item
  Discuss figure 9. what do you think of it? would you presented the
  recommended models differently?
\item
  How valid are the conclusions drawn from this study? Can you think a
  situation in which authors recommendation would be difficult or even
  impossible to implement?
\end{itemize}

\end{tcolorbox}




\end{document}
