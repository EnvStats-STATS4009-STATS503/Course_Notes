% Options for packages loaded elsewhere
\PassOptionsToPackage{unicode}{hyperref}
\PassOptionsToPackage{hyphens}{url}
\PassOptionsToPackage{dvipsnames,svgnames,x11names}{xcolor}
%
\documentclass[
  letterpaper,
  DIV=11,
  numbers=noendperiod]{scrartcl}

\usepackage{amsmath,amssymb}
\usepackage{iftex}
\ifPDFTeX
  \usepackage[T1]{fontenc}
  \usepackage[utf8]{inputenc}
  \usepackage{textcomp} % provide euro and other symbols
\else % if luatex or xetex
  \usepackage{unicode-math}
  \defaultfontfeatures{Scale=MatchLowercase}
  \defaultfontfeatures[\rmfamily]{Ligatures=TeX,Scale=1}
\fi
\usepackage{lmodern}
\ifPDFTeX\else  
    % xetex/luatex font selection
\fi
% Use upquote if available, for straight quotes in verbatim environments
\IfFileExists{upquote.sty}{\usepackage{upquote}}{}
\IfFileExists{microtype.sty}{% use microtype if available
  \usepackage[]{microtype}
  \UseMicrotypeSet[protrusion]{basicmath} % disable protrusion for tt fonts
}{}
\makeatletter
\@ifundefined{KOMAClassName}{% if non-KOMA class
  \IfFileExists{parskip.sty}{%
    \usepackage{parskip}
  }{% else
    \setlength{\parindent}{0pt}
    \setlength{\parskip}{6pt plus 2pt minus 1pt}}
}{% if KOMA class
  \KOMAoptions{parskip=half}}
\makeatother
\usepackage{xcolor}
\setlength{\emergencystretch}{3em} % prevent overfull lines
\setcounter{secnumdepth}{5}
% Make \paragraph and \subparagraph free-standing
\ifx\paragraph\undefined\else
  \let\oldparagraph\paragraph
  \renewcommand{\paragraph}[1]{\oldparagraph{#1}\mbox{}}
\fi
\ifx\subparagraph\undefined\else
  \let\oldsubparagraph\subparagraph
  \renewcommand{\subparagraph}[1]{\oldsubparagraph{#1}\mbox{}}
\fi

\usepackage{color}
\usepackage{fancyvrb}
\newcommand{\VerbBar}{|}
\newcommand{\VERB}{\Verb[commandchars=\\\{\}]}
\DefineVerbatimEnvironment{Highlighting}{Verbatim}{commandchars=\\\{\}}
% Add ',fontsize=\small' for more characters per line
\usepackage{framed}
\definecolor{shadecolor}{RGB}{241,243,245}
\newenvironment{Shaded}{\begin{snugshade}}{\end{snugshade}}
\newcommand{\AlertTok}[1]{\textcolor[rgb]{0.68,0.00,0.00}{#1}}
\newcommand{\AnnotationTok}[1]{\textcolor[rgb]{0.37,0.37,0.37}{#1}}
\newcommand{\AttributeTok}[1]{\textcolor[rgb]{0.40,0.45,0.13}{#1}}
\newcommand{\BaseNTok}[1]{\textcolor[rgb]{0.68,0.00,0.00}{#1}}
\newcommand{\BuiltInTok}[1]{\textcolor[rgb]{0.00,0.23,0.31}{#1}}
\newcommand{\CharTok}[1]{\textcolor[rgb]{0.13,0.47,0.30}{#1}}
\newcommand{\CommentTok}[1]{\textcolor[rgb]{0.37,0.37,0.37}{#1}}
\newcommand{\CommentVarTok}[1]{\textcolor[rgb]{0.37,0.37,0.37}{\textit{#1}}}
\newcommand{\ConstantTok}[1]{\textcolor[rgb]{0.56,0.35,0.01}{#1}}
\newcommand{\ControlFlowTok}[1]{\textcolor[rgb]{0.00,0.23,0.31}{#1}}
\newcommand{\DataTypeTok}[1]{\textcolor[rgb]{0.68,0.00,0.00}{#1}}
\newcommand{\DecValTok}[1]{\textcolor[rgb]{0.68,0.00,0.00}{#1}}
\newcommand{\DocumentationTok}[1]{\textcolor[rgb]{0.37,0.37,0.37}{\textit{#1}}}
\newcommand{\ErrorTok}[1]{\textcolor[rgb]{0.68,0.00,0.00}{#1}}
\newcommand{\ExtensionTok}[1]{\textcolor[rgb]{0.00,0.23,0.31}{#1}}
\newcommand{\FloatTok}[1]{\textcolor[rgb]{0.68,0.00,0.00}{#1}}
\newcommand{\FunctionTok}[1]{\textcolor[rgb]{0.28,0.35,0.67}{#1}}
\newcommand{\ImportTok}[1]{\textcolor[rgb]{0.00,0.46,0.62}{#1}}
\newcommand{\InformationTok}[1]{\textcolor[rgb]{0.37,0.37,0.37}{#1}}
\newcommand{\KeywordTok}[1]{\textcolor[rgb]{0.00,0.23,0.31}{#1}}
\newcommand{\NormalTok}[1]{\textcolor[rgb]{0.00,0.23,0.31}{#1}}
\newcommand{\OperatorTok}[1]{\textcolor[rgb]{0.37,0.37,0.37}{#1}}
\newcommand{\OtherTok}[1]{\textcolor[rgb]{0.00,0.23,0.31}{#1}}
\newcommand{\PreprocessorTok}[1]{\textcolor[rgb]{0.68,0.00,0.00}{#1}}
\newcommand{\RegionMarkerTok}[1]{\textcolor[rgb]{0.00,0.23,0.31}{#1}}
\newcommand{\SpecialCharTok}[1]{\textcolor[rgb]{0.37,0.37,0.37}{#1}}
\newcommand{\SpecialStringTok}[1]{\textcolor[rgb]{0.13,0.47,0.30}{#1}}
\newcommand{\StringTok}[1]{\textcolor[rgb]{0.13,0.47,0.30}{#1}}
\newcommand{\VariableTok}[1]{\textcolor[rgb]{0.07,0.07,0.07}{#1}}
\newcommand{\VerbatimStringTok}[1]{\textcolor[rgb]{0.13,0.47,0.30}{#1}}
\newcommand{\WarningTok}[1]{\textcolor[rgb]{0.37,0.37,0.37}{\textit{#1}}}

\providecommand{\tightlist}{%
  \setlength{\itemsep}{0pt}\setlength{\parskip}{0pt}}\usepackage{longtable,booktabs,array}
\usepackage{calc} % for calculating minipage widths
% Correct order of tables after \paragraph or \subparagraph
\usepackage{etoolbox}
\makeatletter
\patchcmd\longtable{\par}{\if@noskipsec\mbox{}\fi\par}{}{}
\makeatother
% Allow footnotes in longtable head/foot
\IfFileExists{footnotehyper.sty}{\usepackage{footnotehyper}}{\usepackage{footnote}}
\makesavenoteenv{longtable}
\usepackage{graphicx}
\makeatletter
\def\maxwidth{\ifdim\Gin@nat@width>\linewidth\linewidth\else\Gin@nat@width\fi}
\def\maxheight{\ifdim\Gin@nat@height>\textheight\textheight\else\Gin@nat@height\fi}
\makeatother
% Scale images if necessary, so that they will not overflow the page
% margins by default, and it is still possible to overwrite the defaults
% using explicit options in \includegraphics[width, height, ...]{}
\setkeys{Gin}{width=\maxwidth,height=\maxheight,keepaspectratio}
% Set default figure placement to htbp
\makeatletter
\def\fps@figure{htbp}
\makeatother
% definitions for citeproc citations
\NewDocumentCommand\citeproctext{}{}
\NewDocumentCommand\citeproc{mm}{%
  \begingroup\def\citeproctext{#2}\cite{#1}\endgroup}
\makeatletter
 % allow citations to break across lines
 \let\@cite@ofmt\@firstofone
 % avoid brackets around text for \cite:
 \def\@biblabel#1{}
 \def\@cite#1#2{{#1\if@tempswa , #2\fi}}
\makeatother
\newlength{\cslhangindent}
\setlength{\cslhangindent}{1.5em}
\newlength{\csllabelwidth}
\setlength{\csllabelwidth}{3em}
\newenvironment{CSLReferences}[2] % #1 hanging-indent, #2 entry-spacing
 {\begin{list}{}{%
  \setlength{\itemindent}{0pt}
  \setlength{\leftmargin}{0pt}
  \setlength{\parsep}{0pt}
  % turn on hanging indent if param 1 is 1
  \ifodd #1
   \setlength{\leftmargin}{\cslhangindent}
   \setlength{\itemindent}{-1\cslhangindent}
  \fi
  % set entry spacing
  \setlength{\itemsep}{#2\baselineskip}}}
 {\end{list}}
\usepackage{calc}
\newcommand{\CSLBlock}[1]{\hfill\break\parbox[t]{\linewidth}{\strut\ignorespaces#1\strut}}
\newcommand{\CSLLeftMargin}[1]{\parbox[t]{\csllabelwidth}{\strut#1\strut}}
\newcommand{\CSLRightInline}[1]{\parbox[t]{\linewidth - \csllabelwidth}{\strut#1\strut}}
\newcommand{\CSLIndent}[1]{\hspace{\cslhangindent}#1}

% load packages
\usepackage{geometry}
\usepackage{xcolor}
\usepackage{eso-pic}
\usepackage{fancyhdr}
\usepackage{sectsty}
\usepackage{fontspec}
\usepackage{titlesec}

%% Set page size with a wider right margin
\geometry{a4paper, total={170mm,257mm}, left=20mm, top=20mm, bottom=20mm, right=50mm}

%% Let's define some colours
\definecolor{uniblue}{HTML}{003865}
\definecolor{burgundy}{HTML}{7D2239}
\definecolor{cobalt}{HTML}{005C8A}
\definecolor{lavender}{HTML}{5B4D94}
\definecolor{leaf}{HTML}{006630}
\definecolor{moss}{HTML}{385A4F}
\definecolor{pillarbox}{HTML}{B30C00}
\definecolor{rust}{HTML}{9A3A06}
\definecolor{sandstone}{HTML}{52473B}
\definecolor{skyblue}{HTML}{005398}
\definecolor{slate}{HTML}{4F5961}
\definecolor{thistle}{HTML}{951272}

%\definecolor{light}{HTML}{E6E6FA} % original from template - redefined below as uni blue at 10 percent:
\colorlet{light}{uniblue!10}
%\definecolor{highlight}{HTML}{800080} % original from template - redefined below as uni's skyblue:
\colorlet{highlight}{skyblue}
%\definecolor{dark}{HTML}{330033} % original from template - redefined below as uni blue at 100 percent:
\colorlet{dark}{uniblue}

%% Let's add the border on the right hand side 
\AddToShipoutPicture{% 
    \AtPageLowerLeft{% 
        \put(\LenToUnit{\dimexpr\paperwidth-3cm},0){% 
            \color{light}\rule{3cm}{\LenToUnit\paperheight}%
          }%
     }%
     % logo
    \AtPageLowerLeft{% start the bar at the bottom right of the page
        \put(\LenToUnit{\dimexpr\paperwidth-2.25cm},27.2cm){% move it to the top right
            \color{light}\includegraphics[width=2.25cm]{_extensions/nrennie/PrettyPDF/uni_logo_boxed.jpg}
          }%
     }%
}

%% Style the page number
\fancypagestyle{mystyle}{
  \fancyhf{}
  \renewcommand\headrulewidth{0pt}
  \fancyfoot[R]{\thepage}
  \fancyfootoffset{3.5cm}
}
\setlength{\footskip}{20pt}

%% style the chapter/section fonts
\chapterfont{\color{uniblue}\fontsize{20}{16.8}\selectfont}
\sectionfont{\color{uniblue}\fontsize{20}{16.8}\selectfont}
\subsectionfont{\color{skyblue}\fontsize{14}{16.8}\selectfont}
\titleformat{\subsection}
  {\color{uniblue!90}\sffamily\Large\bfseries}{\thesubsection}{1em}{}[{\titlerule[0.8pt]}]
\subsubsectionfont{\color{cobalt}}

\renewcommand\thesection{\color{slate}\arabic{section}}
  
% left align title
\makeatletter
\renewcommand{\maketitle}{\bgroup\setlength{\parindent}{0pt}
\begin{flushleft}
  {\color{uniblue}\sffamily\huge\textbf{\@title}} \vspace{0.3cm} \newline
  {\Large {\@subtitle}} \newline
  \@author
\end{flushleft}\egroup
}
\makeatother

%% Use some custom fonts
\setsansfont{Ubuntu}[
    Path=_extensions/nrennie/PrettyPDF/Ubuntu/,
    Scale=0.9,
    Extension = .ttf,
    UprightFont=*-Regular,
    BoldFont=*-Bold,
    ItalicFont=*-Italic,
    ]

\setmainfont{Ubuntu}[
    Path=_extensions/nrennie/PrettyPDF/Ubuntu/,
    Scale=0.9,
    Extension = .ttf,
    UprightFont=*-Regular,
    BoldFont=*-Bold,
    ItalicFont=*-Italic,
    ]
\KOMAoption{captions}{tableheading}
\makeatletter
\@ifpackageloaded{tcolorbox}{}{\usepackage[skins,breakable]{tcolorbox}}
\@ifpackageloaded{fontawesome5}{}{\usepackage{fontawesome5}}
\definecolor{quarto-callout-color}{HTML}{909090}
\definecolor{quarto-callout-note-color}{HTML}{0758E5}
\definecolor{quarto-callout-important-color}{HTML}{CC1914}
\definecolor{quarto-callout-warning-color}{HTML}{EB9113}
\definecolor{quarto-callout-tip-color}{HTML}{00A047}
\definecolor{quarto-callout-caution-color}{HTML}{FC5300}
\definecolor{quarto-callout-color-frame}{HTML}{acacac}
\definecolor{quarto-callout-note-color-frame}{HTML}{4582ec}
\definecolor{quarto-callout-important-color-frame}{HTML}{d9534f}
\definecolor{quarto-callout-warning-color-frame}{HTML}{f0ad4e}
\definecolor{quarto-callout-tip-color-frame}{HTML}{02b875}
\definecolor{quarto-callout-caution-color-frame}{HTML}{fd7e14}
\makeatother
\makeatletter
\@ifpackageloaded{caption}{}{\usepackage{caption}}
\AtBeginDocument{%
\ifdefined\contentsname
  \renewcommand*\contentsname{Table of contents}
\else
  \newcommand\contentsname{Table of contents}
\fi
\ifdefined\listfigurename
  \renewcommand*\listfigurename{List of Figures}
\else
  \newcommand\listfigurename{List of Figures}
\fi
\ifdefined\listtablename
  \renewcommand*\listtablename{List of Tables}
\else
  \newcommand\listtablename{List of Tables}
\fi
\ifdefined\figurename
  \renewcommand*\figurename{Figure}
\else
  \newcommand\figurename{Figure}
\fi
\ifdefined\tablename
  \renewcommand*\tablename{Table}
\else
  \newcommand\tablename{Table}
\fi
}
\@ifpackageloaded{float}{}{\usepackage{float}}
\floatstyle{ruled}
\@ifundefined{c@chapter}{\newfloat{codelisting}{h}{lop}}{\newfloat{codelisting}{h}{lop}[chapter]}
\floatname{codelisting}{Listing}
\newcommand*\listoflistings{\listof{codelisting}{List of Listings}}
\makeatother
\makeatletter
\makeatother
\makeatletter
\@ifpackageloaded{caption}{}{\usepackage{caption}}
\@ifpackageloaded{subcaption}{}{\usepackage{subcaption}}
\makeatother
\makeatletter
\@ifpackageloaded{tcolorbox}{}{\usepackage[skins,breakable]{tcolorbox}}
\makeatother
\makeatletter
\@ifundefined{shadecolor}{\definecolor{shadecolor}{rgb}{.97, .97, .97}}{}
\makeatother
\makeatletter
\@ifundefined{codebgcolor}{\definecolor{codebgcolor}{named}{light}}{}
\makeatother
\makeatletter
\ifdefined\Shaded\renewenvironment{Shaded}{\begin{tcolorbox}[enhanced, boxrule=0pt, sharp corners, breakable, colback={codebgcolor}, frame hidden]}{\end{tcolorbox}}\fi
\makeatother
\ifLuaTeX
  \usepackage{selnolig}  % disable illegal ligatures
\fi
\usepackage{bookmark}

\IfFileExists{xurl.sty}{\usepackage{xurl}}{} % add URL line breaks if available
\urlstyle{same} % disable monospaced font for URLs
\hypersetup{
  pdftitle={Understanding our Data},
  colorlinks=true,
  linkcolor={highlight},
  filecolor={Maroon},
  citecolor={Blue},
  urlcolor={highlight},
  pdfcreator={LaTeX via pandoc}}

\title{Understanding our Data}
\author{}
\date{}

\begin{document}
\maketitle

\pagestyle{mystyle}

\section{Overview}\label{overview}

In this session, we will be looking at uncertainty and variability, and
how we can measure these and incorporate them into our conclusions.
Next, we will examine various environmental and ecological data sources,
highlighting critical pre-processing steps such as handling censored
data, outliers, and missing values.

We often talk about uncertainty and error as though they are
interchangeable, but this is not quite correct.

\begin{itemize}
\tightlist
\item
  \textbf{Error} is the difference between the measured value and the
  `true value' of the thing being measured.
\item
  \textbf{Uncertainty} is a quantification of the variability of the
  measurement result.
\end{itemize}

\subsection{Statistical distributions}\label{statistical-distributions}

Practically speaking, we make use of common statistical distributions to
account for uncertainty. These include both continuous and discrete
distributions.

\subsubsection{Continuous distributions}\label{continuous-distributions}

\begin{itemize}
\tightlist
\item
  \emph{Normal}: perhaps the most commonly used distribution in
  statistics. \(X \sim N(\mu, \sigma^2)\).
\item
  \emph{Exponential}: distribution of the time (\(\lambda\)) between
  events. \(X \sim Exp(\lambda)\).
\end{itemize}

\subsubsection{Discrete distributions}\label{discrete-distributions}

\begin{itemize}
\tightlist
\item
  \emph{Poisson}: distribution of the probability of observing a
  specific count (\(\theta\)) within a particular time period.
  \(X \sim Po(\theta)\).
\item
  \emph{Binomial}: distribution of the number of successes in \(n\)
  independent trials where \(\theta\) is the probability of success.
  \(X \sim Bi(n, \theta)\).
\item
  \emph{Negative binomial}: distribution of the number of trials until
  the \(k\)th success is observed. \(X \sim NeBi(k, \theta)\).
\end{itemize}

\subsection{Observational Error}\label{observational-error}

The observational \textbf{error} in a measurement is a single result,
namely the difference between the measured and the true value. The error
may include both a random and a systematic component.

\textbf{Random error} is variation that is observed randomly over a set
of repeat measurements. As you make more measurements, these errors tend
to average out and your estimates will improve in accuracy.

\textbf{Systematic error} is variation that remains constant over
repeated measures. This is typically due to some feature of the
measurement process. Making more measurements will not improve accuracy,
since all new measurements will be affected in the same way. Systematic
error can only be eliminated by identifying the cause of the error.

\begin{tcolorbox}[enhanced jigsaw, leftrule=.75mm, breakable, bottomrule=.15mm, colback=white, colframe=quarto-callout-tip-color-frame, coltitle=black, opacityback=0, toprule=.15mm, left=2mm, arc=.35mm, colbacktitle=quarto-callout-tip-color!10!white, bottomtitle=1mm, titlerule=0mm, toptitle=1mm, title={Exercise 1}, opacitybacktitle=0.6, rightrule=.15mm]

For each of the examples below, consider whether the error is
\textbf{random} or \textbf{systematic}.

\begin{itemize}
\item
  A meter reads 0.01 even when measuring no sample.
\item
  \begin{enumerate}
  \def\labelenumi{(\Alph{enumi})}
  \tightlist
  \item
    random\\
  \end{enumerate}
\item
  \begin{enumerate}
  \def\labelenumi{(\Alph{enumi})}
  \setcounter{enumi}{1}
  \tightlist
  \item
    systematic
  \end{enumerate}
\item
  An old thermometer can only measure the temperature to the nearest 0.5
  degrees. (e.g., 23.5 \(^\circ\)C becomes 23 \(^\circ\)C or 24
  \(^\circ\)C)
\item
  \begin{enumerate}
  \def\labelenumi{(\Alph{enumi})}
  \tightlist
  \item
    random\\
  \end{enumerate}
\item
  \begin{enumerate}
  \def\labelenumi{(\Alph{enumi})}
  \setcounter{enumi}{1}
  \tightlist
  \item
    systematic
  \end{enumerate}
\item
  A poorly designed rainfall monitor often leaks water on windy days.
\item
  \begin{enumerate}
  \def\labelenumi{(\Alph{enumi})}
  \tightlist
  \item
    random\\
  \end{enumerate}
\item
  \begin{enumerate}
  \def\labelenumi{(\Alph{enumi})}
  \setcounter{enumi}{1}
  \tightlist
  \item
    systematic
  \end{enumerate}
\item
  You are asked to measure the volume of an ice cube in a warm
  laboratory.
\item
  \begin{enumerate}
  \def\labelenumi{(\Alph{enumi})}
  \tightlist
  \item
    random\\
  \end{enumerate}
\item
  \begin{enumerate}
  \def\labelenumi{(\Alph{enumi})}
  \setcounter{enumi}{1}
  \tightlist
  \item
    systematic
  \end{enumerate}
\item
  To estimate the abundance of a fish species in a lake, scientists use
  a net with a mesh size equal to the average fish length.
  \texttt{mcq(c("random",\ answer="systematic"))}
\end{itemize}

\end{tcolorbox}

\section{Quantifying uncertainty}\label{quantifying-uncertainty}

\subsection{Standard uncertainty and expanded
uncertainty}\label{standard-uncertainty-and-expanded-uncertainty}

When presenting our results, it is important that we are clear the
uncertainty associated with them. A common approach is to use a
\textbf{standard uncertainty}, which is just the standard deviation,
reported as:
\[\text{estimated value } \pm \text{ standard uncertainty}\]

The standard uncertainty (\(u(\bar{\mathbf{x}})\)) for a vector
\(\mathbf{x}\) of length \(n\) is computed as follows:
\[u(\bar{\mathbf{x}}) = \frac{sd(\mathbf{x})}{\sqrt{(n)}}\]

More generally we can use an \textbf{expanded uncertainty}, which is
obtained by multiplying the standard uncertainty by a factor \(k\). You
have already seen this in statistics as the key building block of a
confidence interval. The value of \(k\) is chosen based on the quantiles
of a standard normal distribution, with a value of \(k=1.96\) (or
\(k=2\)) giving a 95\% confidence interval. The 95\% CI for the mean of
\(\mathbf{x}\) is given as
\(\bar{\mathbf{x}} \pm 1.96 \times u(\bar{\mathbf{x}}).\)

\begin{tcolorbox}[enhanced jigsaw, leftrule=.75mm, breakable, bottomrule=.15mm, colback=white, colframe=quarto-callout-important-color-frame, coltitle=black, opacityback=0, toprule=.15mm, left=2mm, arc=.35mm, colbacktitle=quarto-callout-important-color!10!white, bottomtitle=1mm, titlerule=0mm, toptitle=1mm, title={Example: Bathing water quality}, opacitybacktitle=0.6, rightrule=.15mm]

All bathing water sites in Scotland are classified by SEPA as
``Excellent'', ``Good'', ``Sufficient'' or ``Poor'' in terms of how much
fecal bacteria (from sewage) they contain. The minimum standard all
beaches or bathing water must meet is ``Sufficient''. The sites are
classified based on the 90th and 95th percentiles of samples taken over
the four most recent bathing seasons.

The figure below shows the data from some selected sites.

\begin{figure}[H]

{\centering \includegraphics{images/BathingWater.png}

}

\caption{Boxplots of FS by year for 10 sites. The dashed horizontal
lines represent ``excellent'' (green, lowest), ``good'' (blue, middle)
and ``sufficient'' (red, highest) classification boundaries,
respectively.}

\end{figure}%

The classification is based on a belief that the samples at each site
follow a log-normal distribution. If this assumption does not hold, then
our classifications would not be accurate. Therefore, it is crucial that
we regularly assess this assumption to ensure the safety of our bathing
water. We can use our standard plots to assess log-normality. In the
figure below, the top plots are produced using the untransformed data
and the bottom plots are produced after taking a logarithmic
transformation of the data (FS).

\includegraphics[width=5.20833in,height=\textheight]{images/WaterNormality1.png}
\includegraphics[width=5.20833in,height=\textheight]{images/WaterNormality2.png}\\

\end{tcolorbox}

\begin{tcolorbox}[enhanced jigsaw, leftrule=.75mm, breakable, bottomrule=.15mm, colback=white, colframe=quarto-callout-tip-color-frame, coltitle=black, opacityback=0, toprule=.15mm, left=2mm, arc=.35mm, colbacktitle=quarto-callout-tip-color!10!white, bottomtitle=1mm, titlerule=0mm, toptitle=1mm, title={Exercise 2}, opacitybacktitle=0.6, rightrule=.15mm]

Can we assume that the samples at each site follow a log-normal
distribution?

\begin{itemize}
\tightlist
\item
  \begin{enumerate}
  \def\labelenumi{(\Alph{enumi})}
  \tightlist
  \item
    Yes\\
  \end{enumerate}
\item
  \begin{enumerate}
  \def\labelenumi{(\Alph{enumi})}
  \setcounter{enumi}{1}
  \tightlist
  \item
    No
  \end{enumerate}
\end{itemize}

Solution

Yes, we can assume that the samples at each site follow a log-normal
distribution. From the plots, there is no strong evidence to suggest we
have breached our assumptions. Specifically, the histogram of
log\(_{10}\)(FS) shows that the distribution is not far from a bell
shape, and the points on the Normal Q-Q plot lie close to the line of
equality.

\end{tcolorbox}

\begin{tcolorbox}[enhanced jigsaw, leftrule=.75mm, breakable, bottomrule=.15mm, colback=white, colframe=quarto-callout-important-color-frame, coltitle=black, opacityback=0, toprule=.15mm, left=2mm, arc=.35mm, colbacktitle=quarto-callout-important-color!10!white, bottomtitle=1mm, titlerule=0mm, toptitle=1mm, title={Example: Bathing water quality (continued)}, opacitybacktitle=0.6, rightrule=.15mm]

In the following exercises, we will calculate the standard uncertainty
and a 95\% confidence interval for the mean of log(FS).

\end{tcolorbox}

\begin{tcolorbox}[enhanced jigsaw, leftrule=.75mm, breakable, bottomrule=.15mm, colback=white, colframe=quarto-callout-tip-color-frame, coltitle=black, opacityback=0, toprule=.15mm, left=2mm, arc=.35mm, colbacktitle=quarto-callout-tip-color!10!white, bottomtitle=1mm, titlerule=0mm, toptitle=1mm, title={Exercise 3}, opacitybacktitle=0.6, rightrule=.15mm]

\begin{enumerate}
\def\labelenumi{(\alph{enumi})}
\tightlist
\item
  We have 80 measurements of log(FS), with a mean of \textbf{3.861} and
  a standard deviation of \textbf{1.427}. Use these to calculate the
  standard uncertainty of the population mean log(FS) using our vector
  \(\textbf{x}\).
\end{enumerate}

\emph{Answer (to 3 decimal places):} \_\_\_\_\_

Solution

\[u = \frac{sd(\mathbf{x})}{\sqrt{(n)}} = \frac{1.427}{\sqrt{80}} = 0.160\]

\begin{enumerate}
\def\labelenumi{(\alph{enumi})}
\setcounter{enumi}{1}
\tightlist
\item
  Given the standard uncertainty that calculated in part (a), calculate
  a 95\% confidence interval for the population mean of log(FS).
\end{enumerate}

\emph{Answer (to 3 decimal places):} (\_\_\_\_\_,\_\_\_\_\_)

Solution

A 95\% confidence interval for \(\bar{x}\) is:
\[\bar{x} \pm 1.96 \times u = 3.861 \pm 1.96 \times 0.160 = (3.574, 4.175)\]

\end{tcolorbox}

\subsection{Uncertainty propagation}\label{uncertainty-propagation}

For a measure \(Y\) that is a linear combination of \(n\) quantities
\(X_1, ..., X_n\) (i.e.~\(Y = a_1X_1 + ... + a_nX_n\), with
\(\textbf{a} =(a_1, ..., a_n)\) being a~row vector of coefficients), the
\textbf{combined uncertainty} \(u(Y)\) is calculated as follows:

\[
\begin{aligned}
\text{Var}(Y) 
& = \text{Var}\left(\sum_{j=1}^n a_jX_j\right)\\
&= \sum_{i=1}^ka_i^2 \text{Var}(X_i) + \sum_{i\neq j}a_ia_j \text{Cov}(X_i,X_j) \\
&=\sum_i\sum_j a_ia_j\underbrace{\text{Cov}(X_i,X_j)}_{\rho_{ij}\sigma_i\sigma_j}\\
&\Rightarrow\\
u(Y) &= \sqrt{\sum_{i=1}^{n}\sum_{j=1}^{n}\left(u(X_i)\times u(X_j) \times a_i \times a_j \times \rho_{ij}\right)}
\end{aligned}
\]

where \(u(X_i) = \sigma_i\) and \(u(X_j) = \sigma_j\) are the standard
uncertainties of \(X_i\) and \(X_j\), respectively, and \(\rho_{ij}\) is
the correlation between \(X_i\) and \(X_j\).

If \(X_1, ..., X_n\) are independent, this reduces to:

\[u(Y) = \sqrt{\sum_{i=1}^{n}\left(u(X_i)^2 \times a_i^2\right)}\]

\begin{tcolorbox}[enhanced jigsaw, leftrule=.75mm, breakable, bottomrule=.15mm, colback=white, colframe=quarto-callout-tip-color-frame, coltitle=black, opacityback=0, toprule=.15mm, left=2mm, arc=.35mm, colbacktitle=quarto-callout-tip-color!10!white, bottomtitle=1mm, titlerule=0mm, toptitle=1mm, title={Exercise 4}, opacitybacktitle=0.6, rightrule=.15mm]

Show that the combined uncertainty \(u(Y)\) for
\(Y = a_0 + a_1X_1 + a_2X_2\) (\textbf{not} assuming that
\(X_1, ..., X_n\) are independent) reduces to:

\[u(Y) = \sqrt{a_1^2 u(X_1)^2 + a_2^2 u(X_2)^2 + 2 \rho_{12}u(X_1)u(X_2)a_1a_2}\]

Solution

We have:

\[u(Y) = \sqrt{\sum_{i=1}^{n}\sum_{j=1}^{n}\left(u(X_i)\times u(X_j) \times a_i \times a_j \times \rho_{ij}\right)}\]

where \(i = 1, 2\) and \(j = 1, 2\), i.e.

\[u(Y) = \sqrt{u(X_1) u(X_1) a_1 a_1 \rho_{11} + u(X_1) u(X_2) a_1 a_2 \rho_{12} + u(X_2) u(X_1) a_2 a_1 \rho_{21} + u(X_2) u(X_2) a_2 a_2 \rho_{22}}\]
\[\therefore u(Y) = \sqrt{u(X_1)^2 a_1^2 + 2 u(X_1) u(X_2) a_1 a_2 \rho_{12} + u(X_2)^2 a_2^2}\]

since \(\rho_{11} = \rho_{22} = 1\) and \(\rho_{12} = \rho_{21}\).

\end{tcolorbox}

The \textbf{general uncertainty propagation formula} is as follows. The
standard uncertainty of \(Y = f(X_1, ..., X_n)\) is:

\[u(Y) = \sqrt{\sum_{i=1}^n f'(\mu_i)^2u(X_i)^2}\]

where \(f'(\mu_i)\) is the partial derivative of \(Y\) with respect to
\(X_i\) evaluated at its mean \(\mu_i\).

\begin{tcolorbox}[enhanced jigsaw, leftrule=.75mm, breakable, bottomrule=.15mm, colback=white, colframe=quarto-callout-tip-color-frame, coltitle=black, opacityback=0, toprule=.15mm, left=2mm, arc=.35mm, colbacktitle=quarto-callout-tip-color!10!white, bottomtitle=1mm, titlerule=0mm, toptitle=1mm, title={Exercise 5}, opacitybacktitle=0.6, rightrule=.15mm]

We wish to calculate the area \(A\) of a rectangle, with height \(h\)
and width \(w\). (\(A = h \times w\).) Height and width are measured
with uncertainty \(u(h)\) and \(u(w)\), respectively. Evaluate the
uncertainty on the area \(A\).

Solution

\[u(A) = f ( h,w) = u(h \times w)\]

\[\frac{df}{dh} = w \ \ \text{and} \ \ \frac{df}{dw} = h\]

\[\therefore u(A) = \sqrt{w^2 u(h)^2 + h^2 u(w)^2}\]

\end{tcolorbox}

\begin{tcolorbox}[enhanced jigsaw, leftrule=.75mm, breakable, bottomrule=.15mm, colback=white, colframe=quarto-callout-tip-color-frame, coltitle=black, opacityback=0, toprule=.15mm, left=2mm, arc=.35mm, colbacktitle=quarto-callout-tip-color!10!white, bottomtitle=1mm, titlerule=0mm, toptitle=1mm, title={Exercise 6}, opacitybacktitle=0.6, rightrule=.15mm]

The \textbf{Shannon index} (or Shannon-Wiener diversity index) is widley
used in Ecology to quantify the diversity of a biological community by
considering both species richness and evenness. It is calculated as:

\[
H = -\sum_{i=1}^S p_i \log (p_i)
\]

where \(p_i\) is the proportion of species \(i\) in the community
computed as the ratio between the num. of individuals of a given species
and total number of individual across all species.

\begin{enumerate}
\def\labelenumi{\arabic{enumi}.}
\item
  Suppose that the proportion of each species is estimated with some
  uncertainty \(u(p_i)\). Provide the general form for the uncertainty
  propagation of these proportions on the calculation of \(H\).
\item
  Imagine you go to your garden an find out there are \(S=3\) different
  species of arthropods living there. Then you go out one day and sample
  \(N=100\) individuals and end up collecting \(n_1 = 50 \text{ ants}\),
  \(n_2 = 30 \text{ beetles}\) and \(n_3 = 20 \text{ spiders}\).
  Assuming that number of individuals of a given species follows
  \(n_i \sim \text{Binomial}(N,\theta_i)\), and let
  \(\hat{\theta_i} = \frac{n_i}{N} = p_i\) be the estimator of
  \(\theta_i\) show that \(u(p_i)^2 = p_i(1-p_i)/N\) and then compute
  the uncertainty propagation for the Shannon Index.
\end{enumerate}

Solution

\[
\begin{aligned}
u(H) = \sqrt{\sum_i^S \left(\dfrac{\partial H}{\partial p_i} \right)^2u(p_i)^2}
\end{aligned}
\]

where

\[\dfrac{\partial H}{\partial p_i} = - (\log(p_i) +1)\]

\[\therefore u(H) = \sqrt{\sum_i^S \left(-\log (p_i)-1 \right)^2u(p_i)^2}\]

Assuming \(n_i \sim \text{Binomial}(100,p_i)\) for \(i = 1,2,3\) where
\(p_1 = 50/100; p_2 = 0.3; p_3 = 20/100\). The partial derivatives are
given by

\[
\begin{aligned}\frac{\partial H}{\partial p_1} &= -(\log 0.5 + 1) \approx -0.307 \\\frac{\partial H}{\partial p_2} &= -(\log 0.3 + 1) \approx 0.204 \\\frac{\partial H}{\partial p_3} &= -(\log 0.2 + 1) \approx 0.609\end{aligned}
\]

First, the variance of \(p_i\) (i.e., \(u(p_i)^2\)) is given by:

\[
\begin{aligned}
\text{Var}(p_i) &= \text{Var}\left(\frac{n_i}{N}\right) \\
&= \frac{1}{N^2}\text{Var}(n_i)\\
&= \frac{N p_i(1 - p_i)}{N^2} = \dfrac{p_i(1-p_i)}{N}
\end{aligned}
\]

Thus,

\[
\begin{aligned}
u(p_1)^2 &= \frac{0.5 \times 0.5}{100} = 0.0025 \\
u(p_2)^2 &= \frac{0.3 \times 0.7}{100} = 0.0021 \\
u(p_3)^3 &= \frac{0.2 \times 0.8}{100} = 0.0016
\end{aligned}
\]

Then, \[
u(H)  = \sqrt{0.0025 \times (-0.307)^2 + 0.0021 \times (0.204)^2 + 0.0016 \times (0.609)^2} \approx 0.03
\]

\end{tcolorbox}

\section{Data Sources}\label{data-sources}

\subsection{Ecological and Environmental Data
sources}\label{ecological-and-environmental-data-sources}

Over the last decade, the information available for surveying and
monitoring ecological and environmental resources has changed radically.
The rise of new technologies, novel collection methods, and modern
data-submission platforms have facilitated the access to large volumes
of environmental and ecological data. For example,
Figure~\ref{fig-nbn_records} show the increasing trend in the number of
NBN trust biodiversity records that have become available over the past
20 years.

\begin{figure}

\centering{

\includegraphics[width=5.97917in,height=\textheight]{images/NBN_Trust_Growth.gif}

}

\caption{\label{fig-nbn_records}Number of biodiversity records publicly
available from the NBN Trust'}

\end{figure}%

Today's ecological and environmental data landscape is overwhelmingly
vast - far too extensive to cover comprehensively in one session!
Instead, we'll focus on key data sources and digital technologies that
are currently shaping policy decisions, enabling scientific
breakthroughs, and driving innovation in research.

\url{https://youtu.be/7MywGLpOBWs}

\subsection{Institutional Monitoring
Programmes}\label{institutional-monitoring-programmes}

Institutional Monitoring programmes have long been a primary source of
information for long-term environmental assessment, producing
\textbf{structured datasets} essential for detecting ecological change
and informing evidence-based policy.

These initiatives rely \emph{field surveys} conducted on established
\emph{monitoring networks} to track trends in species populations,
habitat quality, and ecosystem processes - a topic we will explore in
detail in the following session. Their strength lies in rigorous
implementation of \emph{standardized sampling protocols}, which reduces
the observational errors associated with data collection methods.
However, these are typically constrained by other factors. For example,
large-scale programmes are inherently resource-intensive to maintain and
often limited in taxonomic scope (typically focusing on key species),
spatial/geographic coverage, and temporal resolution. Some popular
monitoring schemes are shown below:

\begin{longtable}[]{@{}
  >{\raggedright\arraybackslash}p{(\columnwidth - 2\tabcolsep) * \real{0.3662}}
  >{\raggedright\arraybackslash}p{(\columnwidth - 2\tabcolsep) * \real{0.6338}}@{}}
\toprule\noalign{}
\begin{minipage}[b]{\linewidth}\raggedright
Monitoring Scheme
\end{minipage} & \begin{minipage}[b]{\linewidth}\raggedright
Description
\end{minipage} \\
\midrule\noalign{}
\endhead
\bottomrule\noalign{}
\endlastfoot
United Kingdom Butterfly Monitoring Scheme
(\href{https://ukbms.org/}{UKBMS}) & Protocolized sampling scheme run by
butterfly conservation that has monitored changes in the abundance of
butterflies throughout the United Kingdom since 1976. \\
UK Environmental Change Network (\href{https://ecn.ac.uk/}{ECN}) & UK's
long-term ecosystem monitoring and research programme that has produced
a large collection of publicly available data sets including
meteorological, biogeochemistry and biological data for different
taxonomic groups (Rennie et al. 2020). \\
National Hydrological Monitoring Programme
(\href{https://nrfa.ceh.ac.uk/nhmp}{NHMP}) & The NHMP, particulalry the
National River Flow Archive conveys a national scale management of
hydrological data within the UK hosted by the UKCEH since 1982 collating
hydrometric data from gauging station networks operated by multiple
agencies. \\
Natural Capital and Ecosystem Assessment
(\href{https://environment.data.gov.uk/natural-capital-ecosystem-assessment/about}{NCEA})
& Long-term environmental monitoring of natural capital including data
from freshwater Surveillance Networks, ecosystem condition \& soil
health, forest inventory, estuary and coast surveillance, etc. \\
Breeding Bird Survey (\href{https://www.bto.org/}{BBS}) & Main scheme
for monitoring the population changes of the UK's common breeding birds.
It covers all habitat types and monitors 110 common and widespread
breeding birds using a randomised site selection. \\
\end{longtable}

\subsection{\texorpdfstring{\textbf{Citizen Science Programmes \&
Platforms}}{Citizen Science Programmes \& Platforms}}\label{citizen-science-programmes-platforms}

Citizen science (CS) monitoring involve public participation to collect
large volumes of ecological \& environmental data at a low cost across
broad spatiotemporal scales.

Data submission platforms like iNaturalist and eBird have become an
important groundwork for citizen scientist to submit records and
generate vast, real-time datasets, enabling researchers to track species
distributions, phenology, and ecosystem responses to environmental
change in ways that were previously logistically and financially
unfeasible.

Despite this, the analysis of such data remains challenging as there is
little o no design involved in the sampling protocols of most CS data
recording schemes. A major issue with the lack of a standardized
sampling protocol is that sampling efforts tend to be uneven over time
and space and biased towards human activity centers, locations that are
easy to access, or sites where species are more likely to be found such
as protected area. If unaccounted for, such sources of bias can mislead
Figure~\ref{fig-sampling_effort} (left) shows the sampling effort based
on CS records submitted through the Pl@anet Net App in the French
mediterranean region. If we compare the spatial effort against the
elevation of the region we can clearly see that a \emph{sampling bias}
towards lower elevation values. This would imply that small populations
at lower elevation could be over-sampled and if we had worngly assumed
sampling was evenly distributed, then species distributions at higher
elevation would be under-estimated.

\begin{figure}

\centering{

\includegraphics{images/sampling_eff_ex.png}

}

\caption{\label{fig-sampling_effort}Elevation versus sampling effort
(obtained through the Pl@net Net App) in the French mediterranean region
(Figure taken from Botella et al. (2020))}

\end{figure}%

Harnessing the power of CS data is not an easy task.

\begin{longtable}[]{@{}
  >{\raggedright\arraybackslash}p{(\columnwidth - 2\tabcolsep) * \real{0.5556}}
  >{\raggedright\arraybackslash}p{(\columnwidth - 2\tabcolsep) * \real{0.4444}}@{}}
\toprule\noalign{}
\begin{minipage}[b]{\linewidth}\raggedright
Advantages 😄👍
\end{minipage} & \begin{minipage}[b]{\linewidth}\raggedright
Disadvantages 😔👎
\end{minipage} \\
\midrule\noalign{}
\endhead
\bottomrule\noalign{}
\endlastfoot
Extensive taxonomic, spatial and temporal coverage. & Under-reporting of
rare and inconspicuous species. \\
Eye-catching species that are easily identifiable by participants. &
Varying recording skills and uneven sampling effort. \\
\end{longtable}

\subsection{Biological Collections}\label{biological-collections}

Biological collections constitute probably the oldest form of historical
data reservoirs. For over 300 years, naturalists have been collecting
and preserving biological specimens, initially for personal curiosity
and public display.~Today, their value has expanded far beyond their
original purpose; they are now recognized as critical sources of
information for addressing modern global challenges~like biodiversity
loss and climate change. Now housed mostly in \textbf{museums} and
\textbf{herbaria} throughout the world, these biological collections,
and their associated systematic research, provided the basis for much
biological research.

For instance, the Natural History Museum in London safeguards a
collection of over 80 million specimens, spanning 4.5 billion years of
Earth's history to the present. This unparalleled archive, along with
many others, is increasingly accessible through digital
\href{https://data.nhm.ac.uk/?_gl=1*xkk*_ga*NDE2OTQ4MTI2LjE3NTU2MjI5ODY.*_ga_PYMKGK73C4*czE3NTU2MjI5ODYkbzEkZzEkdDE3NTU2MjI5OTQkajUyJGwwJGgw}{data
portals}, enabling researchers worldwide to analyse historical trends
and understand the distribution of biodiversity and geodiversity through
time.

Despite their immense value, biological collections data are subject to
significant limitations and biases that need bo considered.

\begin{itemize}
\item
  Most historic collection were obtained in an opportunistic manner
  without following any particular sampling protocol (largely dependent
  on the particular interests of the collector).
\item
  Often there is no information about the collection methods or effort
  employed.
\item
  Limited in the geographic coverage and typically biased near centres
  of human activity and along the roads or during specific seasons.
\item
  The information associated with each collection or specimen (e.g.,
  species, sex, collection date, location, collector's name,
  morphological measurements, habitat description) may vary widely which
  limits the environmental context and ecological questions~that can be
  addressed.
\item
  Strongly biased towards specific taxonomic groups, especially birds
  and mammals
\end{itemize}

\begin{tcolorbox}[enhanced jigsaw, leftrule=.75mm, breakable, bottomrule=.15mm, colback=white, colframe=quarto-callout-tip-color-frame, coltitle=black, opacityback=0, toprule=.15mm, left=2mm, arc=.35mm, colbacktitle=quarto-callout-tip-color!10!white, bottomtitle=1mm, titlerule=0mm, toptitle=1mm, title={Exercise 7}, opacitybacktitle=0.6, rightrule=.15mm]

Read the paper by Pyke and Ehrlich (2010) and discuss three scenarios
where biological collections have been used tp address different
environmental issues and ecological questions.

\end{tcolorbox}

\subsection{\texorpdfstring{\textbf{Data Repositories \&
Portals}}{Data Repositories \& Portals}}\label{data-repositories-portals}

Data repositories have become a major sources of information for modern
environmental and ecological research, serving as centralized, curated
platforms that aggregate, preserve, and disseminate vast quantities of
data from diverse sources.~

These digital archives - ranging from global biodiversity databases like
the Global Biodiversity Information Facility (GBIF) to thematic
collections such as the National Biodiversity Network (NBN) Atlas -
standardize and harmonize heterogeneous datasets, enabling researchers
to access, share, and reuse data across disciplines and geographic
boundaries. Often, these repositories are integrated into
comprehensive~\textbf{data portals} that host interactive visualisation
tools, web-based applications, programming interfaces (APIs) and data
catalogues, transforming static archives into dynamic platforms for
exploration and discovery (see e.g.~UK-SCAPE
\href{https://connect-apps.ceh.ac.uk/common_plant_change/?_gl=1*a9wt35*_ga*NDI3MDQzMjc2LjE3NTQ2NjE1MzE.*_ga_27CMQ4NHKV*czE3NTUyNzkxNjkkbzMkZzEkdDE3NTUyNzk2OTckajkkbDAkaDA}{plant
diversity trends} and
\href{https://connect-apps.ceh.ac.uk/targeting_revisits_grasshoppers/?_gl=1*1513pqk*_ga*NDI3MDQzMjc2LjE3NTQ2NjE1MzE.*_ga_27CMQ4NHKV*czE3NTUyNzkxNjkkbzMkZzEkdDE3NTUyNzk2OTckajkkbDAkaDA.}{Grasshoppers
and Allied Species Recording Scheme}).

\begin{tcolorbox}[enhanced jigsaw, leftrule=.75mm, breakable, bottomrule=.15mm, colback=white, colframe=quarto-callout-tip-color-frame, coltitle=black, opacityback=0, toprule=.15mm, left=2mm, arc=.35mm, colbacktitle=quarto-callout-tip-color!10!white, bottomtitle=1mm, titlerule=0mm, toptitle=1mm, title={Exercise 8}, opacitybacktitle=0.6, rightrule=.15mm]

Select two or three of the following data repositories. For each,
examine some of the available datasets and list the types of uncertainty
or error that might affect the data quality and reliability.

\begin{longtable}[]{@{}
  >{\raggedright\arraybackslash}p{(\columnwidth - 2\tabcolsep) * \real{0.3889}}
  >{\raggedright\arraybackslash}p{(\columnwidth - 2\tabcolsep) * \real{0.6111}}@{}}
\toprule\noalign{}
\begin{minipage}[b]{\linewidth}\raggedright
Data Repository
\end{minipage} & \begin{minipage}[b]{\linewidth}\raggedright
Description
\end{minipage} \\
\midrule\noalign{}
\endhead
\bottomrule\noalign{}
\endlastfoot
\href{https://www.movebank.org/cms/movebank-main}{Move Bank} & Movebank
is a global, open-data repository and research platform that specializes
in managing, sharing, and analyzing \textbf{animal tracking} and
bio-logging data. \\
Global Biodiversity Information Facility
(\href{https://www.gbif.org/}{GBIF}) & GBIF is an international
open-data infrastructure that provides free and universal access to over
two billion \textbf{species occurrence records} from a vast network of
museums, research institutions, and citizen science platforms
worldwide. \\
National Biodiversity Network (\href{https://nbnatlas.org/}{NBN}) & The
NBN Atlas is the UK's largest repository of publicly available
biodiversity data, aggregating and providing open access to millions of
\textbf{species records} from a wide range of recording societies,
conservation NGOs, and research institutions across the country. \\
Biological Records Centre (\href{https://www.brc.ac.uk/}{BRC}) & The
Biological Records Centre (BRC) is a national UK facility that supports
and coordinates a network of volunteer recording schemes and societies
to collect, manage, and disseminate high-quality data on terrestrial and
freshwater \textbf{species distributions}. \\
National River Flow Archive (\href{https://nrfa.ceh.ac.uk/}{NRFA}) & The
National River Flow Archive (NRFA) is ta hydrometric data repository
hosted by the CEH, curating and providing open access to river flow,
groundwater level, and rainfall time-series from a national network of
monitoring stations. \\
\href{https://ukwrp.ceh.ac.uk/}{UK Lakes Portal} & The UK Lakes Portal
is a comprehensive data hub that provides access to physical, chemical,
and biological monitoring data for lakes and reservoirs across the
United Kingdom \\
World Ocean Database
(\href{https://www.ncei.noaa.gov/products/world-ocean-database}{WOD}) &
The World Ocean Database (WOD) is the world's largest publicly
available, quality-controlled repository of uniformly formatted
oceanographic profile and plankton data, spanning centuries of global
marine observations. \\
\href{https://ukwrp.ceh.ac.uk/?_gl=1*1r52k3b*_ga*NDI3MDQzMjc2LjE3NTQ2NjE1MzE.*_ga_27CMQ4NHKV*czE3NTU2MDk2NjgkbzUkZzEkdDE3NTU2MDk5MzIkajYwJGwwJGgw}{UK
Water Resources Portal} & The UK Water Resources Portal is an
interactive online platform that provides access to current and
historical data on water availability, including river flows,
groundwater levels, rainfall, and reservoir stocks. \\
Water quality data archive
(\href{https://environment.data.gov.uk/water-quality/view/landing}{WIMS})
& The Water Quality Archive provides data on water quality measurements.
Samples are taken at sampling points around England and can be from
coastal or estuarine waters, rivers, lakes, ponds, canals or
groundwaters. \\
\href{https://environment.data.gov.uk/ecology/explorer/}{Ecology \& Fish
Data Explorer} & This is an online data portal that provides access to
ecological monitoring data for English rivers and lakes, including fish
populations, invertebrate surveys, and plant communities. \\
Knowledge Network for Biocomplexity
(\href{https://knb.ecoinformatics.org/}{KNB}) & The Knowledge Network
for Biocomplexity (KNB) is an open-source data repository that enables
the discovery, management, sharing, and synthesis of complex,
heterogeneous ecological and environmental datasets. \\
\end{longtable}

\end{tcolorbox}

\subsection{\texorpdfstring{\textbf{Processed information
products}}{Processed information products}}\label{processed-information-products}

Processed information products~transform raw measurements into refined,
analysis-ready resources tailored for decision-makers and researchers.
Unlike primary data repositories, these products undergo rigorous
calibration, integration, and modelling to generate authoritative maps,
indicators, and synthesized datasets.

\begin{tcolorbox}[enhanced jigsaw, leftrule=.75mm, breakable, bottomrule=.15mm, colback=white, colframe=quarto-callout-important-color-frame, coltitle=black, opacityback=0, toprule=.15mm, left=2mm, arc=.35mm, colbacktitle=quarto-callout-important-color!10!white, bottomtitle=1mm, titlerule=0mm, toptitle=1mm, title={Example: WorldClim}, opacitybacktitle=0.6, rightrule=.15mm]

\href{https://www.worldclim.org/}{WorldClim} is a widely used set of
global, high-resolution climate surfaces (raster maps) that provide
interpolated estimates of historical and future projections (using
global climate models
\href{https://wcrp-cmip.org/cmip-phases/cmip6/}{CMIP} ) of temperature,
precipitation, and other bioclimatic variables. The historical data
layers are generated by applying advanced spatial interpolation
algorithms to an extensive global network of weather station records,
creating continuous, gap-free rasters. These surfaces serve as the
foundational data for species distribution modeling, ecological
forecasting, and a vast range of other environmental research
applications.

\begin{center}
\includegraphics[width=4.6875in,height=\textheight]{images/worldclim.jpeg}
\end{center}

\end{tcolorbox}

Nowadays, it is common that contemporary data products are synthesized
based on a combination of multiple data sources, including field
surveys, citizen science and advanced \textbf{remote sensing} data from
satellite and aerial platforms.

\subsubsection{Remote sensing}\label{remote-sensing}

Remote sensing refer the process of obtaining information of an object
from a distance, typically from aircraft or satellites. Recent advances
in bioinformatics, GIS technologies and remote sensing techniques have
changed radically how we monitor the Earth's environment at multiple
spatial and temporal scales. These technologies enables the systematic,
non-invasive, and often near-real-time collection of data across vast
and inaccessible regions. The resulting data are then calibrated,
classified, and modeled using specialized algorithms to generate diverse
information products, such as land cover classifications, vegetation
indices, and digital elevation models.

While remote sensing-based products enable the quantification of
ecological and environmental parameters across extensive geographic
scales, they are often subject to substantial uncertainties. These
include systematic errors from sensor calibration, spatial and temporal
resolution constraints, and generally lower accuracy compared to direct
\emph{in-situ} field measurements. Consequently, remote sensing data are
often validated using data collected~\emph{in-situ} to assess and ensure
their accuracy.~

\begin{tcolorbox}[enhanced jigsaw, leftrule=.75mm, breakable, bottomrule=.15mm, colback=white, colframe=quarto-callout-important-color-frame, coltitle=black, opacityback=0, toprule=.15mm, left=2mm, arc=.35mm, colbacktitle=quarto-callout-important-color!10!white, bottomtitle=1mm, titlerule=0mm, toptitle=1mm, title={Example: Digital Elevation Models}, opacitybacktitle=0.6, rightrule=.15mm]

Digital Elevation Models (DEMs) are digital representations of the
earth's topographic surface. DEMs providing a continuous and
quantitative model of terrain morphology and are typically stored as a
raster grid where each cell (pixel) contains an elevation value. The
accuracy of DEMs is determined primarily by the resolution of the model
(the size of the area represented by each individual grid cell). For
example, the

Shuttle RaDAR Topography Mission (SRTM), aquired by NASA using a
Synthetic Aperture Radar (SAR) instrument, provide elevation data for
any country and is available from the \texttt{geodata} R package.

\begin{center}
\includegraphics[width=2.72917in,height=\textheight]{images/dem.jpeg}
\end{center}

\end{tcolorbox}

\begin{tcolorbox}[enhanced jigsaw, leftrule=.75mm, breakable, bottomrule=.15mm, colback=white, colframe=quarto-callout-important-color-frame, coltitle=black, opacityback=0, toprule=.15mm, left=2mm, arc=.35mm, colbacktitle=quarto-callout-important-color!10!white, bottomtitle=1mm, titlerule=0mm, toptitle=1mm, title={Example: Land Cover Maps}, opacitybacktitle=0.6, rightrule=.15mm]

Land cover maps describe the physical material on the Earth's surface.
They are created by applying automated algorithms to satellite or aerial
imagery to identify features such as grassland, woodland, rivers \&
lakes or man-made structures such as roads and buildings.

For example, UK CEH has produced a series of
\href{https://www.ceh.ac.uk/data/ukceh-land-cover-maps}{Land Cover Maps}
which are a series of spatially continuous, raster-based classification
products, derived from the automated analysis of Earth observation data
(primarily from the Sentinel-2 satellite constellation), which provide
consistent, national-scale representations of surface vegetation and
land use classes.

\begin{center}
\includegraphics[width=2.84375in,height=\textheight]{images/LCM 2020 10m.png}
\end{center}

Other widely used global products include MODIS Land Cover, which offers
a long-term, coarse-resolution record of global change since 2001,
and~ESA WorldCover, which provides a high-resolution (10m) global
snapshot designed for detailed thematic mapping (the later is available
on the \texttt{geodata} R package).

\end{tcolorbox}

\begin{tcolorbox}[enhanced jigsaw, leftrule=.75mm, breakable, bottomrule=.15mm, colback=white, colframe=quarto-callout-important-color-frame, coltitle=black, opacityback=0, toprule=.15mm, left=2mm, arc=.35mm, colbacktitle=quarto-callout-important-color!10!white, bottomtitle=1mm, titlerule=0mm, toptitle=1mm, title={Example: NDVI Vegetation Index}, opacitybacktitle=0.6, rightrule=.15mm]

Vegetation indeces derived from remote sensing utilize spectral data
from satellite or aerial sensors to quantify and monitor plant health,
structure, and function across landscapes. These indeces are founded on
the principle that vegetation absorbs red light (around 660 nm) for
photosynthesis while highly reflecting near-infrared (NIR) light (around
800 nm) due to its internal leaf structure.

This contrast is captured by the Normalized Difference Vegetation Index
(NDVI), which is calculated as

\[
NDVI = \dfrac{(NIR- Red)}{(NIR - Red)}
\]

The resulting value, which ranges from -1 to +1, provides a standardized
measure of greenness; values close to +1 indicate dense, healthy
vegetation, values near 0 represent bare soil, and negative values
typically correspond to water. By translating raw spectral data into
this simple yet robust index, remote sensing enables the tracking of
phenological cycles, the assessment of drought stress, and the
estimation of primary productivity on a global scale.

\begin{center}
\includegraphics[width=3.85417in,height=\textheight]{images/ndvi.png}
\end{center}

\end{tcolorbox}

\subsection{\texorpdfstring{\textbf{Research-Generated
Data}}{Research-Generated Data}}\label{research-generated-data}

Research-generated data repositories, such as
\href{https://datadryad.org/about}{Dryad} and
\href{https://zenodo.org/}{Zenodo}, are cornerstone platforms in the
modern scientific workflow, explicitly designed to uphold the principles
of transparency, reproducibility, and open data access. Unlike passive
archives, these repositories require researchers to actively deposit the
precise datasets, code, and scripts used to generate the results
published in peer-reviewed journals. By assigning persistent digital
object identifiers (DOIs) to these materials, they create a permanent,
citable record that allows other scientists to independently verify,
replicate, and build upon published findings. This process is
fundamental to detecting errors, reducing redundancy, and accelerating
scientific discovery, effectively transforming a single study's output
into a reusable resource for the entire research community and
safeguarding the integrity of the scientific record.

\begin{tcolorbox}[enhanced jigsaw, leftrule=.75mm, breakable, bottomrule=.15mm, colback=white, colframe=quarto-callout-tip-color-frame, coltitle=black, opacityback=0, toprule=.15mm, left=2mm, arc=.35mm, colbacktitle=quarto-callout-tip-color!10!white, bottomtitle=1mm, titlerule=0mm, toptitle=1mm, title={Exercise 9}, opacitybacktitle=0.6, rightrule=.15mm]

\begin{enumerate}
\def\labelenumi{\arabic{enumi}.}
\tightlist
\item
  \textbf{Choose a Repository:} Select either
  \href{https://datadryad.org/about}{Dryad} or
  \href{https://zenodo.org/}{Zenodo}.
\item
  \textbf{Find a Dataset:} Browse or search for a dataset related to a
  topic in environmental science or ecology that interests you (e.g.,
  ``pollination,'' ``microplastics,'' ``forest fragmentation,''
  ``climate change adaptation'').
\item
  \textbf{Select and Record:} Choose one specific dataset and note down:

  \begin{itemize}
  \item
    The full citation for the dataset (including its DOI).
  \item
    The title of the associated publication (if provided).
  \end{itemize}
\end{enumerate}

How did you find this dataset? Was the search intuitive? Is the dataset
openly available? Could you download the files without restrictions?

\end{tcolorbox}

\section{Data Preprocessing}\label{data-preprocessing}

Environmental and Ecological systems are inherently complex due to the
large number of interrelated biological, physical, and social
components. This complexity arises stochastic processes that operate
across vastly different spatial and temporal scales. Adding to this
complexity, analyzing these systems~becomes~a challenging task due to
the heterogeneity of available data and the different sources of
uncertainty that impact the quality of the data. Data collection methods
vary widely and spatial and temporal sampling schemes may be too sparse
to fully capture overall system behavior. Consequently, we often have to
deal with issues such as outliers, missing values, and highly uncertain
information.

Fortunately, many of these data quality issues can be addressed. This is
typically done through a rigorous data \emph{pre-processing} phase
before formal analysis, and through statistical models that explicitly
account for the observational process of how the data was collected.

Data pre-processing is crucial stage in any sort of ecological or
environmental data analysis and it includes data cleaning, outlier
detection, missing value treatment, handling censored data,
transformation, and the creation of new derived variables. The goal is
to create a robust, consistent dataset ready for analysis while
carefully documenting all changes to preserve the integrity of the
original information.

\subsection{Censored Data}\label{censored-data}

\textbf{Censored data} are data where we are restricted in our knowledge
about them in some way or other. Often this will be because we only know
that the data value lies below a certain minimum value (or above a
certain maximum). For example, if we had scales which only weighed up to
10kg, we would not know the exact weight of any object greater than
10kg.

For environmental data, it is more common to have data which are
censored at some minimum value. This is because many pieces of measuring
equipment will have an analytical \textbf{limit of detection}. A limit
of detection is \emph{the lowest concentration that can be distinguished
with reasonable confidence from a blank} (a hypothetical sample with a
value of zero). The limit of detection is often denoted \(c_L\).

Censoring has a huge impact on how we interpret our data. The two plots
below show the same data, but the right panel is `censored' with two
different limits of detection (some with an LOD of 0.5, others with an
LOD of 1.5).

\begin{figure}

\begin{minipage}{0.50\linewidth}
\includegraphics[width=3.64583in,height=\textheight]{images/ConcUncensored.png}\end{minipage}%
%
\begin{minipage}{0.50\linewidth}
\includegraphics[width=3.64583in,height=\textheight]{images/ConcCensored.png}\end{minipage}%

\end{figure}%

Censored observations are not completely without information. We still
know they are equal to or more extreme than the limit. For an LOD, we
might therefore report the data point as either ``not detected'' or
`\(< c_L\)'. Removing them from our study would not be sensible, since
this would lead to us \emph{overestimating} the mean and probably also
\emph{underestimating} the variance. We therefore need to find a way to
incorporate these censored data points into our analysis.

We can't simply use the minimum value of the LOD. This would ignore the
fact that the values are often \emph{below} this. In the plot below, the
LOD reduces after every 100 observations (e.g.~because of better quality
equipment), and this leads to an artificial trend.

\begin{center}
\includegraphics[width=4.25in,height=\textheight]{images/TSM1.png}
\end{center}

\hfill\break

\subsubsection{Simple Substitution}\label{simple-substitution}

The simplest approach for dealing with LODs is via \textbf{simple
substitution}. This involves taking the LOD value and multiplying it by
a fixed constant, e.g.~by replacing all \(<c_L\) values with \(0.5c_L\).

This approach is fairly popular because it is simple and easy to
implement. However, it only works if there is a small proportion of
censored data (maximum 10--15\%). If there is a higher proportion, it
tends to overestimate the mean.\\

\subsubsection{Distribution-based
approaches}\label{distribution-based-approaches}

It is generally preferable to use a more statistics-based approach which
accounts for the data distribution. The basic idea is that we estimate
the statistical distribution of the data in a way that takes into
account the censoring. We can then use this estimated distribution to
simulate values for our censored points.

Commonly used distribution-based approaches are \textbf{Maximum
Likelihood}, \textbf{Kaplan-Meier}, and \textbf{Regression on Order
Statistics}.

\paragraph{Maximum Likelihood}\label{maximum-likelihood}

The maximum likelihood approach is a \emph{parametric} approach. It
requires us to specify a statistical \textbf{distribution} which is a
close fit to the data. We then identify the \textbf{parameters} of this
distribution that maximize the likelihood of obtaining a dataset like
ours.

This ML approach has to take into account the likelihood of obtaining:

\begin{itemize}
\tightlist
\item
  the observed values in our dataset
\item
  the correct proportion of data being censored, i.e.~the proportion
  falling below our detection limit(s)
\end{itemize}

\begin{center}
\includegraphics[width=5.39583in,height=\textheight]{images/MLEcensoring.png}
\end{center}

\textbf{Advantages}

\begin{itemize}
\tightlist
\item
  Able to handle multiple limits of detection.
\item
  Good for estimating summary statistics with a suitably large sample
  size.
\item
  MLE explicitly accounts for the underlying distribution of the data
  (if known).
\end{itemize}

\textbf{Disadvantages}

\begin{itemize}
\tightlist
\item
  More applicable to larger datasets (n \textgreater{} 50).
\item
  Reliant on specifying the correct distribution, otherwise estimates
  can be incorrect.
\item
  Transforming data to fit a distribution can potentially cause biased
  estimators.
\end{itemize}

\paragraph{Kaplan-Meier Approach}\label{kaplan-meier-approach}

The Kaplan-Meier approach is a \emph{nonparametric} approach. I.e., it
doesn't require a distributional assumption. It's often used in survival
analysis for estimating summary statistics for right-censored data.
However, it can be applied to left-censored data by `flipping' the data
and subtracting from a fixed constant.

In survival analysis, Kaplan-Meier estimates the probability that an
observation will survive past a certain time. In our `flipped' context,
it gives the probability that an observation will fall below the limit
of detection.

\begin{tcolorbox}[enhanced jigsaw, leftrule=.75mm, breakable, bottomrule=.15mm, colback=white, colframe=quarto-callout-important-color-frame, coltitle=black, opacityback=0, toprule=.15mm, left=2mm, arc=.35mm, colbacktitle=quarto-callout-important-color!10!white, bottomtitle=1mm, titlerule=0mm, toptitle=1mm, title={Example: Cadmium in fish}, opacitybacktitle=0.6, rightrule=.15mm]

We can illustrate this using an example of cadmium levels in fish.
Cadmium is a heavy metal identified as having potential health risks. We
observe cadmium levels in fish livers in two different regions of the
Rocky Mountains.

Due to variation in data collection, there are four different LODs (0.2,
0.3, 0.4 and 0.6 \(\mu\)g per litre).

\begin{center}
\includegraphics[width=1.66667in,height=\textheight]{images/CadmiumData.png}
\end{center}

Plotting the data shows the potential impact of censoring. The left
panel shows all the data (plotting censored values as equal to the LOD),
while the right panel excludes those which have been censored.

\begin{center}
\includegraphics[width=3.95833in,height=\textheight]{images/CadmiumBoxplot.png}
\end{center}

We can use the NADA (Nondetects and Data Analysis) package in R. The
\texttt{cenfit} function applies the Kaplan-Meier method. This package
automatically `flips' the data, since it is designed for environmental
data.

\begin{Shaded}
\begin{Highlighting}[]
\NormalTok{blinky }\OtherTok{\textless{}{-}} \FunctionTok{cenfit}\NormalTok{(obs, censored, groups)}
\end{Highlighting}
\end{Shaded}

\begin{verbatim}
           n  n.cen  median    mean      sd
groups=1   9      3     0.4   0.589   0.352
groups=2  10      1     3.0  10.540  25.069
\end{verbatim}

There are clear differences between the locations in terms of both
median and standard deviation.

The \texttt{cendiff} function tests for significant differences between
the groups. This uses a chi-squared hypothesis test:

\begin{itemize}
\tightlist
\item
  H0: Median cadmium levels are the same in Region 1 and Region 2
\item
  H1: Median cadmium levels are different in Region 1 and Region 2
\end{itemize}

\begin{Shaded}
\begin{Highlighting}[]
\FunctionTok{cendiff}\NormalTok{(obs, censored, groups)}
\end{Highlighting}
\end{Shaded}

\begin{verbatim}
                N  Observed Expected (O-E)^2/E (O-E)^2/V
groups=1        9      2.84     6.13      1.76      7.02
groups=2        10     6.84     3.55      3.05      7.02

Chisq=7 on 1 degrees of freedom, p= 0.00808 
\end{verbatim}

We can also plot the empirical cumulative distribution function (ECDF),
taking into account the LODs. Note that this works in the opposite
direction from regular survival plots due to the `flipping' of the data.

\begin{center}
\includegraphics[width=3.80208in,height=\textheight]{images/CadmiumECDF.png}
\end{center}

\end{tcolorbox}

\textbf{Advantages}

\begin{itemize}
\tightlist
\item
  Nonparametric, so no need to assume underlying distribution.
\item
  Can easily account for multiple LODs.
\item
  Works for large numbers of censored datapoints (\(>\) 50\%).
\end{itemize}

\textbf{Disadvantages}

\begin{itemize}
\tightlist
\item
  Quite simplistic --- identical to simple substitution if we only have
  one LOD.
\item
  Less reliable for values near and below the LOD.
\item
  The mean tends to be overestimated --- need to rely on median.
\end{itemize}

\paragraph{Regression on Order Statistics
(ROS)}\label{regression-on-order-statistics-ros}

Regression on Order Statistics is a \emph{semi-parametric} approach.
I.e., it combines elements of parametric and nonparametric models. It
follows a two-step approach:

\begin{enumerate}
\def\labelenumi{\arabic{enumi}.}
\tightlist
\item
  Plot the uncensored values on a probability plot (QQ plot) and use
  linear regression to approximate the parameters of the underlying data
  distribution.
\item
  Use this fitted distribution to impute estimates for the censored
  values.
\end{enumerate}

There is an assumption that the censored measures are normally (or
lognormally) distributed.

\begin{tcolorbox}[enhanced jigsaw, leftrule=.75mm, breakable, bottomrule=.15mm, colback=white, colframe=quarto-callout-important-color-frame, coltitle=black, opacityback=0, toprule=.15mm, left=2mm, arc=.35mm, colbacktitle=quarto-callout-important-color!10!white, bottomtitle=1mm, titlerule=0mm, toptitle=1mm, title={Example: Bathing water quality (continued)}, opacitybacktitle=0.6, rightrule=.15mm]

The plot shows the uncensored points and their probability plot
regression model. The \texttt{NADA} package in R uses lognormal as
default. The plot suggests this is sensible. We then use this fitted
model to estimate the values of the censored observations based on their
normal quantiles.

\begin{center}
\includegraphics[width=3.10417in,height=\textheight]{images/ROSQQ.png}
\end{center}

We can compare our ROS approach to simple substitution for the bathing
water example used earlier. The left panel (ROS) shows no trend present,
the right panel (simple) has an artificial trend.

\begin{figure}[H]

\begin{minipage}{0.50\linewidth}
\includegraphics{images/ROSEstimates.png}\end{minipage}%
%
\begin{minipage}{0.50\linewidth}
\includegraphics{images/TSM1.png}\end{minipage}%

\end{figure}%

\end{tcolorbox}

\textbf{Advantages}

\begin{itemize}
\tightlist
\item
  Can be applied to a wide variety of environmental datasets.
\item
  Works with multiple LODs, but still not too simplistic with a single
  LOD.
\item
  Can be used with up to 80\% censored datapoints.
\end{itemize}

\textbf{Disadvantages}

\begin{itemize}
\tightlist
\item
  Semiparametric approach --- requires a distributional model to be
  assumed.
\item
  Specifically requires normality (or lognormality) for estimation of
  parameters.
\item
  Two-stage model introduces extra source of variability.
\end{itemize}

\subsection{Outliers}\label{outliers}

An outlier is an extreme or unusual observation in our dataset. These
will often (but not always) have a large influence on the outcomes of
our analysis. We have to find ways to identify and deal with outliers.

There are two main categories of outlier: (1) genuine but extreme
values, and (2) data errors. If we have a genuinely extreme value, we
should try to accommodate these in our analysis. Not doing so would mean
that we are ignoring a real feature of our data. There are robust
modelling techniques that allow us to incorporate outliers. On the other
hand, if we have an outlier due to data error, we can either try to
correct it (where possible) or remove it, since this does not reflect a
real observation.

\subsubsection{Identifying outliers}\label{identifying-outliers}

An outlier is an observation that does not follow the general trend of
the rest of the data. It is often helpful to plot your data, since
outliers are sometimes very obvious in boxplots or scatterplots or even
maps! E.g., Figure~\ref{fig-movebank} shows an Elk's animal track with
two unusual observations. To assess whether these are errors from the
tracking device or genuine unusual movement patterns (e.g., escaping a
predator), we could compute the velocity between points. This is done
using the time stamp and the step length (the straight-line distance
traveled). We can then assess how likely it is for the animal to have
moved at that calculated speed.

\begin{figure}

\centering{

\href{https://www.movebank.org/cms/webapp?gwt_fragment=page=search_map}{\includegraphics[width=3.54167in,height=\textheight]{images/outlier_animal.png}}

}

\caption{\label{fig-movebank}Elk tracking data in southwestern Alberta.
The Blue line indicates the tracking for one individual with blue and
red crosses showing the start and end point of the track respectively.}

\end{figure}%

Graphical techniques are a commonly used for detecting outliers, but
sometimes identifying outliers visually is not straightforward and can
become difficult with larger data sets. This has led to the development
of a variety of outlier detection techniques. There are also several
statistical approaches that can be applied to identify datapoints that
are significantly different from the rest. These include (but are not
limited to) \emph{tests of discordancy}, \emph{Chauvenet's criterion},
\emph{Grubb's test} and \emph{Dixon's test}.

\paragraph{Test of discordancy}\label{test-of-discordancy}

This is a hypothesis test, where the null hypothesis is that each
datapoint comes from a given data distribution F. We look at the maximum
(or minimum) value of our sample, \(x_{(n)}\) and test whether it is a
reasonable sample from F. If the maximum (or minimum) value could
reasonably come from this distribution, we have no outliers. If the
maximum (or minimum) value is an outlier, we check the second highest
(or lowest) and so on.

\paragraph{Chauvenet's criterion}\label{chauvenets-criterion}

This test assumes that our data are from a Normal distribution. For our
dataset of size \(n\), we calculate the mean (\(\mu\)) and standard
deviation (\(\sigma\)). We then use the Normal probability density
function to estimate the probability (\(p\)) of obtaining a value as
extreme our more extreme than or suspected outlier. If
\(p \times n < 0.5\), then our value is considered to be an outlier.

\paragraph{Grubbs' test}\label{grubbs-test}

Again, this test assumes that our data are from an N(\(\mu, \sigma^2\))
distribution. The Grubbs' test statistic is the largest absolute
deviation from this mean in terms of units of the standard deviation:
\[G = \frac{\max\limits_{i=1,\ldots,n} |y_i - \mu|}{\sigma}.\] This is
compared to the \(t(N-2)\) distribution to obtain a p-value. If we
identify an outlier, we repeat the process for the next most extreme
value.

\paragraph{Dixon's test}\label{dixons-test}

Rather than looking at the dataset as a whole, Dixon's test compares the
outlier to the next most extreme value. Let the \emph{gap} be the
distance between our suspected outlier and the closest value, and the
\emph{range} be the full range of the dataset. Then we compare
\(Q = \frac{\mbox{gap}}{\mbox{range}}\) to a specifically designed
reference table.

This test is only suitable if there is a single distinct suspected
outlier. If there were two similar outliers, then the gap would not be
large.

\begin{tcolorbox}[enhanced jigsaw, leftrule=.75mm, breakable, bottomrule=.15mm, colback=white, colframe=quarto-callout-important-color-frame, coltitle=black, opacityback=0, toprule=.15mm, left=2mm, arc=.35mm, colbacktitle=quarto-callout-important-color!10!white, bottomtitle=1mm, titlerule=0mm, toptitle=1mm, title={Example: Effectiveness of mosquito sprays}, opacitybacktitle=0.6, rightrule=.15mm]

The example dataset below shows the effectiveness of several mosquito
sprays.

\begin{center}
\includegraphics[width=4.95833in,height=\textheight]{images/InsectSpray.png}
\end{center}

The \texttt{outliers} package in R contains functions for Grubbs' and
Dixon's tests. Here we apply the Grubbs' test to the insect spray data.

\begin{Shaded}
\begin{Highlighting}[]
\FunctionTok{grubbs.test}\NormalTok{(InsectSprays}\SpecialCharTok{$}\NormalTok{count)}
\end{Highlighting}
\end{Shaded}

\begin{verbatim}
Grubbs test for one outlier

data:  InsectSprays$count
G = 2.29062, U = 0.92506, p-value = 0.719
alternative hypothesis: highest value 26 is an outlier
\end{verbatim}

\end{tcolorbox}

\begin{tcolorbox}[enhanced jigsaw, leftrule=.75mm, breakable, bottomrule=.15mm, colback=white, colframe=quarto-callout-tip-color-frame, coltitle=black, opacityback=0, toprule=.15mm, left=2mm, arc=.35mm, colbacktitle=quarto-callout-tip-color!10!white, bottomtitle=1mm, titlerule=0mm, toptitle=1mm, title={Exercise 10}, opacitybacktitle=0.6, rightrule=.15mm]

What conclusion can we draw from the results of the above test?

\begin{itemize}
\tightlist
\item
  \begin{enumerate}
  \def\labelenumi{(\Alph{enumi})}
  \tightlist
  \item
    There is no statistically significant evidence that the point with
    the highest value is an outlier\\
  \end{enumerate}
\item
  \begin{enumerate}
  \def\labelenumi{(\Alph{enumi})}
  \setcounter{enumi}{1}
  \tightlist
  \item
    There is statistically significant evidence that the point with the
    highest value is an outlier
  \end{enumerate}
\end{itemize}

Solution

Since the p-value for the Grubbs' test is greater than 0.05, we can
conclude that there is no statistically significant evidence that the
point with the highest value is an outlier.

\end{tcolorbox}

\begin{tcolorbox}[enhanced jigsaw, leftrule=.75mm, breakable, bottomrule=.15mm, colback=white, colframe=quarto-callout-important-color-frame, coltitle=black, opacityback=0, toprule=.15mm, left=2mm, arc=.35mm, colbacktitle=quarto-callout-important-color!10!white, bottomtitle=1mm, titlerule=0mm, toptitle=1mm, title={Example: Effectiveness of mosquito sprays (continued)}, opacitybacktitle=0.6, rightrule=.15mm]

We can also apply the Grubbs' test to the data for just one spray type
(spray C).

\begin{Shaded}
\begin{Highlighting}[]
\FunctionTok{grubbs.test}\NormalTok{(InsectSprays}\SpecialCharTok{$}\NormalTok{count[InsectSprays}\SpecialCharTok{$}\NormalTok{spray}\SpecialCharTok{==}\StringTok{"C"}\NormalTok{])}
\end{Highlighting}
\end{Shaded}

\begin{verbatim}
G = 2.48917, U = 0.38553, p-value = 0.0153
alternative hypothesis: highest value 7 is an outlier
\end{verbatim}

\end{tcolorbox}

\begin{tcolorbox}[enhanced jigsaw, leftrule=.75mm, breakable, bottomrule=.15mm, colback=white, colframe=quarto-callout-tip-color-frame, coltitle=black, opacityback=0, toprule=.15mm, left=2mm, arc=.35mm, colbacktitle=quarto-callout-tip-color!10!white, bottomtitle=1mm, titlerule=0mm, toptitle=1mm, title={Exercise 11}, opacitybacktitle=0.6, rightrule=.15mm]

What conclusion can we draw from the results of the above test?

\begin{itemize}
\tightlist
\item
  \begin{enumerate}
  \def\labelenumi{(\Alph{enumi})}
  \tightlist
  \item
    There is no statistically significant evidence that the point with
    the highest value is an outlier\\
  \end{enumerate}
\item
  \begin{enumerate}
  \def\labelenumi{(\Alph{enumi})}
  \setcounter{enumi}{1}
  \tightlist
  \item
    There is statistically significant evidence that the point with the
    highest value is an outlier
  \end{enumerate}
\end{itemize}

Solution

Since the p-value for the Grubbs' test is less than 0.05, we can
conclude that there is statistically significant evidence that the point
with the highest value is an outlier.

This is unsurprising from the plot, as since the point is drawn as a
circle above the upper whisker of the boxplot, we see that R has
identified this point as being more than 1.5 times the interquartile
range above the upper quartile. Note, however, that this does not
necessarily mean that this point is an outlier, so that a test was still
required.

\end{tcolorbox}

\subsubsection{Dealing with outliers}\label{dealing-with-outliers}

We generally do not want to discard outliers. Sometimes, we can fit a
model with and without them to assess their impact on the results.
Additionally, we can use robust alternatives to summary statistics, for
example median instead of mean, and median absolute deviation (MAD)
instead of standard deviation.

The median absolute deviation is defined as:
\[\mbox{MAD} = \mbox{median}|y_i - \tilde{y}|\] where \(\tilde{y}\) is
the median of our dataset. In other words, we find all the distances
between our points and the median, and then take the median of those
distances.

\subsection{Missing Data}\label{missing-data}

Environmental data are very prone to missing values. There is a whole
discipline of statistics related to this, and we will just touch on the
topic.

Data can be missing for any number of reasons. Adverse weather (e.g.,
rainfall, snow, drought or wind) can affect measuring equipment or
prevent access to the location. We can have missing data due to the
failure of scientific equipment or samples being lost or damaged.
Monitoring networks also often change in size over time, with data
considered ``missing'' at a certain site before that site was introduced
or after it was removed.

\begin{tcolorbox}[enhanced jigsaw, leftrule=.75mm, breakable, bottomrule=.15mm, colback=white, colframe=quarto-callout-tip-color-frame, coltitle=black, opacityback=0, toprule=.15mm, left=2mm, arc=.35mm, colbacktitle=quarto-callout-tip-color!10!white, bottomtitle=1mm, titlerule=0mm, toptitle=1mm, title={Exercise 12}, opacitybacktitle=0.6, rightrule=.15mm]

What causes of missing data do each of the three examples in the images
below illustrate? Are these data missing at random?

\begin{center}
\includegraphics[width=4.67708in,height=\textheight]{images/MissingData.png}
\end{center}

\begin{itemize}
\tightlist
\item
  Top image: MERIS data over Ireland.

  Solution

  Cloud cover means that the satellite cannot observe the lakes, oceans
  or land. These data are not missing at random, since cloud cover is
  likely to change over the seasons.
\item
  Middle images: MODIS data over the Aral Sea, for two timepoints.

  Solution

  The Aral Sea has decreased in size over the years, due to human
  impacts. Suppose that we wish to measure chlorophyll levels in a
  certain location in the lake. The changing size of the lake means that
  some locations that had data in previous years will have missing data
  (since there is no water present at that location) in more recent
  years. The data are therefore not missing at random.
\item
  Bottom image: MODIS data over Lake Superior.

  Solution

  Ice cover means that the satellite cannot observe the lake water.
  These data are not missing at random, since ice cover occurs during
  the coldest times of the year. This may be problematic, if it occurs
  during peaks or troughs of patterns of the variable that we wish to
  measure (e.g., chlorophyll may reach its lowest values at the same
  time that ice cover appears and prevents the satellite from recording
  measurements).
\end{itemize}

\end{tcolorbox}

The technique we use to deal with missing data depends on the type of
missingness. If there are a handful of datapoints missing at random, we
can essentially ignore this and carry out our analysis as usual.
However, if they are missing in some sort of systematic way (e.g., a
whole month missing due to bad weather), we may instead look at some
form of \textbf{imputation}.

\subsubsection{Imputation}\label{imputation}

Imputation is a process which involves predicting the missing values via
some form of statistical method. There are two main forms of imputation:

\begin{itemize}
\tightlist
\item
  \textbf{Single imputation} involves generating one value in place of
  each missing value.
\item
  \textbf{Multiple imputation} involves generating several values in
  place of each missing value.
\end{itemize}

Single imputation has the advantage of being simpler, allowing for a
straightforward analysis once the missing values have been estimated.
Multiple imputation does a better job of accounting for the uncertainty
of the imputation process, but it makes the final analysis more complex.

Our approach for generating the imputed value will vary depending on the
context. In the simplest case, we may replace missing values with the
overall mean --- usually only if we have very limited information. More
commonly, we may use neighbouring values, or some form of seasonal mean.
These will usually work reasonably well as long as we do not have too
many missing data. A more complex approach is to fit a more general
statistical model, perhaps taking account of other variables and/or
using random components.

To handle missing data, models that incorporate it should be chosen over
simply ignoring the missing values. If the missing data can be predicted
based on the observed data, imputation models can effectively estimate
them.

\subsection*{References}\label{references}
\addcontentsline{toc}{subsection}{References}

\begin{itemize}
\item
  For more details on \textbf{uncertainties and their propagation}:

  Kirkup, L., \& Frenkel, B. (2006). \emph{An introduction to
  uncertainty in measurement using the GUM (guide to the expression of
  uncertainty in measurement)}. Cambridge University Press.
  \href{https://go.exlibris.link/HnNPJjFZ}{Available from the University
  Library here}.
\item
  For more details on \textbf{censored data}:

  Helsel, D. R. (2012). \emph{Statistics for censored environmental data
  using Minitab and R} (2nd ed.). Wiley.
  \href{https://go.exlibris.link/nSYV0zLq}{Available from the University
  Library here}.
\end{itemize}

\phantomsection\label{refs}
\begin{CSLReferences}{1}{0}
\bibitem[\citeproctext]{ref-botella2020}
Botella, Christophe, Alexis Joly, Pascal Monestiez, Pierre Bonnet, and
François Munoz. 2020. {``Bias in Presence-Only Niche Models Related to
Sampling Effort and Species Niches: Lessons for Background Point
Selection.''} Edited by Mirko Di Febbraro. \emph{PLOS ONE} 15 (5):
e0232078. \url{https://doi.org/10.1371/journal.pone.0232078}.

\bibitem[\citeproctext]{ref-pyke2010}
Pyke, Graham H., and Paul R. Ehrlich. 2010. {``Biological Collections
and Ecological/Environmental Research: A Review, Some Observations and a
Look to the Future.''} \emph{Biological Reviews} 85 (2): 247--66.
\url{https://doi.org/10.1111/j.1469-185x.2009.00098.x}.

\bibitem[\citeproctext]{ref-rennie2020}
Rennie, Susannah, Chris Andrews, Sarah Atkinson, Deborah Beaumont, Sue
Benham, Vic Bowmaker, Jan Dick, et al. 2020. {``The UK Environmental
Change Network Datasets {\textendash} Integrated and Co-Located Data for
Long-Term Environmental Research (1993{\textendash}2015).''} \emph{Earth
System Science Data} 12 (1): 87--107.
\url{https://doi.org/10.5194/essd-12-87-2020}.

\end{CSLReferences}



\end{document}
