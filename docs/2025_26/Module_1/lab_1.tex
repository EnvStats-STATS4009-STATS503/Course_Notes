% Options for packages loaded elsewhere
\PassOptionsToPackage{unicode}{hyperref}
\PassOptionsToPackage{hyphens}{url}
\PassOptionsToPackage{dvipsnames,svgnames,x11names}{xcolor}
%
\documentclass[
  letterpaper,
  DIV=11,
  numbers=noendperiod]{scrartcl}

\usepackage{amsmath,amssymb}
\usepackage{iftex}
\ifPDFTeX
  \usepackage[T1]{fontenc}
  \usepackage[utf8]{inputenc}
  \usepackage{textcomp} % provide euro and other symbols
\else % if luatex or xetex
  \usepackage{unicode-math}
  \defaultfontfeatures{Scale=MatchLowercase}
  \defaultfontfeatures[\rmfamily]{Ligatures=TeX,Scale=1}
\fi
\usepackage{lmodern}
\ifPDFTeX\else  
    % xetex/luatex font selection
\fi
% Use upquote if available, for straight quotes in verbatim environments
\IfFileExists{upquote.sty}{\usepackage{upquote}}{}
\IfFileExists{microtype.sty}{% use microtype if available
  \usepackage[]{microtype}
  \UseMicrotypeSet[protrusion]{basicmath} % disable protrusion for tt fonts
}{}
\makeatletter
\@ifundefined{KOMAClassName}{% if non-KOMA class
  \IfFileExists{parskip.sty}{%
    \usepackage{parskip}
  }{% else
    \setlength{\parindent}{0pt}
    \setlength{\parskip}{6pt plus 2pt minus 1pt}}
}{% if KOMA class
  \KOMAoptions{parskip=half}}
\makeatother
\usepackage{xcolor}
\setlength{\emergencystretch}{3em} % prevent overfull lines
\setcounter{secnumdepth}{5}
% Make \paragraph and \subparagraph free-standing
\makeatletter
\ifx\paragraph\undefined\else
  \let\oldparagraph\paragraph
  \renewcommand{\paragraph}{
    \@ifstar
      \xxxParagraphStar
      \xxxParagraphNoStar
  }
  \newcommand{\xxxParagraphStar}[1]{\oldparagraph*{#1}\mbox{}}
  \newcommand{\xxxParagraphNoStar}[1]{\oldparagraph{#1}\mbox{}}
\fi
\ifx\subparagraph\undefined\else
  \let\oldsubparagraph\subparagraph
  \renewcommand{\subparagraph}{
    \@ifstar
      \xxxSubParagraphStar
      \xxxSubParagraphNoStar
  }
  \newcommand{\xxxSubParagraphStar}[1]{\oldsubparagraph*{#1}\mbox{}}
  \newcommand{\xxxSubParagraphNoStar}[1]{\oldsubparagraph{#1}\mbox{}}
\fi
\makeatother

\usepackage{color}
\usepackage{fancyvrb}
\newcommand{\VerbBar}{|}
\newcommand{\VERB}{\Verb[commandchars=\\\{\}]}
\DefineVerbatimEnvironment{Highlighting}{Verbatim}{commandchars=\\\{\}}
% Add ',fontsize=\small' for more characters per line
\usepackage{framed}
\definecolor{shadecolor}{RGB}{241,243,245}
\newenvironment{Shaded}{\begin{snugshade}}{\end{snugshade}}
\newcommand{\AlertTok}[1]{\textcolor[rgb]{0.68,0.00,0.00}{#1}}
\newcommand{\AnnotationTok}[1]{\textcolor[rgb]{0.37,0.37,0.37}{#1}}
\newcommand{\AttributeTok}[1]{\textcolor[rgb]{0.40,0.45,0.13}{#1}}
\newcommand{\BaseNTok}[1]{\textcolor[rgb]{0.68,0.00,0.00}{#1}}
\newcommand{\BuiltInTok}[1]{\textcolor[rgb]{0.00,0.23,0.31}{#1}}
\newcommand{\CharTok}[1]{\textcolor[rgb]{0.13,0.47,0.30}{#1}}
\newcommand{\CommentTok}[1]{\textcolor[rgb]{0.37,0.37,0.37}{#1}}
\newcommand{\CommentVarTok}[1]{\textcolor[rgb]{0.37,0.37,0.37}{\textit{#1}}}
\newcommand{\ConstantTok}[1]{\textcolor[rgb]{0.56,0.35,0.01}{#1}}
\newcommand{\ControlFlowTok}[1]{\textcolor[rgb]{0.00,0.23,0.31}{\textbf{#1}}}
\newcommand{\DataTypeTok}[1]{\textcolor[rgb]{0.68,0.00,0.00}{#1}}
\newcommand{\DecValTok}[1]{\textcolor[rgb]{0.68,0.00,0.00}{#1}}
\newcommand{\DocumentationTok}[1]{\textcolor[rgb]{0.37,0.37,0.37}{\textit{#1}}}
\newcommand{\ErrorTok}[1]{\textcolor[rgb]{0.68,0.00,0.00}{#1}}
\newcommand{\ExtensionTok}[1]{\textcolor[rgb]{0.00,0.23,0.31}{#1}}
\newcommand{\FloatTok}[1]{\textcolor[rgb]{0.68,0.00,0.00}{#1}}
\newcommand{\FunctionTok}[1]{\textcolor[rgb]{0.28,0.35,0.67}{#1}}
\newcommand{\ImportTok}[1]{\textcolor[rgb]{0.00,0.46,0.62}{#1}}
\newcommand{\InformationTok}[1]{\textcolor[rgb]{0.37,0.37,0.37}{#1}}
\newcommand{\KeywordTok}[1]{\textcolor[rgb]{0.00,0.23,0.31}{\textbf{#1}}}
\newcommand{\NormalTok}[1]{\textcolor[rgb]{0.00,0.23,0.31}{#1}}
\newcommand{\OperatorTok}[1]{\textcolor[rgb]{0.37,0.37,0.37}{#1}}
\newcommand{\OtherTok}[1]{\textcolor[rgb]{0.00,0.23,0.31}{#1}}
\newcommand{\PreprocessorTok}[1]{\textcolor[rgb]{0.68,0.00,0.00}{#1}}
\newcommand{\RegionMarkerTok}[1]{\textcolor[rgb]{0.00,0.23,0.31}{#1}}
\newcommand{\SpecialCharTok}[1]{\textcolor[rgb]{0.37,0.37,0.37}{#1}}
\newcommand{\SpecialStringTok}[1]{\textcolor[rgb]{0.13,0.47,0.30}{#1}}
\newcommand{\StringTok}[1]{\textcolor[rgb]{0.13,0.47,0.30}{#1}}
\newcommand{\VariableTok}[1]{\textcolor[rgb]{0.07,0.07,0.07}{#1}}
\newcommand{\VerbatimStringTok}[1]{\textcolor[rgb]{0.13,0.47,0.30}{#1}}
\newcommand{\WarningTok}[1]{\textcolor[rgb]{0.37,0.37,0.37}{\textit{#1}}}

\providecommand{\tightlist}{%
  \setlength{\itemsep}{0pt}\setlength{\parskip}{0pt}}\usepackage{longtable,booktabs,array}
\usepackage{calc} % for calculating minipage widths
% Correct order of tables after \paragraph or \subparagraph
\usepackage{etoolbox}
\makeatletter
\patchcmd\longtable{\par}{\if@noskipsec\mbox{}\fi\par}{}{}
\makeatother
% Allow footnotes in longtable head/foot
\IfFileExists{footnotehyper.sty}{\usepackage{footnotehyper}}{\usepackage{footnote}}
\makesavenoteenv{longtable}
\usepackage{graphicx}
\makeatletter
\newsavebox\pandoc@box
\newcommand*\pandocbounded[1]{% scales image to fit in text height/width
  \sbox\pandoc@box{#1}%
  \Gscale@div\@tempa{\textheight}{\dimexpr\ht\pandoc@box+\dp\pandoc@box\relax}%
  \Gscale@div\@tempb{\linewidth}{\wd\pandoc@box}%
  \ifdim\@tempb\p@<\@tempa\p@\let\@tempa\@tempb\fi% select the smaller of both
  \ifdim\@tempa\p@<\p@\scalebox{\@tempa}{\usebox\pandoc@box}%
  \else\usebox{\pandoc@box}%
  \fi%
}
% Set default figure placement to htbp
\def\fps@figure{htbp}
\makeatother
% definitions for citeproc citations
\NewDocumentCommand\citeproctext{}{}
\NewDocumentCommand\citeproc{mm}{%
  \begingroup\def\citeproctext{#2}\cite{#1}\endgroup}
\makeatletter
 % allow citations to break across lines
 \let\@cite@ofmt\@firstofone
 % avoid brackets around text for \cite:
 \def\@biblabel#1{}
 \def\@cite#1#2{{#1\if@tempswa , #2\fi}}
\makeatother
\newlength{\cslhangindent}
\setlength{\cslhangindent}{1.5em}
\newlength{\csllabelwidth}
\setlength{\csllabelwidth}{3em}
\newenvironment{CSLReferences}[2] % #1 hanging-indent, #2 entry-spacing
 {\begin{list}{}{%
  \setlength{\itemindent}{0pt}
  \setlength{\leftmargin}{0pt}
  \setlength{\parsep}{0pt}
  % turn on hanging indent if param 1 is 1
  \ifodd #1
   \setlength{\leftmargin}{\cslhangindent}
   \setlength{\itemindent}{-1\cslhangindent}
  \fi
  % set entry spacing
  \setlength{\itemsep}{#2\baselineskip}}}
 {\end{list}}
\usepackage{calc}
\newcommand{\CSLBlock}[1]{\hfill\break\parbox[t]{\linewidth}{\strut\ignorespaces#1\strut}}
\newcommand{\CSLLeftMargin}[1]{\parbox[t]{\csllabelwidth}{\strut#1\strut}}
\newcommand{\CSLRightInline}[1]{\parbox[t]{\linewidth - \csllabelwidth}{\strut#1\strut}}
\newcommand{\CSLIndent}[1]{\hspace{\cslhangindent}#1}

% load packages
\usepackage{geometry}
\usepackage{xcolor}
\usepackage{eso-pic}
\usepackage{fancyhdr}
\usepackage{sectsty}
\usepackage{fontspec}
\usepackage{titlesec}

%% Set page size with a wider right margin
\geometry{a4paper, total={170mm,257mm}, left=20mm, top=20mm, bottom=20mm, right=50mm}

%% Let's define some colours
\definecolor{uniblue}{HTML}{003865}
\definecolor{burgundy}{HTML}{7D2239}
\definecolor{cobalt}{HTML}{005C8A}
\definecolor{lavender}{HTML}{5B4D94}
\definecolor{leaf}{HTML}{006630}
\definecolor{moss}{HTML}{385A4F}
\definecolor{pillarbox}{HTML}{B30C00}
\definecolor{rust}{HTML}{9A3A06}
\definecolor{sandstone}{HTML}{52473B}
\definecolor{skyblue}{HTML}{005398}
\definecolor{slate}{HTML}{4F5961}
\definecolor{thistle}{HTML}{951272}

%\definecolor{light}{HTML}{E6E6FA} % original from template - redefined below as uni blue at 10 percent:
\colorlet{light}{uniblue!10}
%\definecolor{highlight}{HTML}{800080} % original from template - redefined below as uni's skyblue:
\colorlet{highlight}{skyblue}
%\definecolor{dark}{HTML}{330033} % original from template - redefined below as uni blue at 100 percent:
\colorlet{dark}{uniblue}

%% Let's add the border on the right hand side 
\AddToShipoutPicture{% 
    \AtPageLowerLeft{% 
        \put(\LenToUnit{\dimexpr\paperwidth-3cm},0){% 
            \color{light}\rule{3cm}{\LenToUnit\paperheight}%
          }%
     }%
     % logo
    \AtPageLowerLeft{% start the bar at the bottom right of the page
        \put(\LenToUnit{\dimexpr\paperwidth-2.25cm},27.2cm){% move it to the top right
            \color{light}\includegraphics[width=2.25cm]{_extensions/nrennie/PrettyPDF/uni_logo_boxed.jpg}
          }%
     }%
}

%% Style the page number
\fancypagestyle{mystyle}{
  \fancyhf{}
  \renewcommand\headrulewidth{0pt}
  \fancyfoot[R]{\thepage}
  \fancyfootoffset{3.5cm}
}
\setlength{\footskip}{20pt}

%% style the chapter/section fonts
\chapterfont{\color{uniblue}\fontsize{20}{16.8}\selectfont}
\sectionfont{\color{uniblue}\fontsize{20}{16.8}\selectfont}
\subsectionfont{\color{skyblue}\fontsize{14}{16.8}\selectfont}
\titleformat{\subsection}
  {\color{uniblue!90}\sffamily\Large\bfseries}{\thesubsection}{1em}{}[{\titlerule[0.8pt]}]
\subsubsectionfont{\color{cobalt}}

\renewcommand\thesection{\color{slate}\arabic{section}}
  
% left align title
\makeatletter
\renewcommand{\maketitle}{\bgroup\setlength{\parindent}{0pt}
\begin{flushleft}
  {\color{uniblue}\sffamily\huge\textbf{\@title}} \vspace{0.3cm} \newline
  {\Large {\@subtitle}} \newline
  \@author
\end{flushleft}\egroup
}
\makeatother

%%% Use some custom fonts
\setsansfont{Ubuntu}[
    Path=_extensions/nrennie/PrettyPDF/Ubuntu/,
    Scale=0.9,
    Extension = .ttf,
    UprightFont=*-Regular,
    BoldFont=*-Bold,
    ItalicFont=*-Italic,
    ]

\setmainfont{Ubuntu}[
    Path=_extensions/nrennie/PrettyPDF/Ubuntu/,
    Scale=0.9,
    Extension = .ttf,
    UprightFont=*-Regular,
    BoldFont=*-Bold,
    ItalicFont=*-Italic,
    ]
\KOMAoption{captions}{tableheading}
\makeatletter
\@ifpackageloaded{tcolorbox}{}{\usepackage[skins,breakable]{tcolorbox}}
\@ifpackageloaded{fontawesome5}{}{\usepackage{fontawesome5}}
\definecolor{quarto-callout-color}{HTML}{909090}
\definecolor{quarto-callout-note-color}{HTML}{0758E5}
\definecolor{quarto-callout-important-color}{HTML}{CC1914}
\definecolor{quarto-callout-warning-color}{HTML}{EB9113}
\definecolor{quarto-callout-tip-color}{HTML}{00A047}
\definecolor{quarto-callout-caution-color}{HTML}{FC5300}
\definecolor{quarto-callout-color-frame}{HTML}{acacac}
\definecolor{quarto-callout-note-color-frame}{HTML}{4582ec}
\definecolor{quarto-callout-important-color-frame}{HTML}{d9534f}
\definecolor{quarto-callout-warning-color-frame}{HTML}{f0ad4e}
\definecolor{quarto-callout-tip-color-frame}{HTML}{02b875}
\definecolor{quarto-callout-caution-color-frame}{HTML}{fd7e14}
\makeatother
\makeatletter
\@ifpackageloaded{caption}{}{\usepackage{caption}}
\AtBeginDocument{%
\ifdefined\contentsname
  \renewcommand*\contentsname{Table of contents}
\else
  \newcommand\contentsname{Table of contents}
\fi
\ifdefined\listfigurename
  \renewcommand*\listfigurename{List of Figures}
\else
  \newcommand\listfigurename{List of Figures}
\fi
\ifdefined\listtablename
  \renewcommand*\listtablename{List of Tables}
\else
  \newcommand\listtablename{List of Tables}
\fi
\ifdefined\figurename
  \renewcommand*\figurename{Figure}
\else
  \newcommand\figurename{Figure}
\fi
\ifdefined\tablename
  \renewcommand*\tablename{Table}
\else
  \newcommand\tablename{Table}
\fi
}
\@ifpackageloaded{float}{}{\usepackage{float}}
\floatstyle{ruled}
\@ifundefined{c@chapter}{\newfloat{codelisting}{h}{lop}}{\newfloat{codelisting}{h}{lop}[chapter]}
\floatname{codelisting}{Listing}
\newcommand*\listoflistings{\listof{codelisting}{List of Listings}}
\makeatother
\makeatletter
\makeatother
\makeatletter
\@ifpackageloaded{caption}{}{\usepackage{caption}}
\@ifpackageloaded{subcaption}{}{\usepackage{subcaption}}
\makeatother
\makeatletter
\@ifpackageloaded{tcolorbox}{}{\usepackage[skins,breakable]{tcolorbox}}
\makeatother
\makeatletter
\@ifundefined{shadecolor}{\definecolor{shadecolor}{rgb}{.97, .97, .97}}{}
\makeatother
\makeatletter
\@ifundefined{codebgcolor}{\definecolor{codebgcolor}{named}{light}}{}
\makeatother
\makeatletter
\ifdefined\Shaded\renewenvironment{Shaded}{\begin{tcolorbox}[enhanced, boxrule=0pt, colback={codebgcolor}, sharp corners, frame hidden, breakable]}{\end{tcolorbox}}\fi
\makeatother

\usepackage{bookmark}

\IfFileExists{xurl.sty}{\usepackage{xurl}}{} % add URL line breaks if available
\urlstyle{same} % disable monospaced font for URLs
\hypersetup{
  pdftitle={Lab session 1},
  colorlinks=true,
  linkcolor={highlight},
  filecolor={Maroon},
  citecolor={Blue},
  urlcolor={highlight},
  pdfcreator={LaTeX via pandoc}}


\title{Lab session 1}
\author{}
\date{}

\begin{document}
\maketitle

\pagestyle{mystyle}


\textbf{Aim of this practical session:}

In this first practical we are going

\begin{itemize}
\tightlist
\item
  To explore graphically some environmental data sets.
\item
  To explore different estimation techniques for censored data
  (Section~\ref{sec-LoD}).
\item
  Design a monitoring network using the GRTS algorithm
  (Section~\ref{sec-design}).
\end{itemize}

\section{Part 1: Limits of detection}\label{sec-LoD}

Data reported at a limit of detection represent a common problem in
environmental studies. For this problem, measurements on ammonia from a
stretch of river will be studied, with 31\% censored data. The data are
available in the file \texttt{SiteSoar.csv}.

The variables in the dataset are as follows:

\begin{longtable}[]{@{}
  >{\raggedright\arraybackslash}p{(\linewidth - 2\tabcolsep) * \real{0.5000}}
  >{\raggedright\arraybackslash}p{(\linewidth - 2\tabcolsep) * \real{0.5000}}@{}}
\toprule\noalign{}
\begin{minipage}[b]{\linewidth}\raggedright
Variable
\end{minipage} & \begin{minipage}[b]{\linewidth}\raggedright
Meaning
\end{minipage} \\
\midrule\noalign{}
\endhead
\bottomrule\noalign{}
\endlastfoot
\texttt{year} & Year of observation \\
\texttt{month} & Month of observation \\
\texttt{day} & Day of observation (within month) \\
\texttt{doy} & Day of observation (from start of year) \\
\texttt{Ammonia} & Ammonia level \\
\texttt{censored} & Censoring indicator (\texttt{FALSE}: no censoring;
\texttt{TRUE}: censored at limit of detection) \\
\end{longtable}

Here we will consider three different methods of dealing with censored
data (i.e., three strategies for replacement). These are:

\begin{itemize}
\tightlist
\item
  Kaplan-Meier estimates,
\item
  Maximum Likelihood Estimators (MLEs), and
\item
  Regression on Order Statistics (ROS). Note: ROS computes a linear
  regression for data (or their logs) versus their normal scores (from a
  Normal probability plot).
\end{itemize}

\begin{Shaded}
\begin{Highlighting}[]
\CommentTok{\# Load required R packages:}

\FunctionTok{library}\NormalTok{(NADA)       }\CommentTok{\# for analyzing censored observations}
\FunctionTok{library}\NormalTok{(ggplot2)    }\CommentTok{\# For visualizing our data}
\FunctionTok{library}\NormalTok{(patchwork)  }\CommentTok{\# For plotting multiple ggplot objects together}

\CommentTok{\# Read in data (making sure to set the working directory to the }
\CommentTok{\# appropriate location):}
\NormalTok{Soar }\OtherTok{\textless{}{-}} \FunctionTok{read.csv}\NormalTok{(}\StringTok{"datasets/SiteSoar.csv"}\NormalTok{, }\AttributeTok{header =} \ConstantTok{TRUE}\NormalTok{)}

\CommentTok{\# Examine structure of the dataset:}
\FunctionTok{str}\NormalTok{(Soar)}
\end{Highlighting}
\end{Shaded}

\begin{verbatim}
'data.frame':   224 obs. of  6 variables:
 $ year    : int  1995 1995 1995 1996 2004 2003 2002 2002 2002 2000 ...
 $ month   : int  11 9 4 3 3 7 12 11 7 11 ...
 $ day     : int  21 30 24 20 24 24 22 13 25 2 ...
 $ doy     : int  326 274 115 80 84 206 357 318 207 307 ...
 $ Ammonia : num  0.064 0.03 0.03 0.049 0.03 0.03 0.174 0.115 0.03 0.091 ...
 $ censored: logi  FALSE TRUE TRUE FALSE TRUE TRUE ...
\end{verbatim}

\subsection{Visualizing our data}\label{visualizing-our-data}

Lets create some exploratory plots to visualize the relationship between
Ammonia and time.

\begin{Shaded}
\begin{Highlighting}[]
\FunctionTok{ggplot}\NormalTok{(}\AttributeTok{data =}\NormalTok{ Soar,}\FunctionTok{aes}\NormalTok{(}\AttributeTok{y =}\NormalTok{ Ammonia, }\AttributeTok{x =}\NormalTok{ year))}\SpecialCharTok{+}\FunctionTok{geom\_point}\NormalTok{()}\SpecialCharTok{+}
  \FunctionTok{ggplot}\NormalTok{(}\AttributeTok{data =}\NormalTok{ Soar,}\FunctionTok{aes}\NormalTok{(}\AttributeTok{y =}\NormalTok{ Ammonia, }\AttributeTok{x =}\NormalTok{ month))}\SpecialCharTok{+}\FunctionTok{geom\_point}\NormalTok{()}\SpecialCharTok{+}
  \FunctionTok{ggplot}\NormalTok{(}\AttributeTok{data =}\NormalTok{ Soar,}\FunctionTok{aes}\NormalTok{(}\AttributeTok{y =}\NormalTok{ Ammonia, }\AttributeTok{x =}\NormalTok{ day))}\SpecialCharTok{+}\FunctionTok{geom\_point}\NormalTok{()}
\end{Highlighting}
\end{Shaded}

\pandocbounded{\includegraphics[keepaspectratio]{lab_1_files/figure-pdf/unnamed-chunk-3-1.pdf}}

\begin{tcolorbox}[enhanced jigsaw, coltitle=black, colback=white, titlerule=0mm, colframe=quarto-callout-tip-color-frame, left=2mm, opacityback=0, rightrule=.15mm, bottomtitle=1mm, bottomrule=.15mm, leftrule=.75mm, title={Question}, arc=.35mm, colbacktitle=quarto-callout-tip-color!10!white, toptitle=1mm, toprule=.15mm, breakable, opacitybacktitle=0.6]

What do you notice about the relationship between Ammonia and time? Do
you notice any patterns or \emph{unusual} observations?

See Solution

We would like to focus here on the relationship between Ammonia and
time. The key thing that we notice is the single point that has a high
value of Ammonia. This seems like it could be an error rather than a
true value (possibly a decimal point in the wrong place). We can also
see this by looking at the dataset --- type \texttt{View(Soar)}.

We could justify removing the outlier, since its value is an order of
magnitude larger than the other values, so is probably an error. We
could alternatively justify not removing the outlier, since we would
need to first consult a subject-matter expert before concluding that it
was an unlikely value of Ammonia. (It's up to you which you choose, as
long as you justify it. I might remove this value.)

There are no obvious trends or seasonal patterns in the data, from these
plots.

\end{tcolorbox}

\begin{tcolorbox}[enhanced jigsaw, coltitle=black, colback=white, titlerule=0mm, colframe=quarto-callout-warning-color-frame, left=2mm, opacityback=0, rightrule=.15mm, bottomtitle=1mm, bottomrule=.15mm, leftrule=.75mm, title={Task 1}, arc=.35mm, colbacktitle=quarto-callout-warning-color!10!white, toptitle=1mm, toprule=.15mm, breakable, opacitybacktitle=0.6]

By looking at the exploratory plots there seems to be an outlier given
by an Ammonia value that is larger than 2. Suppose we want to remove
this from any further analysis (provided we have reasonable
justification to do it). Remove the outlier from the dataset, you can
use the \texttt{filter()} function from the \texttt{dplyr} package to
achieve this.

Take hint

We can use the \texttt{filter} function from \texttt{dplyr} to subset
our data according to a logical statement. Load the \texttt{dplyr}
library and type \texttt{?filter} for further details.

Click here to see the solution

\begin{Shaded}
\begin{Highlighting}[]
\FunctionTok{library}\NormalTok{(dplyr)}

\NormalTok{Soar }\OtherTok{\textless{}{-}}\NormalTok{ Soar }\SpecialCharTok{\%\textgreater{}\%} \FunctionTok{filter}\NormalTok{(Ammonia }\SpecialCharTok{\textless{}}\DecValTok{2}\NormalTok{)}
\end{Highlighting}
\end{Shaded}

\end{tcolorbox}

Lets look into some further statistics using the \texttt{pctCen} and
\texttt{censummary} functions from \texttt{NADA}:

\begin{Shaded}
\begin{Highlighting}[]
\CommentTok{\# Further summary statistics:}
\FunctionTok{pctCen}\NormalTok{(Soar}\SpecialCharTok{$}\NormalTok{Ammonia, Soar}\SpecialCharTok{$}\NormalTok{censored)     }\DocumentationTok{\#\# percent of censored data}
\end{Highlighting}
\end{Shaded}

\begin{verbatim}
[1] 31.39013
\end{verbatim}

\begin{Shaded}
\begin{Highlighting}[]
\FunctionTok{censummary}\NormalTok{(Soar}\SpecialCharTok{$}\NormalTok{Ammonia, Soar}\SpecialCharTok{$}\NormalTok{censored) }\DocumentationTok{\#\# like summary cmd but for censored data}
\end{Highlighting}
\end{Shaded}

\begin{verbatim}
all:
        n     n.cen   pct.cen       min       max 
223.00000  70.00000  31.39013   0.01200   0.27000 

limits:
  limit  n uncen   pexceed
1  0.00  0    22 1.0000000
2  0.03 70   131 0.5874439
\end{verbatim}

\textbf{What does this tell us?} You can ignore the limit of
\texttt{0.00}, since the function adds this by default if there is only
one positive limit in the dataset --- you can see that \texttt{n} takes
the value 0 here, meaning that there are no data points censored at 0.

In the first line, \texttt{uncen} tells us that there are 22 data points
that have values between 0 and the LoD of 0.03, so this emphasises that
just because we have a LoD in the data, that doesn't mean that all data
points will be censored here, since different instruments (that have
different LoDs) may have been used to generate the same dataset.

In the second line, we see that there are 70 data points censored at the
LoD of 0.03. There are 131 data points that take values above 0.03.
\texttt{pexceed} tells us that the probability of exceeding the LoD is
\(131/(131+70+22) = 0.589\).

\subsection{Dealing with censored
observations}\label{dealing-with-censored-observations}

\subsubsection{ROS}\label{ros}

First, we will implement Regression on Order Statistics (ROS) using the
\texttt{cenros} function

\begin{Shaded}
\begin{Highlighting}[]
\DocumentationTok{\#\# 1. ROS:}
\NormalTok{ROS }\OtherTok{\textless{}{-}} \FunctionTok{cenros}\NormalTok{(Soar}\SpecialCharTok{$}\NormalTok{Ammonia, Soar}\SpecialCharTok{$}\NormalTok{censored) }
\FunctionTok{plot}\NormalTok{(ROS, }\AttributeTok{plot.censored =} \ConstantTok{TRUE}\NormalTok{) }\DocumentationTok{\#\# plots the modelled censored }
\end{Highlighting}
\end{Shaded}

\pandocbounded{\includegraphics[keepaspectratio]{lab_1_files/figure-pdf/unnamed-chunk-6-1.pdf}}

\textbf{Interpretation of plot:} Here, do the filled black circles
generally follow a straight line? Yes, so the model seems reasonable to
use here.

The imputed values are the empty black circles, and these seem
reasonable --- we have already seen that the model seems to be
appropriate, so this should be the case.

\begin{Shaded}
\begin{Highlighting}[]
\FunctionTok{summary}\NormalTok{(ROS) }\DocumentationTok{\#\# more info about the ROS regression}
\end{Highlighting}
\end{Shaded}

\begin{verbatim}

Call:
lm(formula = obs.transformed ~ pp.nq, na.action = na.action)

Residuals:
     Min       1Q   Median       3Q      Max 
-0.23643 -0.09932 -0.00931  0.08775  0.51484 

Coefficients:
            Estimate Std. Error t value Pr(>|t|)    
(Intercept) -3.21742    0.01090 -295.29   <2e-16 ***
pp.nq        0.81982    0.01175   69.75   <2e-16 ***
---
Signif. codes:  0 '***' 0.001 '**' 0.01 '*' 0.05 '.' 0.1 ' ' 1

Residual standard error: 0.1194 on 151 degrees of freedom
Multiple R-squared:  0.9699,    Adjusted R-squared:  0.9697 
F-statistic:  4866 on 1 and 151 DF,  p-value: < 2.2e-16
\end{verbatim}

\texttt{pp.nq} should be statistically significant --- it is (p
\textless{} 0.05).

\begin{Shaded}
\begin{Highlighting}[]
\FunctionTok{print}\NormalTok{(ROS)   }\DocumentationTok{\#\# prints a simple summary of the ROS model.}
\end{Highlighting}
\end{Shaded}

\begin{verbatim}
           n        n.cen       median         mean           sd 
223.00000000  70.00000000   0.03600000   0.05577297   0.04960383 
\end{verbatim}

This is what we really care about here --- this represents the summary
of the censored data, with mean 0.056 and standard deviation 0.049.
We'll use this to generate our imputed values later.

\subsubsection{Kaplan-Meier}\label{kaplan-meier}

Now lets look into the Kaplan-Meier

\begin{Shaded}
\begin{Highlighting}[]
\DocumentationTok{\#\# 2. Kaplan{-}Meier:}
\NormalTok{KM }\OtherTok{\textless{}{-}} \FunctionTok{cenfit}\NormalTok{(Soar}\SpecialCharTok{$}\NormalTok{Ammonia, Soar}\SpecialCharTok{$}\NormalTok{censored)  }\DocumentationTok{\#\# constructs a Kaplan{-}Meier model}
\FunctionTok{plot}\NormalTok{(KM)   }\DocumentationTok{\#\# survival function plot}
\end{Highlighting}
\end{Shaded}

\pandocbounded{\includegraphics[keepaspectratio]{lab_1_files/figure-pdf/unnamed-chunk-9-1.pdf}}

On the x-axis we have the Concentration values and on the y-axis we have
the proportion of observations \(\leq\) to each concentration. Don't
worry too much about the interpretation of this plot. It tells us about
the probability of the data being at most a certain value (using the
information that we have from the censored and uncensored data). What we
are actually interested is in the following output:

\begin{Shaded}
\begin{Highlighting}[]
\FunctionTok{print}\NormalTok{(KM)}
\end{Highlighting}
\end{Shaded}

\begin{verbatim}
           n        n.cen       median         mean           sd 
223.00000000  70.00000000   0.03600000   0.05590147   0.04967880 
\end{verbatim}

\subsubsection{MLE}\label{mle}

\begin{Shaded}
\begin{Highlighting}[]
\DocumentationTok{\#\# 3. MLE}
\NormalTok{MLE }\OtherTok{\textless{}{-}} \FunctionTok{cenmle}\NormalTok{(Soar}\SpecialCharTok{$}\NormalTok{Ammonia, Soar}\SpecialCharTok{$}\NormalTok{censored) }\DocumentationTok{\#\# constructs a Maximum Likelihood model}
\FunctionTok{plot}\NormalTok{(MLE)}
\end{Highlighting}
\end{Shaded}

\pandocbounded{\includegraphics[keepaspectratio]{lab_1_files/figure-pdf/unnamed-chunk-11-1.pdf}}

\begin{tcolorbox}[enhanced jigsaw, coltitle=black, colback=white, titlerule=0mm, colframe=quarto-callout-tip-color-frame, left=2mm, opacityback=0, rightrule=.15mm, bottomtitle=1mm, bottomrule=.15mm, leftrule=.75mm, title={Question}, arc=.35mm, colbacktitle=quarto-callout-tip-color!10!white, toptitle=1mm, toprule=.15mm, breakable, opacitybacktitle=0.6]

What can you tell about the plot above?

See Solution

Most of the points follow the fitted line, so this model appears
appropriate. (We should not worry too much about the few points that lie
far from the line --- this doesn't mean that our model is not
appropriate.)

\end{tcolorbox}

Lets look into some summaries for the model

\begin{Shaded}
\begin{Highlighting}[]
\FunctionTok{summary}\NormalTok{(MLE)}
\end{Highlighting}
\end{Shaded}

\begin{verbatim}
             Value Std. Error      z       p
(Intercept) -3.252     0.0612 -53.16 0.00000
Log(scale)  -0.167     0.0604  -2.77 0.00559

Scale = 0.846 

Log Normal distribution
Loglik(model)= 189.9   Loglik(intercept only)= 189.9 
Loglik-r:  0 

Number of Newton-Raphson Iterations: 5 
n = 223 
\end{verbatim}

\begin{Shaded}
\begin{Highlighting}[]
\FunctionTok{print}\NormalTok{(MLE)}
\end{Highlighting}
\end{Shaded}

\begin{verbatim}
           n        n.cen       median         mean           sd 
223.00000000  70.00000000   0.03871202   0.05536429   0.05660577 
\end{verbatim}

\subsection{Imputation}\label{imputation}

First, lets compare the estimated mean and sd of each method. We can use
the \texttt{censtats} function to achieve this:

\begin{Shaded}
\begin{Highlighting}[]
\FunctionTok{censtats}\NormalTok{(Soar}\SpecialCharTok{$}\NormalTok{Ammonia, Soar}\SpecialCharTok{$}\NormalTok{censored)}
\end{Highlighting}
\end{Shaded}

\begin{verbatim}
        n     n.cen   pct.cen 
223.00000  70.00000  31.39013 
\end{verbatim}

\begin{verbatim}
        median       mean         sd
K-M 0.03600000 0.05590147 0.04967880
ROS 0.03600000 0.05577297 0.04960383
MLE 0.03871202 0.05536429 0.05660577
\end{verbatim}

Now, lets create a function that draws a value between 0 and a given
\emph{LoD} based on a \(\mathrm{Normal}(\mu,\sigma)\) density.

\begin{Shaded}
\begin{Highlighting}[]
\NormalTok{fx\_lod }\OtherTok{=} \ControlFlowTok{function}\NormalTok{(lod,mean,sd) \{}
    \ControlFlowTok{repeat}\NormalTok{ \{}
\NormalTok{      x }\OtherTok{\textless{}{-}} \FunctionTok{rnorm}\NormalTok{(}\DecValTok{1}\NormalTok{, mean, sd) }\CommentTok{\# generate a value from N(mu,sigma)}
      \ControlFlowTok{if}\NormalTok{ (x }\SpecialCharTok{\textgreater{}=} \DecValTok{0} \SpecialCharTok{\&\&}\NormalTok{ x }\SpecialCharTok{\textless{}=}\NormalTok{ lod) }\CommentTok{\# repeat unless the generated value is \textgreater{}=0 and \textless{}LoD}
        \FunctionTok{return}\NormalTok{(x)   }
\NormalTok{    \}}
\NormalTok{\}}
\end{Highlighting}
\end{Shaded}

For example taking the mean and sd from the ROS output and a
\(LoD = 0.3\) we have

\begin{Shaded}
\begin{Highlighting}[]
\FunctionTok{fx\_lod}\NormalTok{(}\FloatTok{0.3}\NormalTok{,}\FunctionTok{mean}\NormalTok{(ROS),}\FunctionTok{sd}\NormalTok{(ROS))}
\end{Highlighting}
\end{Shaded}

\begin{verbatim}
[1] 0.03568448
\end{verbatim}

Now we can input the censored values by applying the custom-built
\texttt{fx\_lod()} function using the estimated mean and variance from
the ROS as follows:

\begin{Shaded}
\begin{Highlighting}[]
\NormalTok{Soar}\SpecialCharTok{$}\NormalTok{imputed.ROS }\OtherTok{\textless{}{-}} \FunctionTok{ifelse}\NormalTok{(}
\NormalTok{  Soar}\SpecialCharTok{$}\NormalTok{censored }\SpecialCharTok{==}\NormalTok{ F,}
\NormalTok{  Soar}\SpecialCharTok{$}\NormalTok{Ammonia,  }\CommentTok{\# Keep original if not censored }
  \CommentTok{\# otherwise apply the fx\_lod function for each censored observation}
  \FunctionTok{sapply}\NormalTok{(Soar}\SpecialCharTok{$}\NormalTok{Ammonia[Soar}\SpecialCharTok{$}\NormalTok{censored], fx\_lod, }\AttributeTok{mean =} \FunctionTok{mean}\NormalTok{(ROS), }\AttributeTok{sd =} \FunctionTok{sd}\NormalTok{(ROS))}
\NormalTok{)}
\end{Highlighting}
\end{Shaded}

Lets visualize our results. We can plot the original and imputed Ammonia
values against the day of the year as follows:

\begin{Shaded}
\begin{Highlighting}[]
\FunctionTok{ggplot}\NormalTok{(}\AttributeTok{data=}\NormalTok{Soar,}\FunctionTok{aes}\NormalTok{(}\AttributeTok{y=}\NormalTok{Ammonia,}\AttributeTok{x=}\NormalTok{doy,}\AttributeTok{color=}\NormalTok{censored))}\SpecialCharTok{+}
  \FunctionTok{geom\_point}\NormalTok{() }\SpecialCharTok{+}
  \FunctionTok{scale\_color\_discrete}\NormalTok{(}\AttributeTok{name=}\StringTok{"Censored"}\NormalTok{)}\SpecialCharTok{+}
\FunctionTok{ggplot}\NormalTok{(}\AttributeTok{data=}\NormalTok{Soar,}\FunctionTok{aes}\NormalTok{(}\AttributeTok{y=}\NormalTok{imputed.ROS,}\AttributeTok{x=}\NormalTok{doy,}\AttributeTok{color=}\NormalTok{censored))}\SpecialCharTok{+}
  \FunctionTok{geom\_point}\NormalTok{() }\SpecialCharTok{+} \FunctionTok{scale\_color\_discrete}\NormalTok{(}\AttributeTok{name=}\StringTok{"Imputed"}\NormalTok{)}
\end{Highlighting}
\end{Shaded}

\begin{center}
\pandocbounded{\includegraphics[keepaspectratio]{lab_1_files/figure-pdf/unnamed-chunk-17-1.pdf}}
\end{center}

Instead, we can create a decimal year or fractional year timestamp by
combining the year with the day of year:

\begin{Shaded}
\begin{Highlighting}[]
\NormalTok{Soar}\SpecialCharTok{$}\NormalTok{year.day }\OtherTok{\textless{}{-}}\NormalTok{ Soar}\SpecialCharTok{$}\NormalTok{year }\SpecialCharTok{+}\NormalTok{ Soar}\SpecialCharTok{$}\NormalTok{doy }\SpecialCharTok{/} \DecValTok{366}

\FunctionTok{ggplot}\NormalTok{(}\AttributeTok{data=}\NormalTok{Soar,}\FunctionTok{aes}\NormalTok{(}\AttributeTok{y=}\NormalTok{Ammonia,}\AttributeTok{x=}\NormalTok{year.day,}\AttributeTok{color=}\NormalTok{censored))}\SpecialCharTok{+}
  \FunctionTok{geom\_point}\NormalTok{() }\SpecialCharTok{+}
  \FunctionTok{scale\_color\_discrete}\NormalTok{(}\AttributeTok{name=}\StringTok{"Censored"}\NormalTok{)}\SpecialCharTok{+}
\FunctionTok{ggplot}\NormalTok{(}\AttributeTok{data=}\NormalTok{Soar,}\FunctionTok{aes}\NormalTok{(}\AttributeTok{y=}\NormalTok{imputed.ROS,}\AttributeTok{x=}\NormalTok{year.day,}\AttributeTok{color=}\NormalTok{censored))}\SpecialCharTok{+}
  \FunctionTok{geom\_point}\NormalTok{() }\SpecialCharTok{+} \FunctionTok{scale\_color\_discrete}\NormalTok{(}\AttributeTok{name=}\StringTok{"Imputed"}\NormalTok{)}
\end{Highlighting}
\end{Shaded}

\begin{center}
\pandocbounded{\includegraphics[keepaspectratio]{lab_1_files/figure-pdf/unnamed-chunk-18-1.pdf}}
\end{center}

\begin{tcolorbox}[enhanced jigsaw, coltitle=black, colback=white, titlerule=0mm, colframe=quarto-callout-warning-color-frame, left=2mm, opacityback=0, rightrule=.15mm, bottomtitle=1mm, bottomrule=.15mm, leftrule=.75mm, title={Task 2}, arc=.35mm, colbacktitle=quarto-callout-warning-color!10!white, toptitle=1mm, toprule=.15mm, breakable, opacitybacktitle=0.6]

\begin{enumerate}
\def\labelenumi{\arabic{enumi}.}
\item
  Add to other columns to the \texttt{Soar} data set names
  \texttt{Soar\$imputed.KM} and \texttt{Soar\$imputed.MLE} containing
  the imputed values for KM and MLE respectively.
\item
  Create three plots that compare the imputed ammonia values against the
  decimal year for each method. Discuss how changing the approach taken
  affects the imputed values.
\end{enumerate}

Click here to see the solution

\begin{Shaded}
\begin{Highlighting}[]
\NormalTok{Soar}\SpecialCharTok{$}\NormalTok{imputed.KM }\OtherTok{\textless{}{-}} \FunctionTok{ifelse}\NormalTok{(}
\NormalTok{  Soar}\SpecialCharTok{$}\NormalTok{censored }\SpecialCharTok{==}\NormalTok{ F,}
\NormalTok{  Soar}\SpecialCharTok{$}\NormalTok{Ammonia,  }\CommentTok{\# Keep original if not censored }
  \CommentTok{\# otherwise apply the fx\_lod function for each censored observation}
  \FunctionTok{sapply}\NormalTok{(Soar}\SpecialCharTok{$}\NormalTok{Ammonia[Soar}\SpecialCharTok{$}\NormalTok{censored], fx\_lod, }\AttributeTok{mean =} \FunctionTok{mean}\NormalTok{(KM), }\AttributeTok{sd =} \FunctionTok{sd}\NormalTok{(KM))}
\NormalTok{)}

\NormalTok{Soar}\SpecialCharTok{$}\NormalTok{imputed.MLE }\OtherTok{\textless{}{-}} \FunctionTok{ifelse}\NormalTok{(}
\NormalTok{  Soar}\SpecialCharTok{$}\NormalTok{censored }\SpecialCharTok{==}\NormalTok{ F,}
\NormalTok{  Soar}\SpecialCharTok{$}\NormalTok{Ammonia,  }\CommentTok{\# Keep original if not censored }
  \CommentTok{\# otherwise apply the fx\_lod function for each censored observation}
  \FunctionTok{sapply}\NormalTok{(Soar}\SpecialCharTok{$}\NormalTok{Ammonia[Soar}\SpecialCharTok{$}\NormalTok{censored], fx\_lod, }\AttributeTok{mean =} \FunctionTok{mean}\NormalTok{(MLE), }\AttributeTok{sd =} \FunctionTok{sd}\NormalTok{(MLE))}
\NormalTok{)}

\FunctionTok{ggplot}\NormalTok{(}\AttributeTok{data=}\NormalTok{Soar,}\FunctionTok{aes}\NormalTok{(}\AttributeTok{y=}\NormalTok{imputed.ROS,}\AttributeTok{x=}\NormalTok{year.day,}\AttributeTok{color=}\NormalTok{censored))}\SpecialCharTok{+}
  \FunctionTok{geom\_point}\NormalTok{() }\SpecialCharTok{+} 
  \FunctionTok{scale\_color\_discrete}\NormalTok{(}\AttributeTok{name=}\StringTok{"Imputed"}\NormalTok{)}\SpecialCharTok{+}
  \FunctionTok{ggtitle}\NormalTok{(}\StringTok{"ROS imputation"}\NormalTok{) }\SpecialCharTok{+}
\FunctionTok{ggplot}\NormalTok{(}\AttributeTok{data=}\NormalTok{Soar,}\FunctionTok{aes}\NormalTok{(}\AttributeTok{y=}\NormalTok{imputed.KM,}\AttributeTok{x=}\NormalTok{year.day,}\AttributeTok{color=}\NormalTok{censored))}\SpecialCharTok{+}
  \FunctionTok{geom\_point}\NormalTok{() }\SpecialCharTok{+} 
  \FunctionTok{scale\_color\_discrete}\NormalTok{(}\AttributeTok{name=}\StringTok{"Imputed"}\NormalTok{)}\SpecialCharTok{+}
  \FunctionTok{ggtitle}\NormalTok{(}\StringTok{"KM imputation"}\NormalTok{) }\SpecialCharTok{+}
\FunctionTok{ggplot}\NormalTok{(}\AttributeTok{data=}\NormalTok{Soar,}\FunctionTok{aes}\NormalTok{(}\AttributeTok{y=}\NormalTok{imputed.MLE,}\AttributeTok{x=}\NormalTok{year.day,}\AttributeTok{color=}\NormalTok{censored))}\SpecialCharTok{+}
  \FunctionTok{geom\_point}\NormalTok{() }\SpecialCharTok{+} 
  \FunctionTok{scale\_color\_discrete}\NormalTok{(}\AttributeTok{name=}\StringTok{"Imputed"}\NormalTok{)}\SpecialCharTok{+}
  \FunctionTok{ggtitle}\NormalTok{(}\StringTok{"MLE imputation"}\NormalTok{)}
\end{Highlighting}
\end{Shaded}

\pandocbounded{\includegraphics[keepaspectratio]{lab_1_files/figure-pdf/unnamed-chunk-19-1.pdf}}

\end{tcolorbox}

\section{Part 2: Designing a monitoring network}\label{sec-design}

Spatial sampling design is critical for ensuring monitoring data are
representative and cost-effective. This practical introduces
\textbf{generalized random-tessellation stratified (GRTS) sampling}
using the \texttt{spsurvey} package in R. GRTS provides spatially
balanced samples---avoiding clustering while ensuring geographic
coverage---and is widely used in environmental monitoring programs. We
will design a monitoring network for lakes in the Northeastern US. The
\texttt{NE\_Lakes} dataset contains the spatial information of 195 lakes
in the Northeastern United States which are summarised in the following
table:

\begin{longtable}[]{@{}
  >{\raggedright\arraybackslash}p{(\linewidth - 2\tabcolsep) * \real{0.5000}}
  >{\raggedright\arraybackslash}p{(\linewidth - 2\tabcolsep) * \real{0.5000}}@{}}
\toprule\noalign{}
\begin{minipage}[b]{\linewidth}\raggedright
Variable
\end{minipage} & \begin{minipage}[b]{\linewidth}\raggedright
Meaning
\end{minipage} \\
\midrule\noalign{}
\endhead
\bottomrule\noalign{}
\endlastfoot
AREA & Lake area in hectares. \\
AREA\_CAT & Lake area categories based on a hectare cutoff. \\
ELEV & Elevation in meters. \\
ELEV\_CAT & Elevation categories based on a meter cutoff. \\
geometry & POINT geometry using the NAD83 / Conus Albers coordinate
reference system (EPSG: 5070) \\
\end{longtable}

We begin by first loading the data set contained within the
\texttt{spsurvey} package, additionally we will load the
\texttt{mapview} and \texttt{sf} packages for visualizing our data.

\begin{Shaded}
\begin{Highlighting}[]
\FunctionTok{library}\NormalTok{(spsurvey)}
\FunctionTok{library}\NormalTok{(mapview)}
\FunctionTok{data}\NormalTok{(}\StringTok{"NE\_Lakes"}\NormalTok{)}
\end{Highlighting}
\end{Shaded}

The \texttt{NE\_Lakes} data is a Simple Features (\texttt{sf}) object
containing the spatial information for the point-location of 195 lakes
in the Northeastern United States. We can use the \texttt{mapview}
function to visualise the distribution of lakes and colored them by
their size as follow:

\begin{Shaded}
\begin{Highlighting}[]
\FunctionTok{mapview}\NormalTok{(NE\_Lakes,}\AttributeTok{zcol=}\StringTok{"AREA\_CAT"}\NormalTok{)}
\end{Highlighting}
\end{Shaded}

\begin{verbatim}
file:///C:/Users/admin/AppData/Local/Temp/RtmpA7svHa/file36d433ee7b64/widget36d41c5e294b.html screenshot completed
\end{verbatim}

\pandocbounded{\includegraphics[keepaspectratio]{lab_1_files/figure-pdf/unnamed-chunk-21-1.pdf}}

\begin{tcolorbox}[enhanced jigsaw, coltitle=black, colback=white, titlerule=0mm, colframe=quarto-callout-warning-color-frame, left=2mm, opacityback=0, rightrule=.15mm, bottomtitle=1mm, bottomrule=.15mm, leftrule=.75mm, title={Task 3}, arc=.35mm, colbacktitle=quarto-callout-warning-color!10!white, toptitle=1mm, toprule=.15mm, breakable, opacitybacktitle=0.6]

In the previous plot we have shown the spatial distribution of lakes
colored by their size. Create a map that shows the lake distribution
colored by low and high elevation levels. Do you see any patterns?

Click here to see the solution

\begin{Shaded}
\begin{Highlighting}[]
\FunctionTok{mapview}\NormalTok{(NE\_Lakes,}\AttributeTok{zcol=}\StringTok{"ELEV\_CAT"}\NormalTok{)}
\end{Highlighting}
\end{Shaded}

\pandocbounded{\includegraphics[keepaspectratio]{lab_1_files/figure-pdf/unnamed-chunk-22-1.pdf}}

\end{tcolorbox}

\subsection{Independent Random
Sampling}\label{independent-random-sampling}

Suppose we want to select a random sample of 50 lakes using independent
random sampling without replacement. To do so we can use the
\texttt{irs} function from \texttt{spsurvey} to draw a random sample of
size 50 with equal probabilities as follows:

\begin{Shaded}
\begin{Highlighting}[]
\NormalTok{eqprob\_irs }\OtherTok{\textless{}{-}} \FunctionTok{irs}\NormalTok{(NE\_Lakes, }\AttributeTok{n\_base =} \DecValTok{50}\NormalTok{)}
\end{Highlighting}
\end{Shaded}

To visualize the selected sites (lakes) we can use the \texttt{sf}
library and \texttt{ggplot} packages:

\begin{Shaded}
\begin{Highlighting}[]
\FunctionTok{ggplot}\NormalTok{() }\SpecialCharTok{+}
  \FunctionTok{geom\_sf}\NormalTok{(}\AttributeTok{data=}\NormalTok{NE\_Lakes,}\FunctionTok{aes}\NormalTok{(}\AttributeTok{color=}\StringTok{"Not Selected"}\NormalTok{)) }\SpecialCharTok{+}
  \FunctionTok{geom\_sf}\NormalTok{(}\AttributeTok{data=}\NormalTok{eqprob\_irs}\SpecialCharTok{$}\NormalTok{sites\_base,}\FunctionTok{aes}\NormalTok{(}\AttributeTok{color=}\StringTok{"Selected"}\NormalTok{))}
\end{Highlighting}
\end{Shaded}

\begin{center}
\pandocbounded{\includegraphics[keepaspectratio]{lab_1_files/figure-pdf/unnamed-chunk-24-1.pdf}}
\end{center}

\subsection{GRTS with equal inclusion
probabilities}\label{grts-with-equal-inclusion-probabilities}

Now, we will implement the GRTS algorithm using the \texttt{grts()}
function to select a spatially balanced sample of size 50 where each
lake has an equal inclusion probability,

\begin{Shaded}
\begin{Highlighting}[]
\NormalTok{eqprob }\OtherTok{\textless{}{-}} \FunctionTok{grts}\NormalTok{(NE\_Lakes, }\AttributeTok{n\_base =} \DecValTok{50}\NormalTok{)}
\end{Highlighting}
\end{Shaded}

You can either, \texttt{mapview} (interactive map), \texttt{ggplot()}
(static map) or the R base \texttt{plot()} functions to visualize the
selected site for monitoring.

\subsection{\texorpdfstring{Base R \texttt{plot()}}{Base R plot()}}

\begin{Shaded}
\begin{Highlighting}[]
\FunctionTok{plot}\NormalTok{(eqprob)}
\end{Highlighting}
\end{Shaded}

\pandocbounded{\includegraphics[keepaspectratio]{lab_1_files/figure-pdf/unnamed-chunk-26-1.pdf}}

\subsection{\texorpdfstring{\texttt{ggplot} and
\texttt{sf}}{ggplot and sf}}

\begin{Shaded}
\begin{Highlighting}[]
\FunctionTok{ggplot}\NormalTok{() }\SpecialCharTok{+}
  \FunctionTok{geom\_sf}\NormalTok{(}\AttributeTok{data=}\NormalTok{eqprob}\SpecialCharTok{$}\NormalTok{sites\_base)}
\end{Highlighting}
\end{Shaded}

\pandocbounded{\includegraphics[keepaspectratio]{lab_1_files/figure-pdf/unnamed-chunk-27-1.pdf}}

\subsection{\texorpdfstring{\texttt{mapview}}{mapview}}

\begin{Shaded}
\begin{Highlighting}[]
\FunctionTok{mapview}\NormalTok{(eqprob}\SpecialCharTok{$}\NormalTok{sites\_base)}
\end{Highlighting}
\end{Shaded}

\begin{verbatim}
file:///C:/Users/admin/AppData/Local/Temp/RtmpA7svHa/file36d42bb93b7a/widget36d474802b7f.html screenshot completed
\end{verbatim}

\pandocbounded{\includegraphics[keepaspectratio]{lab_1_files/figure-pdf/unnamed-chunk-28-1.pdf}}

\subsection{GRTS with statified
sampling}\label{grts-with-statified-sampling}

Instead of sampling from the entire sampling area simultaneously, we can
apply the GRTS algorithm for a given strata and select samples from each
stratum independently of other strata.

In this example we will obtain a GRTS sample stratified by the lake
elevation categories where all lake within a stratum have equal
inclusion probabilities:

\begin{Shaded}
\begin{Highlighting}[]
\NormalTok{n\_strata }\OtherTok{\textless{}{-}} \FunctionTok{c}\NormalTok{(}\AttributeTok{low =} \DecValTok{35}\NormalTok{, }\AttributeTok{high =} \DecValTok{15}\NormalTok{)}

\NormalTok{eqprob\_strat }\OtherTok{\textless{}{-}} \FunctionTok{grts}\NormalTok{(NE\_Lakes, }\AttributeTok{n\_base =}\NormalTok{ n\_strata,}
                     \AttributeTok{stratum\_var =} \StringTok{"ELEV\_CAT"}\NormalTok{)}
\end{Highlighting}
\end{Shaded}

Here,\texttt{n\_strata} specifies the stratum-specific sample sizes (35
for low elevation category and 15 for the high elevation category).
Notice that the names in \texttt{n\_strata} (low and high) need to match
the names of the stratification variable (\texttt{"ELEV\_CAT"}) in the
\texttt{NE\_Lakes} data. The the \texttt{grts} function receives as
arguments the \texttt{n\_strata} vector and name of the column in the
data that represents the stratification variable via the
\texttt{stratum\_var} argument.

\begin{Shaded}
\begin{Highlighting}[]
\FunctionTok{mapview}\NormalTok{(eqprob}\SpecialCharTok{$}\NormalTok{sites\_base,}
        \AttributeTok{map.types =} \FunctionTok{c}\NormalTok{(}\StringTok{"Esri.WorldShadedRelief"}\NormalTok{),}
        \AttributeTok{col.regions =} \StringTok{"tomato"}\NormalTok{,}
        \AttributeTok{layer.name=}\StringTok{"GTRS sampling"}\NormalTok{)}\SpecialCharTok{+}
\FunctionTok{mapview}\NormalTok{(eqprob\_strat}\SpecialCharTok{$}\NormalTok{sites\_base,}
        \AttributeTok{map.types =} \FunctionTok{c}\NormalTok{(}\StringTok{"Esri.WorldShadedRelief"}\NormalTok{),}
        \AttributeTok{col.regions =} \StringTok{"purple"}\NormalTok{,}
        \AttributeTok{layer.name=}\StringTok{"Stratified GTRS sampling"}\NormalTok{)}
\end{Highlighting}
\end{Shaded}

\begin{verbatim}
file:///C:/Users/admin/AppData/Local/Temp/RtmpA7svHa/file36d4745a96/widget36d41e657ac9.html screenshot completed
\end{verbatim}

\pandocbounded{\includegraphics[keepaspectratio]{lab_1_files/figure-pdf/unnamed-chunk-30-1.pdf}}

\subsection{GRTS with unequal inclusion
probabilities}\label{grts-with-unequal-inclusion-probabilities}

Sometimes we don't want inclusion probabilities to be equal for all
sites. For example, we may want larger lakes to be sampled more
frequently than smaller lakes based on attributes like surface area.

The \texttt{caty\_n} and \texttt{caty\_var} arguments in the
\texttt{gtrs} functions allows us to select a GRTS sample with unequal
inclusion probabilities according to a particular category e.g., lake
area.

\begin{Shaded}
\begin{Highlighting}[]
\NormalTok{caty\_n }\OtherTok{\textless{}{-}} \FunctionTok{c}\NormalTok{(}\AttributeTok{small =} \DecValTok{10}\NormalTok{, }\AttributeTok{large =} \DecValTok{40}\NormalTok{)}
\NormalTok{uneqprob }\OtherTok{\textless{}{-}} \FunctionTok{grts}\NormalTok{(NE\_Lakes, }\AttributeTok{n\_base =} \DecValTok{50}\NormalTok{, }\AttributeTok{caty\_n =}\NormalTok{ caty\_n, }\AttributeTok{caty\_var =} \StringTok{"AREA\_CAT"}\NormalTok{)}
\end{Highlighting}
\end{Shaded}

The \texttt{cat\_n} vector specifies the within-level sample sizes. This
gets passed on to \texttt{grts} via the \texttt{caty\_n} argument. The
names in \texttt{caty\_n} must match the different levels of categorical
variable which in the data (specified via the \texttt{caty\_var}
argument).

The map below shows a sample size of 40 for large lakes and sample size
of 10 small lakes.

\begin{Shaded}
\begin{Highlighting}[]
\FunctionTok{mapview}\NormalTok{(uneqprob}\SpecialCharTok{$}\NormalTok{sites\_base,}
        \AttributeTok{zcol=}\StringTok{"AREA\_CAT"}\NormalTok{,}
        \AttributeTok{map.types =} \FunctionTok{c}\NormalTok{(}\StringTok{"Esri.WorldShadedRelief"}\NormalTok{),}
        \AttributeTok{layer.name=}\StringTok{"GTRS sampling with unequal inclusion probabilities"}\NormalTok{)}
\end{Highlighting}
\end{Shaded}

\begin{verbatim}
file:///C:/Users/admin/AppData/Local/Temp/RtmpA7svHa/file36d412ee5f27/widget36d471ac1efe.html screenshot completed
\end{verbatim}

\pandocbounded{\includegraphics[keepaspectratio]{lab_1_files/figure-pdf/unnamed-chunk-32-1.pdf}}

We can also implement probability proportional to size (PPS) sampling,
where each lake's selection probability is directly proportional to its
surface area. This avoids arbitrary size categories and ensures larger
lakes---which often have greater ecological and socioeconomic
importance---are more likely to be selected. We can conduct PPS sampling
by specifying the lake's areas as an \texttt{aux\_var} variable in the
\texttt{grts()} function:

\begin{Shaded}
\begin{Highlighting}[]
\NormalTok{propprob }\OtherTok{\textless{}{-}} \FunctionTok{grts}\NormalTok{(NE\_Lakes, }\AttributeTok{n\_base =} \DecValTok{50}\NormalTok{, }\AttributeTok{aux\_var =} \StringTok{"AREA"}\NormalTok{)}
\end{Highlighting}
\end{Shaded}

\begin{Shaded}
\begin{Highlighting}[]
\FunctionTok{mapview}\NormalTok{(propprob}\SpecialCharTok{$}\NormalTok{sites\_base,}
                \AttributeTok{zcol=}\StringTok{"AREA"}\NormalTok{,}
        \AttributeTok{map.types =} \FunctionTok{c}\NormalTok{(}\StringTok{"Esri.WorldShadedRelief"}\NormalTok{),}
        \AttributeTok{layer.name=}\StringTok{"GTRS sampling with PPS sampling"}\NormalTok{)}
\end{Highlighting}
\end{Shaded}

\begin{verbatim}
file:///C:/Users/admin/AppData/Local/Temp/RtmpA7svHa/file36d4570d6bdd/widget36d415585796.html screenshot completed
\end{verbatim}

\pandocbounded{\includegraphics[keepaspectratio]{lab_1_files/figure-pdf/unnamed-chunk-34-1.pdf}}

\subsection{\texorpdfstring{Additional features in
\texttt{spsurvey}}{Additional features in spsurvey}}\label{additional-features-in-spsurvey}

Other useful features that have been implemented on the \texttt{grts}
function are (see Dumelle et al. (2023) for a more comprehensive
description of the package):

\begin{itemize}
\item
  \emph{Legacy sites} - allows sites selected from a previous sampling
  scheme to be selected in a new sample
  (\texttt{grts(NE\_Lakes,\ n\_base\ =\ 50,legacy\_sites\ =\ NE\_Lakes\_Legacy)}).
  This is often used to study or monitor the behavior of the sites in
  the network through time.
\item
  \emph{Minimum distance selection} -Sometimes the selected sites are
  too close to each other. We can set a minimum distance between sites
  by setting the \texttt{mindis} argument to a particular distance
  determined by the data CRS (e.g.,
  \texttt{grts(NE\_Lakes,\ n\_base\ =\ 50,\ mindis\ =\ 1600)})
\item
  \emph{Replacement sites} - it is common that once a network has been
  designed the data at some of the selected in the sample not able to
  been able to collected at the site (e.g, due terrain contraints or
  landowner permission). We can then use a nearest neighbor approach to
  selects replacement sites according to the distance between
  GRTS-sampled site and all other sites in the sampling frame that are
  not part of the GRTS sample. E.g., to select a GRTS sample of size 50
  with two nearest neighbor replacement we can run
  \texttt{eqprob\_nn\ \textless{}-\ grts(NE\_Lakes,\ n\_base\ =\ 50,\ n\_near\ =\ 2)}.
\end{itemize}

\subsection{Assessing spatial balance}\label{assessing-spatial-balance}

A practical way to measure spatial balance was developed by Stevens and
Olsen (2004) using Voronoi polygons. In this approach, each sampled site
defines a region containing all locations closer to it than to any other
sampled site. For a spatially balanced design, the total inclusion
probability of all sites within each Voronoi polygon is expected to be
1. Deviation from this ideal can be quantified using a loss metric based
on these polygon totals. One common choice is Pielou's evenness index
(PEI), which assesses how uniformly the inclusion probability is
distributed across the sample sites. Pielou's evenness index (PEI) is
defined as:

\[
\text{PEI} = 1 + \frac{\sum_{i=1}^{n} \frac{v_i}{n} \ln(v_i/n)}{\ln(n)},
\]

where \(n\) is the sample size. PEI is bounded between zero and one. A
PEI of zero indicates perfect spatial balance. As PEI increases, the
spatial balance worsens. The \texttt{sp\_balance()} function which
receives as arguments the (i) design sites and (ii) the sampling frame (
note that if stratified sampling is being compared you also need to
supply the name of the stratified variable in your data via the
\texttt{stratum\_var} arugment).

The following code compares the GRTS with equal inclusion probabilities
again the SRS:

\begin{Shaded}
\begin{Highlighting}[]
\FunctionTok{sp\_balance}\NormalTok{(eqprob}\SpecialCharTok{$}\NormalTok{sites\_base, NE\_Lakes)}
\end{Highlighting}
\end{Shaded}

\begin{verbatim}
  stratum metric      value
1    None pielou 0.03456649
\end{verbatim}

\begin{Shaded}
\begin{Highlighting}[]
\FunctionTok{sp\_balance}\NormalTok{(eqprob\_irs}\SpecialCharTok{$}\NormalTok{sites\_base, NE\_Lakes)}
\end{Highlighting}
\end{Shaded}

\begin{verbatim}
  stratum metric      value
1    None pielou 0.03934743
\end{verbatim}

\begin{tcolorbox}[enhanced jigsaw, coltitle=black, colback=white, titlerule=0mm, colframe=quarto-callout-tip-color-frame, left=2mm, opacityback=0, rightrule=.15mm, bottomtitle=1mm, bottomrule=.15mm, leftrule=.75mm, title={Question}, arc=.35mm, colbacktitle=quarto-callout-tip-color!10!white, toptitle=1mm, toprule=.15mm, breakable, opacitybacktitle=0.6]

Which design has better spatial balance?

\begin{itemize}
\tightlist
\item
  \begin{enumerate}
  \def\labelenumi{(\Alph{enumi})}
  \tightlist
  \item
    IRS\\
  \end{enumerate}
\item
  \begin{enumerate}
  \def\labelenumi{(\Alph{enumi})}
  \setcounter{enumi}{1}
  \tightlist
  \item
    GRTS
  \end{enumerate}
\end{itemize}

\end{tcolorbox}

So far we have applied the GRTS algorithm to point reference data.
However, this methodology can also be applied on linear and areal data
in similar fashion -- the only difference being the geometry type of the
\texttt{sf} object used as argument. In the next exercise, you will be
tasked with designing a river network using the GRTS algorithm for a
section of the Illinois River in Arkansas and Oklahoma

\begin{tcolorbox}[enhanced jigsaw, coltitle=black, colback=white, titlerule=0mm, colframe=quarto-callout-warning-color-frame, left=2mm, opacityback=0, rightrule=.15mm, bottomtitle=1mm, bottomrule=.15mm, leftrule=.75mm, title={Task 4}, arc=.35mm, colbacktitle=quarto-callout-warning-color!10!white, toptitle=1mm, toprule=.15mm, breakable, opacitybacktitle=0.6]

The \texttt{Illinois\_River} data in \texttt{spsurvey} contains the
spatial information of 244 segments of the Illinois River in Arkansas
and Oklahoma. The data can be accessed with
\texttt{data(Illinois\_River)}. Use the \texttt{grts} to:

\begin{enumerate}
\def\labelenumi{\arabic{enumi}.}
\item
  Design a monitoring network with \(n=25\) sampling points using
  \emph{independent random sampling}.
\item
  Design a monitoring network with \(n=25\) sampling points using
  \emph{GRTS sampling} with equal inclusion probabilities.
\item
  Plot both sampling designs using \texttt{ggplot2}, coloring the
  selected sites according to the sampling method.
\end{enumerate}

Finally, compare the spatial balance of the two approaches. Which method
provides better spatial coverage across the river network?

Take hint

You can add multiple \texttt{geom\_sf()} layers to a ggplot object,
e.g.,

\begin{Shaded}
\begin{Highlighting}[]
\FunctionTok{ggplot}\NormalTok{()}\SpecialCharTok{+}
\FunctionTok{geom\_sf}\NormalTok{(}\AttributeTok{data=}\NormalTok{layer\_1)}\SpecialCharTok{+}
\FunctionTok{geom\_sf}\NormalTok{(}\AttributeTok{data=}\NormalTok{leayer\_2) }\SpecialCharTok{+}\NormalTok{...}
\end{Highlighting}
\end{Shaded}

Click here to see the solution

\begin{Shaded}
\begin{Highlighting}[]
\FunctionTok{data}\NormalTok{(Illinois\_River)}

\NormalTok{irs\_linear }\OtherTok{\textless{}{-}} \FunctionTok{irs}\NormalTok{(Illinois\_River, }\AttributeTok{n\_base =} \DecValTok{25}\NormalTok{)}
\NormalTok{eqprob\_linear }\OtherTok{\textless{}{-}} \FunctionTok{grts}\NormalTok{(Illinois\_River, }\AttributeTok{n\_base =} \DecValTok{25}\NormalTok{)}

\FunctionTok{ggplot}\NormalTok{()}\SpecialCharTok{+}
  \FunctionTok{geom\_sf}\NormalTok{(}\AttributeTok{data=}\NormalTok{Illinois\_River,}\AttributeTok{color=}\StringTok{"gray40"}\NormalTok{)}\SpecialCharTok{+}
  \FunctionTok{geom\_sf}\NormalTok{(}\AttributeTok{data=}\NormalTok{eqprob\_linear}\SpecialCharTok{$}\NormalTok{sites\_base,}\FunctionTok{aes}\NormalTok{(}\AttributeTok{color=}\StringTok{"GRTS sampled sites"}\NormalTok{))}\SpecialCharTok{+}
  \FunctionTok{geom\_sf}\NormalTok{(}\AttributeTok{data=}\NormalTok{irs\_linear}\SpecialCharTok{$}\NormalTok{sites\_base,}\FunctionTok{aes}\NormalTok{(}\AttributeTok{color=}\StringTok{"IRS sampled sites"}\NormalTok{))}
\end{Highlighting}
\end{Shaded}

\pandocbounded{\includegraphics[keepaspectratio]{lab_1_files/figure-pdf/unnamed-chunk-36-1.pdf}}

\begin{Shaded}
\begin{Highlighting}[]
\FunctionTok{sp\_balance}\NormalTok{(irs\_linear}\SpecialCharTok{$}\NormalTok{sites\_base, Illinois\_River)}
\end{Highlighting}
\end{Shaded}

\begin{verbatim}
  stratum metric     value
1    None pielou 0.0535789
\end{verbatim}

\begin{Shaded}
\begin{Highlighting}[]
\FunctionTok{sp\_balance}\NormalTok{(eqprob\_linear}\SpecialCharTok{$}\NormalTok{sites\_base, Illinois\_River)}
\end{Highlighting}
\end{Shaded}

\begin{verbatim}
  stratum metric      value
1    None pielou 0.04971315
\end{verbatim}

\begin{Shaded}
\begin{Highlighting}[]
\CommentTok{\# GRTS PEI is smaller and thus provided a better spatially balanced design}
\end{Highlighting}
\end{Shaded}

\end{tcolorbox}

\phantomsection\label{refs}
\begin{CSLReferences}{1}{0}
\bibitem[\citeproctext]{ref-dumelle2023}
Dumelle, Michael, Tom Kincaid, Anthony R. Olsen, and Marc Weber. 2023.
{``{\textbf{Spsurvey}}: Spatial Sampling Design and Analysis in
{\emph{R}}.''} \emph{Journal of Statistical Software} 105 (3).
\url{https://doi.org/10.18637/jss.v105.i03}.

\bibitem[\citeproctext]{ref-stevens2004}
Stevens, Don L, and Anthony R Olsen. 2004. {``Spatially Balanced
Sampling of Natural Resources.''} \emph{Journal of the American
Statistical Association} 99 (465): 262--78.
\url{https://doi.org/10.1198/016214504000000250}.

\end{CSLReferences}




\end{document}
