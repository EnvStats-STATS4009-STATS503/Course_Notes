% Options for packages loaded elsewhere
\PassOptionsToPackage{unicode}{hyperref}
\PassOptionsToPackage{hyphens}{url}
\PassOptionsToPackage{dvipsnames,svgnames,x11names}{xcolor}
%
\documentclass[
  letterpaper,
  DIV=11,
  numbers=noendperiod]{scrartcl}

\usepackage{amsmath,amssymb}
\usepackage{iftex}
\ifPDFTeX
  \usepackage[T1]{fontenc}
  \usepackage[utf8]{inputenc}
  \usepackage{textcomp} % provide euro and other symbols
\else % if luatex or xetex
  \usepackage{unicode-math}
  \defaultfontfeatures{Scale=MatchLowercase}
  \defaultfontfeatures[\rmfamily]{Ligatures=TeX,Scale=1}
\fi
\usepackage{lmodern}
\ifPDFTeX\else  
    % xetex/luatex font selection
\fi
% Use upquote if available, for straight quotes in verbatim environments
\IfFileExists{upquote.sty}{\usepackage{upquote}}{}
\IfFileExists{microtype.sty}{% use microtype if available
  \usepackage[]{microtype}
  \UseMicrotypeSet[protrusion]{basicmath} % disable protrusion for tt fonts
}{}
\makeatletter
\@ifundefined{KOMAClassName}{% if non-KOMA class
  \IfFileExists{parskip.sty}{%
    \usepackage{parskip}
  }{% else
    \setlength{\parindent}{0pt}
    \setlength{\parskip}{6pt plus 2pt minus 1pt}}
}{% if KOMA class
  \KOMAoptions{parskip=half}}
\makeatother
\usepackage{xcolor}
\setlength{\emergencystretch}{3em} % prevent overfull lines
\setcounter{secnumdepth}{5}
% Make \paragraph and \subparagraph free-standing
\makeatletter
\ifx\paragraph\undefined\else
  \let\oldparagraph\paragraph
  \renewcommand{\paragraph}{
    \@ifstar
      \xxxParagraphStar
      \xxxParagraphNoStar
  }
  \newcommand{\xxxParagraphStar}[1]{\oldparagraph*{#1}\mbox{}}
  \newcommand{\xxxParagraphNoStar}[1]{\oldparagraph{#1}\mbox{}}
\fi
\ifx\subparagraph\undefined\else
  \let\oldsubparagraph\subparagraph
  \renewcommand{\subparagraph}{
    \@ifstar
      \xxxSubParagraphStar
      \xxxSubParagraphNoStar
  }
  \newcommand{\xxxSubParagraphStar}[1]{\oldsubparagraph*{#1}\mbox{}}
  \newcommand{\xxxSubParagraphNoStar}[1]{\oldsubparagraph{#1}\mbox{}}
\fi
\makeatother

\usepackage{color}
\usepackage{fancyvrb}
\newcommand{\VerbBar}{|}
\newcommand{\VERB}{\Verb[commandchars=\\\{\}]}
\DefineVerbatimEnvironment{Highlighting}{Verbatim}{commandchars=\\\{\}}
% Add ',fontsize=\small' for more characters per line
\usepackage{framed}
\definecolor{shadecolor}{RGB}{241,243,245}
\newenvironment{Shaded}{\begin{snugshade}}{\end{snugshade}}
\newcommand{\AlertTok}[1]{\textcolor[rgb]{0.68,0.00,0.00}{#1}}
\newcommand{\AnnotationTok}[1]{\textcolor[rgb]{0.37,0.37,0.37}{#1}}
\newcommand{\AttributeTok}[1]{\textcolor[rgb]{0.40,0.45,0.13}{#1}}
\newcommand{\BaseNTok}[1]{\textcolor[rgb]{0.68,0.00,0.00}{#1}}
\newcommand{\BuiltInTok}[1]{\textcolor[rgb]{0.00,0.23,0.31}{#1}}
\newcommand{\CharTok}[1]{\textcolor[rgb]{0.13,0.47,0.30}{#1}}
\newcommand{\CommentTok}[1]{\textcolor[rgb]{0.37,0.37,0.37}{#1}}
\newcommand{\CommentVarTok}[1]{\textcolor[rgb]{0.37,0.37,0.37}{\textit{#1}}}
\newcommand{\ConstantTok}[1]{\textcolor[rgb]{0.56,0.35,0.01}{#1}}
\newcommand{\ControlFlowTok}[1]{\textcolor[rgb]{0.00,0.23,0.31}{\textbf{#1}}}
\newcommand{\DataTypeTok}[1]{\textcolor[rgb]{0.68,0.00,0.00}{#1}}
\newcommand{\DecValTok}[1]{\textcolor[rgb]{0.68,0.00,0.00}{#1}}
\newcommand{\DocumentationTok}[1]{\textcolor[rgb]{0.37,0.37,0.37}{\textit{#1}}}
\newcommand{\ErrorTok}[1]{\textcolor[rgb]{0.68,0.00,0.00}{#1}}
\newcommand{\ExtensionTok}[1]{\textcolor[rgb]{0.00,0.23,0.31}{#1}}
\newcommand{\FloatTok}[1]{\textcolor[rgb]{0.68,0.00,0.00}{#1}}
\newcommand{\FunctionTok}[1]{\textcolor[rgb]{0.28,0.35,0.67}{#1}}
\newcommand{\ImportTok}[1]{\textcolor[rgb]{0.00,0.46,0.62}{#1}}
\newcommand{\InformationTok}[1]{\textcolor[rgb]{0.37,0.37,0.37}{#1}}
\newcommand{\KeywordTok}[1]{\textcolor[rgb]{0.00,0.23,0.31}{\textbf{#1}}}
\newcommand{\NormalTok}[1]{\textcolor[rgb]{0.00,0.23,0.31}{#1}}
\newcommand{\OperatorTok}[1]{\textcolor[rgb]{0.37,0.37,0.37}{#1}}
\newcommand{\OtherTok}[1]{\textcolor[rgb]{0.00,0.23,0.31}{#1}}
\newcommand{\PreprocessorTok}[1]{\textcolor[rgb]{0.68,0.00,0.00}{#1}}
\newcommand{\RegionMarkerTok}[1]{\textcolor[rgb]{0.00,0.23,0.31}{#1}}
\newcommand{\SpecialCharTok}[1]{\textcolor[rgb]{0.37,0.37,0.37}{#1}}
\newcommand{\SpecialStringTok}[1]{\textcolor[rgb]{0.13,0.47,0.30}{#1}}
\newcommand{\StringTok}[1]{\textcolor[rgb]{0.13,0.47,0.30}{#1}}
\newcommand{\VariableTok}[1]{\textcolor[rgb]{0.07,0.07,0.07}{#1}}
\newcommand{\VerbatimStringTok}[1]{\textcolor[rgb]{0.13,0.47,0.30}{#1}}
\newcommand{\WarningTok}[1]{\textcolor[rgb]{0.37,0.37,0.37}{\textit{#1}}}

\providecommand{\tightlist}{%
  \setlength{\itemsep}{0pt}\setlength{\parskip}{0pt}}\usepackage{longtable,booktabs,array}
\usepackage{calc} % for calculating minipage widths
% Correct order of tables after \paragraph or \subparagraph
\usepackage{etoolbox}
\makeatletter
\patchcmd\longtable{\par}{\if@noskipsec\mbox{}\fi\par}{}{}
\makeatother
% Allow footnotes in longtable head/foot
\IfFileExists{footnotehyper.sty}{\usepackage{footnotehyper}}{\usepackage{footnote}}
\makesavenoteenv{longtable}
\usepackage{graphicx}
\makeatletter
\newsavebox\pandoc@box
\newcommand*\pandocbounded[1]{% scales image to fit in text height/width
  \sbox\pandoc@box{#1}%
  \Gscale@div\@tempa{\textheight}{\dimexpr\ht\pandoc@box+\dp\pandoc@box\relax}%
  \Gscale@div\@tempb{\linewidth}{\wd\pandoc@box}%
  \ifdim\@tempb\p@<\@tempa\p@\let\@tempa\@tempb\fi% select the smaller of both
  \ifdim\@tempa\p@<\p@\scalebox{\@tempa}{\usebox\pandoc@box}%
  \else\usebox{\pandoc@box}%
  \fi%
}
% Set default figure placement to htbp
\def\fps@figure{htbp}
\makeatother

% load packages
\usepackage{geometry}
\usepackage{xcolor}
\usepackage{eso-pic}
\usepackage{fancyhdr}
\usepackage{sectsty}
\usepackage{fontspec}
\usepackage{titlesec}

%% Set page size with a wider right margin
\geometry{a4paper, total={170mm,257mm}, left=20mm, top=20mm, bottom=20mm, right=50mm}

%% Let's define some colours
\definecolor{uniblue}{HTML}{003865}
\definecolor{burgundy}{HTML}{7D2239}
\definecolor{cobalt}{HTML}{005C8A}
\definecolor{lavender}{HTML}{5B4D94}
\definecolor{leaf}{HTML}{006630}
\definecolor{moss}{HTML}{385A4F}
\definecolor{pillarbox}{HTML}{B30C00}
\definecolor{rust}{HTML}{9A3A06}
\definecolor{sandstone}{HTML}{52473B}
\definecolor{skyblue}{HTML}{005398}
\definecolor{slate}{HTML}{4F5961}
\definecolor{thistle}{HTML}{951272}

%\definecolor{light}{HTML}{E6E6FA} % original from template - redefined below as uni blue at 10 percent:
\colorlet{light}{uniblue!10}
%\definecolor{highlight}{HTML}{800080} % original from template - redefined below as uni's skyblue:
\colorlet{highlight}{skyblue}
%\definecolor{dark}{HTML}{330033} % original from template - redefined below as uni blue at 100 percent:
\colorlet{dark}{uniblue}

%% Let's add the border on the right hand side 
\AddToShipoutPicture{% 
    \AtPageLowerLeft{% 
        \put(\LenToUnit{\dimexpr\paperwidth-3cm},0){% 
            \color{light}\rule{3cm}{\LenToUnit\paperheight}%
          }%
     }%
     % logo
    \AtPageLowerLeft{% start the bar at the bottom right of the page
        \put(\LenToUnit{\dimexpr\paperwidth-2.25cm},27.2cm){% move it to the top right
            \color{light}\includegraphics[width=2.25cm]{_extensions/nrennie/PrettyPDF/uni_logo_boxed.jpg}
          }%
     }%
}

%% Style the page number
\fancypagestyle{mystyle}{
  \fancyhf{}
  \renewcommand\headrulewidth{0pt}
  \fancyfoot[R]{\thepage}
  \fancyfootoffset{3.5cm}
}
\setlength{\footskip}{20pt}

%% style the chapter/section fonts
\chapterfont{\color{uniblue}\fontsize{20}{16.8}\selectfont}
\sectionfont{\color{uniblue}\fontsize{20}{16.8}\selectfont}
\subsectionfont{\color{skyblue}\fontsize{14}{16.8}\selectfont}
\titleformat{\subsection}
  {\color{uniblue!90}\sffamily\Large\bfseries}{\thesubsection}{1em}{}[{\titlerule[0.8pt]}]
\subsubsectionfont{\color{cobalt}}

\renewcommand\thesection{\color{slate}\arabic{section}}
  
% left align title
\makeatletter
\renewcommand{\maketitle}{\bgroup\setlength{\parindent}{0pt}
\begin{flushleft}
  {\color{uniblue}\sffamily\huge\textbf{\@title}} \vspace{0.3cm} \newline
  {\Large {\@subtitle}} \newline
  \@author
\end{flushleft}\egroup
}
\makeatother

%%% Use some custom fonts
\setsansfont{Ubuntu}[
    Path=_extensions/nrennie/PrettyPDF/Ubuntu/,
    Scale=0.9,
    Extension = .ttf,
    UprightFont=*-Regular,
    BoldFont=*-Bold,
    ItalicFont=*-Italic,
    ]

\setmainfont{Ubuntu}[
    Path=_extensions/nrennie/PrettyPDF/Ubuntu/,
    Scale=0.9,
    Extension = .ttf,
    UprightFont=*-Regular,
    BoldFont=*-Bold,
    ItalicFont=*-Italic,
    ]
\KOMAoption{captions}{tableheading}
\makeatletter
\@ifpackageloaded{tcolorbox}{}{\usepackage[skins,breakable]{tcolorbox}}
\@ifpackageloaded{fontawesome5}{}{\usepackage{fontawesome5}}
\definecolor{quarto-callout-color}{HTML}{909090}
\definecolor{quarto-callout-note-color}{HTML}{0758E5}
\definecolor{quarto-callout-important-color}{HTML}{CC1914}
\definecolor{quarto-callout-warning-color}{HTML}{EB9113}
\definecolor{quarto-callout-tip-color}{HTML}{00A047}
\definecolor{quarto-callout-caution-color}{HTML}{FC5300}
\definecolor{quarto-callout-color-frame}{HTML}{acacac}
\definecolor{quarto-callout-note-color-frame}{HTML}{4582ec}
\definecolor{quarto-callout-important-color-frame}{HTML}{d9534f}
\definecolor{quarto-callout-warning-color-frame}{HTML}{f0ad4e}
\definecolor{quarto-callout-tip-color-frame}{HTML}{02b875}
\definecolor{quarto-callout-caution-color-frame}{HTML}{fd7e14}
\makeatother
\makeatletter
\@ifpackageloaded{caption}{}{\usepackage{caption}}
\AtBeginDocument{%
\ifdefined\contentsname
  \renewcommand*\contentsname{Table of contents}
\else
  \newcommand\contentsname{Table of contents}
\fi
\ifdefined\listfigurename
  \renewcommand*\listfigurename{List of Figures}
\else
  \newcommand\listfigurename{List of Figures}
\fi
\ifdefined\listtablename
  \renewcommand*\listtablename{List of Tables}
\else
  \newcommand\listtablename{List of Tables}
\fi
\ifdefined\figurename
  \renewcommand*\figurename{Figure}
\else
  \newcommand\figurename{Figure}
\fi
\ifdefined\tablename
  \renewcommand*\tablename{Table}
\else
  \newcommand\tablename{Table}
\fi
}
\@ifpackageloaded{float}{}{\usepackage{float}}
\floatstyle{ruled}
\@ifundefined{c@chapter}{\newfloat{codelisting}{h}{lop}}{\newfloat{codelisting}{h}{lop}[chapter]}
\floatname{codelisting}{Listing}
\newcommand*\listoflistings{\listof{codelisting}{List of Listings}}
\makeatother
\makeatletter
\makeatother
\makeatletter
\@ifpackageloaded{caption}{}{\usepackage{caption}}
\@ifpackageloaded{subcaption}{}{\usepackage{subcaption}}
\makeatother
\makeatletter
\@ifpackageloaded{tcolorbox}{}{\usepackage[skins,breakable]{tcolorbox}}
\makeatother
\makeatletter
\@ifundefined{shadecolor}{\definecolor{shadecolor}{rgb}{.97, .97, .97}}{}
\makeatother
\makeatletter
\@ifundefined{codebgcolor}{\definecolor{codebgcolor}{named}{light}}{}
\makeatother
\makeatletter
\ifdefined\Shaded\renewenvironment{Shaded}{\begin{tcolorbox}[sharp corners, boxrule=0pt, breakable, colback={codebgcolor}, enhanced, frame hidden]}{\end{tcolorbox}}\fi
\makeatother

\usepackage{bookmark}

\IfFileExists{xurl.sty}{\usepackage{xurl}}{} % add URL line breaks if available
\urlstyle{same} % disable monospaced font for URLs
\hypersetup{
  pdftitle={Tutorial Sheet 1},
  colorlinks=true,
  linkcolor={highlight},
  filecolor={Maroon},
  citecolor={Blue},
  urlcolor={highlight},
  pdfcreator={LaTeX via pandoc}}


\title{Tutorial Sheet 1}
\author{}
\date{}

\begin{document}
\maketitle

\pagestyle{mystyle}


\section{Part A: Censoring and uncertainty
calculations.}\label{part-a-censoring-and-uncertainty-calculations.}

\subsection*{Task 1}\label{task-1}
\addcontentsline{toc}{subsection}{Task 1}

Several methods for dealing with values marked as being at the limit of
detection within the paper by Eastoe et al (2006). Read the paper
(available below), summarise the different methods compared by the
authors and comment on what the conclusions were.

\subsection*{Task 2}\label{task-2}
\addcontentsline{toc}{subsection}{Task 2}

The Shannon index (or Shannon-Wiener diversity index) is widely used in
Ecology to quantify the diversity of a biological community by
considering both species richness and evenness. It is calculated as:

\[
H = -\sum_{i=1}^S p_i \log (p_i)
\]

where \(p_i\) is the proportion of species \(i\) in the community
computed as the ratio between the num. of individuals of a given species
\(n_i\) and total number of individual across all species \(N\).

\begin{enumerate}
\def\labelenumi{\arabic{enumi}.}
\item
  Suppose a fixed number of individuals \(N\) are sampled and that the
  proportion of each species is estimated with some uncertainty
  \(u(p_i)\). Provide the general form for the uncertainty propagation
  of these proportions on the calculation of \(H\).
\item
  Imagine you go to your garden an find out there are \(S=3\) different
  species of arthropods living there. Then you go out one day and sample
  \(N=100\) individuals and end up collecting \(n_1 = 50 \text{ ants}\),
  \(n_2 = 30 \text{ beetles}\) and \(n_3 = 20 \text{ spiders}\).
  Assuming that number of individuals of a given species follows
  \(n_i \sim \text{Binomial}(N,\theta_i)\), and let
  \(\hat{\theta_i} = \frac{n_i}{N} = p_i\) be the estimator of
  \(\theta_i\) show that \(u(p_i)^2 = p_i(1-p_i)/N\) and then compute
  the uncertainty propagation for the Shannon Index.
\end{enumerate}

\subsection*{Task 3}\label{task-3}
\addcontentsline{toc}{subsection}{Task 3}

Waves have a major influence on the marine environment and ultimately on
the planet's climate and so are often studied by oceanographers. The
period of a particular wave oscillation is measured to be
\(T=(2\pm0.1)s\). What is the uncertainty associated with the frequency,
\(f\), of the wave, where \(f=1/T\)?

\subsection*{Task 4}\label{task-4}
\addcontentsline{toc}{subsection}{Task 4}

10 sets of data (N=13) have been collected. Within each there is a
number of (suspected) outliers.

\begin{verbatim}
     mean median    sd   MAD Nout?
s1  23.82  10.31 33.21 13.98     2
s2  17.72  10.90 24.88  7.64     1
s3  16.58   9.83 25.05  7.52     1
s4  24.13  10.63 33.56 14.13     2
s5  30.68  10.24 39.37 21.07     3
s6  30.82  10.64 39.04 21.08     3
s7  23.79  10.01 33.91 14.47     2
s8  23.95  10.05 34.01 14.39     2
s9  30.96  10.50 39.40 21.33     3
s10 24.31  10.52 33.79 14.32     2
\end{verbatim}

The data for the first sample are shown below:

\begin{verbatim}
9.89, 10.55, 9.67, 10.62, 10.31, 10.21, 10.92, 98.25, 99.03, 9.33, 11.17, 9.78, 9.90
\end{verbatim}

The R code/output below shows the results of various outlier tests.
Comment on this output.

\begin{Shaded}
\begin{Highlighting}[]
\CommentTok{\# Chauvenet’s test for outliers}

\NormalTok{P }\OtherTok{\textless{}{-}} \DecValTok{1}\SpecialCharTok{{-}}\FunctionTok{pnorm}\NormalTok{(s1[}\DecValTok{9}\NormalTok{],}\FunctionTok{mean}\NormalTok{(s1), }\FunctionTok{sd}\NormalTok{(s1))}
\NormalTok{P}\SpecialCharTok{*}\FunctionTok{length}\NormalTok{(s1)}
\NormalTok{[}\DecValTok{1}\NormalTok{] }\FloatTok{0.152972}

\CommentTok{\# Grubbs test for one outlier}

\NormalTok{data}\SpecialCharTok{:}\NormalTok{  s1}
\NormalTok{G }\OtherTok{=} \FloatTok{2.2647}\NormalTok{, U }\OtherTok{=} \FloatTok{0.5370}\NormalTok{, p}\SpecialCharTok{{-}}\NormalTok{value }\OtherTok{=} \FloatTok{0.06811}
\NormalTok{alternative hypothesis}\SpecialCharTok{:}\NormalTok{ highest value }\FloatTok{99.03}\NormalTok{ is an outlier}

\CommentTok{\# Dixon test for outliers}

\NormalTok{data}\SpecialCharTok{:}\NormalTok{  s1}
\NormalTok{Q }\OtherTok{=} \FloatTok{0.98321}\NormalTok{, p}\SpecialCharTok{{-}}\NormalTok{value }\SpecialCharTok{\textless{}} \FloatTok{2.2e{-}16}
\NormalTok{alternative hypothesis}\SpecialCharTok{:}\NormalTok{ highest value }\FloatTok{99.03}\NormalTok{ is an outlier}
\end{Highlighting}
\end{Shaded}

\subsection*{Task 5}\label{task-5}
\addcontentsline{toc}{subsection}{Task 5}

\begin{enumerate}
\def\labelenumi{(\alph{enumi})}
\item
  An ecologist wishes to analyse a dataset that contains a variable with
  around 1\% of its data censored at a limit of detection. The ecologist
  proposes to use a simple substitution method to replace all values at
  the limit of detection (\(c_L\)) with 0.5\(c_L\). Do you agree with
  this approach? Why/ why not?
\item
  The ecologist wishes to analyse another dataset that contains a
  variable with around 55\% of the data censored at a limit of
  detection. The ecologist would like to take the same simple
  substitution approach as in part (a). Do you agree with this approach?
  What advice would you give?
\end{enumerate}

\subsection*{Task 6}\label{task-6}
\addcontentsline{toc}{subsection}{Task 6}

\begin{enumerate}
\def\labelenumi{(\alph{enumi})}
\item
  Suppose that chlorophyll data in a freshwater loch are collected twice
  a week, but the equipment needs to be removed for maintenance once a
  month, so that there are around 12 missing values per year. Can we
  assume that these values are missing at random?
\item
  Suppose that in another loch, the data are also collected twice a
  week, but the monitoring device there only needs to be maintained once
  a year. If this is removed every December, so that there are no values
  for that month, can we assume that these data are missing at random?
  Might we need to impute the data?
\end{enumerate}

\section{Part B: Sampling and
monitoring}\label{part-b-sampling-and-monitoring}

\begin{tcolorbox}[enhanced jigsaw, left=2mm, coltitle=black, colback=white, titlerule=0mm, toprule=.15mm, leftrule=.75mm, colframe=quarto-callout-note-color-frame, bottomrule=.15mm, title=\textcolor{quarto-callout-note-color}{\faInfo}\hspace{0.5em}{Note}, colbacktitle=quarto-callout-note-color!10!white, rightrule=.15mm, opacitybacktitle=0.6, breakable, opacityback=0, toptitle=1mm, arc=.35mm, bottomtitle=1mm]

This part of the tutorial sheet relates to material in Week 3, which
we'll cover in the lectures during the same week as Tutorial 1. You can
attempt these questions in Tutorial 1 if you wish, but you may prefer to
go through these during Tutorial 2 instead.

\end{tcolorbox}

\subsection*{Task 7}\label{task-7}
\addcontentsline{toc}{subsection}{Task 7}

In the case of a simple random sample \(x_1,..., x_n\) of a random
variable, \(X\), assuming the observations are independent,

\begin{enumerate}
\def\labelenumi{(\alph{enumi})}
\tightlist
\item
  derive the expected value of \(X^2\)
\item
  derive the expected value of \(\bar{X}^2\)
\item
  show that the sample variance, \(s^2\), is an unbiased estimator of
  the population variance, \(\sigma^2\).
\end{enumerate}

\subsection*{Task 8}\label{task-8}
\addcontentsline{toc}{subsection}{Task 8}

\href{https://go.exlibris.link/DnZPRrww}{Gilbert (1977)} reports results
of soil sampling at a nuclear weapons test area obtained using
stratified random sampling to assess the total amount of Plutonium found
in surface soil. Use the information in the table below to:

\begin{enumerate}
\def\labelenumi{(\alph{enumi})}
\item
  Estimate the total inventory and derive the estimator for the variance
  of the totals
\item
  Determine the optimal number of population units of measure in each of
  the 4 strata. Find the total number of units to sample, assuming cost
  is fixed (where total=£50,000 and cost per unit is £500 for all
  strata).
\end{enumerate}

\begin{longtable}[]{@{}
  >{\raggedright\arraybackslash}p{(\linewidth - 8\tabcolsep) * \real{0.2000}}
  >{\raggedright\arraybackslash}p{(\linewidth - 8\tabcolsep) * \real{0.2000}}
  >{\raggedright\arraybackslash}p{(\linewidth - 8\tabcolsep) * \real{0.2000}}
  >{\raggedright\arraybackslash}p{(\linewidth - 8\tabcolsep) * \real{0.2000}}
  >{\raggedright\arraybackslash}p{(\linewidth - 8\tabcolsep) * \real{0.2000}}@{}}
\toprule\noalign{}
\begin{minipage}[b]{\linewidth}\raggedright
strata
\end{minipage} & \begin{minipage}[b]{\linewidth}\raggedright
Size \(\times\) area of the stratum \(N_l\)
\end{minipage} & \begin{minipage}[b]{\linewidth}\raggedright
\(n_l\)
\end{minipage} & \begin{minipage}[b]{\linewidth}\raggedright
Mean for stratum
\end{minipage} & \begin{minipage}[b]{\linewidth}\raggedright
Variance \(s^2_l\)
\end{minipage} \\
\midrule\noalign{}
\endhead
\bottomrule\noalign{}
\endlastfoot
1 & 351,000 & 18 & 4.1 & 30.42 \\
2 & 82,300 & 12 & 73 & 10,800 \\
3 & 26,200 & 13 & 270 & 127,413 \\
4 & 11,000 & 20 & 260 & 84,500 \\
\end{longtable}

\begin{tcolorbox}[enhanced jigsaw, left=2mm, coltitle=black, colback=white, titlerule=0mm, toprule=.15mm, leftrule=.75mm, colframe=quarto-callout-tip-color-frame, bottomrule=.15mm, title=\textcolor{quarto-callout-tip-color}{\faLightbulb}\hspace{0.5em}{Tip}, colbacktitle=quarto-callout-tip-color!10!white, rightrule=.15mm, opacitybacktitle=0.6, breakable, opacityback=0, toptitle=1mm, arc=.35mm, bottomtitle=1mm]

The stratified estimator of the population total is

\[
\hat{I} = \sum_{l=1}^{L} N_l \, \bar{y}_l .
\] You may use the fact that the variance for the mean of \(l\)-th
strata is given by

\[
\text{Var}(\bar{y}_l)= \left( 1 - \frac{n_l}{N_l} \right)\frac{s_l^2}{n_l}
\]

\end{tcolorbox}

\subsection*{Task 9}\label{task-9}
\addcontentsline{toc}{subsection}{Task 9}

Discuss the advantages and disadvantages of the three sampling methods
below for mapping a pollutant field:

\begin{itemize}
\tightlist
\item
  Simple random sampling
\item
  Systematic sampling
\item
  Stratified random sampling
\end{itemize}

\subsection*{Task 10}\label{task-10}
\addcontentsline{toc}{subsection}{Task 10}

The Water Framework Directive states:

\emph{``Member states must ensure that enough individual water bodies of
each water type are monitored and determine how many stations are
required to determine the ecological and chemical status of the water
body''}

Discuss briefly how you would translate this statement into a monitoring
programme, given that there are 6 different water body types comprising
10\%, 25\%, 30\%, 20\%, 10\% and 5\% of the total population of 6600
water bodies and that your limited resources only allow you to study a
total of 200 water bodies. Knowledge of the within-type variability is
not available.

\subsection*{Task 11}\label{task-11}
\addcontentsline{toc}{subsection}{Task 11}

Read pages 17--23 of the Analytical Laboratories for the Measurement of
Environmental Radioactivity (ALMERA) report on soil sampling (available
below). Describe briefly the sampling strategy adopted and also the
methods of analysis presented.

\subsection*{Task 12}\label{task-12}
\addcontentsline{toc}{subsection}{Task 12}

Read the SEPA survey of business waste (available below) and describe
briefly the sampling strategy you would propose. Discuss its advantages
and disadvantages.




\end{document}
